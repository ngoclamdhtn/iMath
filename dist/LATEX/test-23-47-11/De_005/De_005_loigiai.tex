\documentclass[12pt,a4paper]{article}
\usepackage[top=1.5cm, bottom=1.5cm, left=2.0cm, right=1.5cm] {geometry}
\usepackage{amsmath,amssymb,txfonts}
\usepackage{tkz-euclide}
\usepackage{setspace}
\usepackage{lastpage}

\usepackage{tikz,tkz-tab}
%\usepackage[solcolor]{ex_test}
%\usepackage[dethi]{ex_test} % Chỉ hiển thị đề thi
\usepackage[loigiai]{ex_test} % Hiển thị lời giải
%\usepackage[color]{ex_test} % Khoanh các đáp án
\everymath{\displaystyle}

\def\colorEX{\color{purple}}
%\def\colorEX{}%Không tô màu đáp án đúng trong tùy chọn loigiai
\renewtheorem{ex}{\color{violet}Câu}
\renewcommand{\FalseEX}{\stepcounter{dapan}{{\bf \textcolor{blue}{\Alph{dapan}.}}}}
\renewcommand{\TrueEX}{\stepcounter{dapan}{{\bf \textcolor{blue}{\Alph{dapan}.}}}}

%---------- Khai báo viết tắt, in đáp án
\newcommand{\hoac}[1]{ %hệ hoặc
    \left[\begin{aligned}#1\end{aligned}\right.}
\newcommand{\heva}[1]{ %hệ và
    \left\{\begin{aligned}#1\end{aligned}\right.}

%Tiêu đề
\newcommand{\tenso}{iMath}
\newcommand{\tentruong}{Phần mềm tạo đề tự động}
\newcommand{\tenkythi}{ĐỀ ÔN TẬP TOÁN 12}
\newcommand{\tenmonthi}{Môn thi: Toán}
\newcommand{\thoigian}{}
\newcommand{\tieude}[2]{
    \noindent
    %Trái
    \begin{minipage}[b]{7cm}
        \centerline{\textbf{\fontsize{13}{0}\selectfont \tenso}}
        \centerline{\textbf{\fontsize{13}{0}\selectfont \tentruong}}
        \centerline{(\textit{Đề thi có #1\ trang})}
    \end{minipage}\hspace{1.5cm}
    %Phải
    \begin{minipage}[b]{9cm}
        \centerline{\textbf{\fontsize{13}{0}\selectfont \tenkythi}}
        \centerline{\textbf{\fontsize{13}{0}\selectfont \tenmonthi}}
        \centerline{\textit{\fontsize{12}{0}\selectfont Thời gian làm bài: \thoigian\ phút}}
    \end{minipage}
    \begin{minipage}[b]{10cm}
        \textbf{Họ và tên HS: }{\tiny\dotfill}
    \end{minipage}
    \begin{minipage}[b]{8cm}
        \hspace*{4cm}\fbox{\bf Mã đề: #2}
    \end{minipage}\vspace{3pt}
}
\newcommand{\chantrang}[2]{\rfoot{Trang \thepage $-$ Mã đề #2}}
\pagestyle{fancy}
\fancyhf{}
\begin{document}
%Thiết lập giãn dọng 1.5cm 
%\setlength{\lineskip}{1.5em}
%Nội dung trắc nghiệm bắt đầu ở đây


\tieude{\pageref{LastPage}}{005}

\chantrang{\pageref{LastPage}}{005}

\setcounter{page}{1}

\setcounter{ex}{0}
\Opensolutionfile{ans}[ans/ans005]
\begin{ex}
 Trong mặt phẳng ${Oxy}$, phương trình đường tròn ${(C)}$ có tâm ${B(-8;5)}$ và đi qua điểm $E(6;-10)$ là
 
\choice
{ $\left(x - 6\right)^{2}+\left(y + 10\right)^{2}=421$ }
   { \True $\left(x + 8\right)^{2}+\left(y - 5\right)^{2}=421$ }
     { $\left(x + 6\right)^{2}+\left(y - 10\right)^{2}=421$ }
    { $\left(x - 8\right)^{2}+\left(y + 5\right)^{2}=421$ }
\loigiai{ 

 Đường tròn ${(C)}$ có bán kính là ${BE}=\sqrt{(6-(-8))^2+(-10-5)^2}=\sqrt{421}$.

Đường tròn ${(C)}$ có phương trình là: $\left(x + 8\right)^{2}+\left(y - 5\right)^{2}=421$. 
 }\end{ex}

\begin{ex}
 Trong mặt phẳng ${Oxy}$, cho đường thẳng $\Delta: x - 4 y - 1=0$ và điểm ${I(9;5)}$. Đường tròn ${(C)}$ có tâm ${I}$ và tiếp xúc với đường thẳng $\Delta$ có phương trình là
 
\choice
{ $\left(x - 9\right)^{2}+\left(y - 5\right)^{2}=144$ }
   { $\left(x - 9\right)^{2}+\left(y - 5\right)^{2}=\frac{12 \sqrt{17}}{17}$ }
     { \True $\left(x - 9\right)^{2}+\left(y - 5\right)^{2}=\frac{144}{17}$ }
    { $\left(x + 9\right)^{2}+\left(y + 5\right)^{2}=\frac{144}{17}$ }
\loigiai{ 

 Đường tròn ${(C)}$ có bán kính là: $R=d(I,\Delta)=\dfrac{|1.9+(-4).5-1|}{ \sqrt{1+16} }=\frac{12 \sqrt{17}}{17}$.

Đường tròn ${(C)}$ có phương trình là: $\left(x - 9\right)^{2}+\left(y - 5\right)^{2}=\frac{144}{17}$. 
 }\end{ex}

\begin{ex}
 Trong không gian ${Oxyz}$, cho vectơ $\overrightarrow{b}=(-3;-7;-4)$. Độ dài vectơ $\overrightarrow{b}$ bằng.\\ 
\choice
{ ${14}$ }
   { ${75}$ }
     { ${74}$ }
    { \True ${\sqrt{74}}$ }
\loigiai{ 
 $|\overrightarrow{b}|=\sqrt{9+49+16}=\sqrt{74}$. 
 }\end{ex}

\begin{ex}
 Trong hệ trục tọa độ ${Oxyz}$, cho hai véctơ $\overrightarrow{d}(2;-9;4)$ và $\overrightarrow{b}(3;-10;4)$. Tọa độ tích có hướng $\left[\overrightarrow{d},\overrightarrow{b}\right]$ là\\ 
\choice
{ $\left[\overrightarrow{d},\overrightarrow{b}\right]=(0;6;5)$ }
   { $\left[\overrightarrow{d},\overrightarrow{b}\right]=(4;8;6)$ }
     { $\left[\overrightarrow{d},\overrightarrow{b}\right]=(8;2;9)$ }
    { \True $\left[\overrightarrow{d},\overrightarrow{b}\right]= (4;4;7)$ }
\loigiai{ 
 $\left[\overrightarrow{d}.\overrightarrow{b}\right]=(4;4;7)$. 
 }\end{ex}

\begin{ex}
 Trong không gian ${Oxyz}$, cho mặt phẳng ${(Q)}$ có phương trình $x + z - 4=0$. Mặt phẳng ${(Q)}$ nhận vectơ nào trong các vectơ sau làm véctơ pháp tuyến.
 
\choice
{ \True $\overrightarrow{n_2}=(-1;0;-1)$ }
   { $\overrightarrow{n_2}=(1;0;-4)$ }
     { $\overrightarrow{n_2}=(1;0;-4)$ }
    { $\overrightarrow{n_2}=(-1;0;1)$ }
\loigiai{ 

 Mặt phẳng ${(Q)}$ có một véctơ pháp tuyến là $\overrightarrow{n}=(1;0;1)$.

 Do đó, mặt phẳng ${(Q)}$ cũng nhận vectơ $\overrightarrow{n_2}=(-1;0;-1)$ làm véctơ pháp tuyến. 
 }\end{ex}

\begin{ex}
 Trong không gian ${Oxyz}$, cho mặt phẳng ${(Q)}$ có phương trình $- 3 x + 27 y - 14 z + 107=0$. Điểm nào trong các điểm sau thuộc mặt phẳng ${(Q)}$?
 
\choice
{ \True $A(-4;-7;-5)$ }
   { $E(-1;-2;-4)$ }
     { $D(-5;-1;-1)$ }
    { $C(-8;1;4)$ }
\loigiai{ 

 Thay tọa độ các điểm vào phương trình mặt phẳng ${(Q)}$ta thấy chỉ có điểm $A(-4;-7;-5)$ thỏa mãn.
 
 }\end{ex}

\begin{ex}
 Trong không gian ${Oxyz}$, cho mặt phẳng ${(R)}$ có phương trình $35 x - 20 y - 23 z + 104=0$. Điểm nào trong các điểm sau không thuộc mặt phẳng ${(R)}$?
 
\choice
{ $B(-6;-3;-2)$ }
   { \True $M(0;1;-2)$ }
     { $I(-7;1;-7)$ }
    { $K(-1;0;3)$ }
\loigiai{ 

 Thay tọa độ các điểm vào phương trình mặt phẳng ${(R)}$ta thấy điểm $M(-1;0;3)$ không thỏa mãn phương trình nên điểm ${M}$ không thuộc mặt phẳng ${(R)}$.
 
 }\end{ex}

\begin{ex}
 Trong không gian ${Oxyz}$, đường thẳng ${d}$ đi qua điểm ${N(-6;-2;7)}$ và nhận vectơ $\vec{u}=(-1;10;-8)$ làm véctơ chỉ phương có phương trình là
 
\choice
{ $\left\{ \begin{array}{l}x = -6-t\\ y = 2-10t\\z = 7-8t\end{array} \right.$ }
   { $\left\{ \begin{array}{l}x = -1-6t\\ y = 10-2t\\z = -8+7t\end{array} \right.$ }
     { $\left\{ \begin{array}{l}x = 6-t\\ y = 2+10t\\z = -7-8t\end{array} \right.$ }
    { \True $\left\{ \begin{array}{l}x = -6-t\\ y = -2+10t\\z = 7-8t\end{array} \right.$ }
\loigiai{ 

 Đường thẳng ${d}$ đi qua điểm ${N(-6;-2;7)}$ nhận vectơ $\vec{u}=(-1;10;-8)$ làm véctơ chỉ phương có phương có phương trình là: $\left\{ \begin{array}{l}x = -6-t\\ y = -2+10t\\z = 7-8t\end{array} \right.$. 
 }\end{ex}

\begin{ex}
 Trong không gian ${Oxyz}$, đường thẳng ${\Delta}$ đi qua điểm ${A(5;5;-7)}$ và nhận vectơ $\overrightarrow{HC}$ làm véctơ chỉ phương với $H(-7;-7;-4)$ và $C(-8;-12;-7)$ có phương trình là
 
\choice
{ $\left\{ \begin{array}{l}x = -5-t\\ y = -5-5t\\z = 7-3t\end{array} \right.$ }
   { \True $\left\{ \begin{array}{l}x = 5-t\\ y = 5-5t\\z = -7-3t\end{array} \right.$ }
     { $\left\{ \begin{array}{l}x = -1+5t\\ y = -5+5t\\z = -3-7t\end{array} \right.$ }
    { $\left\{ \begin{array}{l}x = 5-t\\ y = -5+5t\\z = -7-3t\end{array} \right.$ }
\loigiai{ 

 Ta có: $\overrightarrow{HC}=(-1;-5;-3)$.

Đường thẳng ${\Delta}$ đi qua điểm ${A(5;5;-7)}$ nhận vectơ $\overrightarrow{HC}=(-1;-5;-3)$ làm véctơ chỉ phương có phương trình là: $\left\{ \begin{array}{l}x = 5-t\\ y = 5-5t\\z = -7-3t\end{array} \right.$. 
 }\end{ex}

\begin{ex}
 Trong không gian ${Oxyz}$, đường thẳng ${d}$ đi qua điểm ${C(2;-5;-4)}$ và song song với đường thẳng $\Delta_1:\dfrac{x}{2}=\dfrac{y + 3}{10}=\dfrac{z + 6}{-12}$ có phương trình là
 
\choice
{ $\left\{ \begin{array}{l}x = -2+t\\ y = 5+5t\\z = 4-6t\end{array} \right.$ }
   { $\left\{ \begin{array}{l}x = 1+2t\\ y = 5-5t\\z = -6-4t\end{array} \right.$ }
     { \True $\left\{ \begin{array}{l}x = 2+t\\ y = -5+5t\\z = -4-6t\end{array} \right.$ }
    { $\left\{ \begin{array}{l}x = 2+t\\ y = 5-5t\\z = -4-6t\end{array} \right.$ }
\loigiai{ 

 Đường thẳng ${\Delta_1}$ có véctơ chỉ phương là $\overrightarrow{u_1}=(2;10;-12)$.

Đường thẳng ${d}$ song song với ${\Delta_1}$ nên có một véctơ chỉ phương là vectơ $\vec{u}=(1;5;-6)$.

Đường thẳng ${d}$ đi qua điểm ${C(2;-5;-4)}$ nhận vectơ $\vec{u}=(1;5;-6)$ làm véctơ chỉ phương có phương có phương trình là: $\left\{ \begin{array}{l}x = 2+t\\ y = -5+5t\\z = -4-6t\end{array} \right.$. 
 }\end{ex}

\begin{ex}
 Trong không gian ${Oxyz}$, cho đường thẳng ${\Delta}: \dfrac{x + 7}{-3}=\dfrac{y - 4}{2}=\dfrac{z - 3}{5}$. Phương trình tham số của đường thẳng ${\Delta}$ là
 
\choice
{ $\left\{ \begin{array}{l}x = 7-3t\\ y = -4+2t\\z = -3+5t\end{array} \right.$ }
   { $\left\{ \begin{array}{l}x = -7-3t\\ y = -4-2t\\z = 3+5t\end{array} \right.$ }
     { $\left\{ \begin{array}{l}x = -3-7t\\ y = 2+4t\\z = 5+3t\end{array} \right.$ }
    { \True $\left\{ \begin{array}{l}x = -7-3t\\ y = 4+2t\\z = 3+5t\end{array} \right.$ }
\loigiai{ 

 Đường thẳng ${\Delta}$ đi qua điểm ${B(-7;4;3)}$ nhận vectơ $\vec{u}=(-3;2;5)$ làm véctơ chỉ phương có phương có phương trình là: $\left\{ \begin{array}{l}x = -7-3t\\ y = 4+2t\\z = 3+5t\end{array} \right.$. 
 }\end{ex}

\begin{ex}
 Trong không gian ${Oxyz}$, cho đường thẳng ${\Delta}:\left\{ \begin{array}{l}x = -4+7t\\ y = -7+7t\\z = -7+5t\end{array} \right.$. Phương trình chính tắc của đường thẳng ${\Delta}$ là
 
\choice
{ \True $\dfrac{x + 4}{7}=\dfrac{y + 7}{7}=\dfrac{z + 7}{5}$ }
   { $\dfrac{x - 7}{-4}=\dfrac{y - 7}{-7}=\dfrac{z - 5}{-7}$ }
     { $\dfrac{x - 4}{7}=\dfrac{y - 7}{7}=\dfrac{z - 7}{5}$ }
    { $\dfrac{x + 7}{-4}=\dfrac{y + 7}{-7}=\dfrac{z + 5}{-7}$ }
\loigiai{ 

 Đường thẳng ${\Delta}$ đi qua điểm ${A(-4;-7;-7)}$ nhận vectơ $\vec{u}=(7;7;5)$ làm véctơ chỉ phương có phương có phương trình là: $\dfrac{x + 4}{7}=\dfrac{y + 7}{7}=\dfrac{z + 7}{5}$. 
 }\end{ex}

\begin{ex}
 Trong không gian ${Oxyz}$, cho đường thẳng ${d}:\left\{ \begin{array}{l}x = -2-5t\\ y = 7+9t\\z = 6+6t\end{array} \right.$. Đường thẳng ${d}$ nhận vectơ nào sau đây làm véctơ chỉ phương?
 
\choice
{ $\overrightarrow{u_1}=(5;9;-6)$ }
   { $\overrightarrow{u_4}=(2;-7;-6)$ }
     { \True $\overrightarrow{u_2}=(10;-18;-12)$ }
    { $\overrightarrow{u_3}=(-2;7;6)$ }
\loigiai{ 

 Đường thẳng ${d}$ nhận vectơ $\vec{u}=(-5;9;6)$ làm véctơ chỉ phương nên cũng nhận vectơ $\overrightarrow{u_2}=(10;-18;-12)$ làm véctơ chỉ phương. 
 }\end{ex}

\begin{ex}
 Trong không gian ${Oxyz}$, cho đường thẳng ${d}:\left\{ \begin{array}{l}x = -8-3t\\ y = 6-8t\\z = 3+6t\end{array} \right.$. Đường thẳng ${d}$ đi qua điểm nào trong các điểm sau?
 
\choice
{ $C=(-11;-15;-6)$ }
   { \True $D=(-14;-10;15)$ }
     { $B=(8;-6;-3)$ }
    { $A=(-3;-8;6)$ }
\loigiai{ 

 Tồn tại $t=2 \Rightarrow x=-14,y=-10,z=15$ nên đường thẳng ${d}$ đi qua $D=(-14;-10;15)$. 
 }\end{ex}

\begin{ex}
 Trong không gian ${Oxyz}$, cho đường thẳng ${\Delta}:\dfrac{x + 3}{2}=\dfrac{y + 5}{1}=\dfrac{z - 5}{-9}$. Đường thẳng ${\Delta}$ nhận vectơ nào sau đây làm véctơ chỉ phương?
 
\choice
{ $\overrightarrow{u_4}=(-2;1;9)$ }
   { \True $\overrightarrow{u_3}=(-4;-2;18)$ }
     { $\overrightarrow{u_2}=(-3;-5;5)$ }
    { $\overrightarrow{u_1}=(3;5;-5)$ }
\loigiai{ 

 Đường thẳng ${\Delta}$ nhận vectơ $\vec{u}=(2;1;-9)$ làm véctơ chỉ phương nên cũng nhận vectơ $\overrightarrow{u_3}=(-4;-2;18)$ làm véctơ chỉ phương. 
 }\end{ex}

\begin{ex}
 Trong không gian ${Oxyz}$, cho đường thẳng ${d}:\dfrac{x - 5}{-3}=\dfrac{y - 2}{-9}=\dfrac{z - 7}{8}$. Đường thẳng ${d}$ đi qua điểm nào trong các điểm sau?
 
\choice
{ $D=(-3;-9;8)$ }
   { $B=(13;18;-8)$ }
     { \True $A=(11;20;-9)$ }
    { $C=(-5;-2;-7)$ }
\loigiai{ 

 Đường thẳng ${d}$ có phương trình tham số là $\left\{ \begin{array}{l}x = 5-3t\\ y = 2-9t\\z = 7+8t\end{array} \right.$.

Tồn tại $t=-2 \Rightarrow x=11,y=20,z=-9$ nên đường thẳng ${d}$ đi qua $A=(11;20;-9)$. 
 }\end{ex}

\begin{ex}
 Trong không gian ${Oxyz}$, tọa độ giao điểm của đường thẳng ${d}:\dfrac{x}{-2}=\dfrac{y - 9}{1}=\dfrac{z - 23}{7}$ và mặt phẳng $(R):- 2 x - 6 y - 2 z + 36=0$ là điểm $H(a;b;c)$. Tính $P=a+b+c$.
 
\choice
{ \True ${8}$ }
   { ${18}$ }
     { ${38}$ }
    { ${-7}$ }
\loigiai{ 

 Đường thẳng ${d}$ có phương trình tham số là $\left\{ \begin{array}{l}x = -2t\\ y = 9+t\\z = 23+7t\end{array} \right.$.

Xét phương trình $-2(- 2 t)-6(t + 9)-2(7 t + 23)+36=0\Rightarrow t=-4$.

Tọa độ giao điểm của ${d}$ và ${(R)}$ là $H(8;5;-5)$.

 Vậy $P=8+5-5=8$. 
 }\end{ex}

\begin{ex}
 Trong không gian ${Oxyz}$, cho đường thẳng ${\Delta}:\dfrac{x - 6}{4}=\dfrac{y + 3}{-2}=\dfrac{z - 1}{3}$ và điểm $A(-2;-6;1)$.

 Hình chiếu vuông góc của điểm $A$ trên đường thẳng ${\Delta}$ là điểm $H(a;b;c)$. Tính $P=a+b+c$.
 
\choice
{ ${\frac{182}{29}}$ }
   { ${\frac{7}{29}}$ }
     { ${\frac{196}{29}}$ }
    { \True ${- \frac{14}{29}}$ }
\loigiai{ 

 Đường thẳng ${\Delta}$ có véctơ chỉ phương là $\vec{u}=(4;-2;3)$.

Gọi $H(6+4t;-3-2t;1+3t)$.

$\overrightarrow{AH}=(4 t + 8;3 - 2 t;3 t)$.

$\overrightarrow{AH}.\overrightarrow{u}=0\Leftrightarrow 4(4 t + 8)-2(3 - 2 t)+3(3 t)=0$$\Rightarrow t=- \frac{26}{29}$. 

Tọa độ điểm $H(\frac{70}{29};- \frac{35}{29};- \frac{49}{29})$. Vậy $P=\frac{70}{29}- \frac{35}{29}- \frac{49}{29}=- \frac{14}{29}$. 
 }\end{ex}

\begin{ex}
 Trong không gian ${Oxyz}$, cho hai điểm $M(-4;-2;-4),N(12;20;20)$. Mặt cầu ${(S)}$ có đường kính ${MN}$ có phương trình là
 
\choice
{ $\left(x + 4\right)^{2}+\left(y + 9\right)^{2}+\left(z + 8\right)^{2}=329$ }
   { \True $\left(x - 4\right)^{2}+\left(y - 9\right)^{2}+\left(z - 8\right)^{2}=329$ }
     { $\left(x - 4\right)^{2}+\left(y - 9\right)^{2}+\left(z - 8\right)^{2}=1316$ }
    { $\left(x + 4\right)^{2}+\left(y + 9\right)^{2}+\left(z + 8\right)^{2}=\sqrt{329}$ }
\loigiai{ 

 Mặt cầu ${(S)}$ có tâm ${I(4;9;8)}$ là trung điểm của đoạn thẳng ${MN}$. 

${MN}=\sqrt{\left(12-(-4)\right)^2 + \left(20-(-2)\right)^2 + \left(20-(-4)\right)^2}= 2 \sqrt{329}$.

 ${(S)}$ có bán kính $R=\dfrac{MN}{2}=\sqrt{329}$.

Phương trình mặt cầu: $\left(x - 4\right)^{2}+\left(y - 9\right)^{2}+\left(z - 8\right)^{2}=329$. 
 }\end{ex}


 \begin{center}
-----HẾT-----
\end{center}

 \Closesolutionfile{ans}
% \newpage
% BẢNG ĐÁP ÁN MÃ ĐỀ 005
% \inputansbox{2}{ans/ans005}


\end{document}