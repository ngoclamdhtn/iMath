\documentclass[12pt,a4paper]{article}
\usepackage[top=1.5cm, bottom=1.5cm, left=2.0cm, right=1.5cm] {geometry}
\usepackage{amsmath,amssymb,txfonts}
\usepackage{tkz-euclide}
\usepackage{setspace}
\usepackage{lastpage}

\usepackage{tikz,tkz-tab}
%\usepackage[solcolor]{ex_test}
\usepackage[dethi]{ex_test} % Chỉ hiển thị đề thi
%\usepackage[loigiai]{ex_test} % Hiển thị lời giải
%\usepackage[color]{ex_test} % Khoanh các đáp án
\everymath{\displaystyle}

\def\colorEX{\color{purple}}
%\def\colorEX{}%Không tô màu đáp án đúng trong tùy chọn loigiai
\renewtheorem{ex}{\color{violet}Câu}
\renewcommand{\FalseEX}{\stepcounter{dapan}{{\bf \textcolor{blue}{\Alph{dapan}.}}}}
\renewcommand{\TrueEX}{\stepcounter{dapan}{{\bf \textcolor{blue}{\Alph{dapan}.}}}}

%---------- Khai báo viết tắt, in đáp án
\newcommand{\hoac}[1]{ %hệ hoặc
    \left[\begin{aligned}#1\end{aligned}\right.}
\newcommand{\heva}[1]{ %hệ và
    \left\{\begin{aligned}#1\end{aligned}\right.}

%Tiêu đề
\newcommand{\tenso}{iMath}
\newcommand{\tentruong}{Phần mềm tạo đề tự động}
\newcommand{\tenkythi}{ĐỀ ÔN TẬP TOÁN 12}
\newcommand{\tenmonthi}{Môn học: Toán}
\newcommand{\thoigian}{}
\newcommand{\tieude}[2]{
    \noindent
    %Trái
    \begin{minipage}[b]{7cm}
        \centerline{\textbf{\fontsize{13}{0}\selectfont \tenso}}
        \centerline{\textbf{\fontsize{13}{0}\selectfont \tentruong}}
        \centerline{(\textit{Đề thi có #1\ trang})}
    \end{minipage}\hspace{1.5cm}
    %Phải
    \begin{minipage}[b]{9cm}
        \centerline{\textbf{\fontsize{13}{0}\selectfont \tenkythi}}
        \centerline{\textbf{\fontsize{13}{0}\selectfont \tenmonthi}}
        \centerline{\textit{\fontsize{12}{0}\selectfont Thời gian làm bài: \thoigian\ phút}}
    \end{minipage}
    \begin{minipage}[b]{10cm}
        \textbf{Họ và tên HS: }{\tiny\dotfill}
    \end{minipage}
    \begin{minipage}[b]{8cm}
        \hspace*{4cm}\fbox{\bf Mã đề: #2}
    \end{minipage}\vspace{3pt}
}
\newcommand{\chantrang}[2]{\rfoot{Trang \thepage $-$ Mã đề #2}}
\pagestyle{fancy}
\fancyhf{}

\begin{document}
%Thiết lập giãn dọng 1.5cm 
%\setlength{\lineskip}{1.5em}


%Nội dung trắc nghiệm bắt đầu ở đây


\tieude{\pageref{LastPage}}{004}

\chantrang{\pageref{LastPage}}{004}

\setcounter{page}{1}

\setcounter{ex}{0}
\Opensolutionfile{ans}[ans/ans004]
\begin{ex}
 Trong mặt phẳng ${Oxy}$, phương trình đường tròn ${(C)}$ có tâm ${H(4;-6)}$ và đi qua điểm $N(5;0)$ là
 
\choice
{ $\left(x + 4\right)^{2}+\left(y - 6\right)^{2}=37$ }
   { \True $\left(x - 4\right)^{2}+\left(y + 6\right)^{2}=37$ }
     { $\left(x + 5\right)^{2}+y^{2}=37$ }
    { $\left(x - 5\right)^{2}+y^{2}=37$ }
\loigiai{ 

 Đường tròn ${(C)}$ có bán kính là ${HN}=\sqrt{(5-4)^2+(0-(-6))^2}=\sqrt{37}$.

Đường tròn ${(C)}$ có phương trình là: $\left(x - 4\right)^{2}+\left(y + 6\right)^{2}=37$. 
 }\end{ex}

\begin{ex}
 Trong mặt phẳng ${Oxy}$, cho đường thẳng $\Delta: - x - 5 y - 7=0$ và điểm ${N(7;4)}$. Đường tròn ${(C)}$ có tâm ${N}$ và tiếp xúc với đường thẳng $\Delta$ có phương trình là
 
\choice
{ $\left(x + 7\right)^{2}+\left(y + 4\right)^{2}=\frac{578}{13}$ }
   { $\left(x - 7\right)^{2}+\left(y - 4\right)^{2}=\frac{17 \sqrt{26}}{13}$ }
     { \True $\left(x - 7\right)^{2}+\left(y - 4\right)^{2}=\frac{578}{13}$ }
    { $\left(x - 7\right)^{2}+\left(y - 4\right)^{2}=1156$ }
\loigiai{ 

 Đường tròn ${(C)}$ có bán kính là: $R=d(N,\Delta)=\dfrac{|(-1).7+(-5).4-7|}{ \sqrt{1+25} }=\frac{17 \sqrt{26}}{13}$.

Đường tròn ${(C)}$ có phương trình là: $\left(x - 7\right)^{2}+\left(y - 4\right)^{2}=\frac{578}{13}$. 
 }\end{ex}

\begin{ex}
 Trong không gian ${Oxyz}$, cho vectơ $\overrightarrow{c}=(5;2;8)$. Độ dài vectơ $\overrightarrow{c}$ bằng.\\ 
\choice
{ ${93}$ }
   { ${15}$ }
     { \True ${\sqrt{93}}$ }
    { ${94}$ }
\loigiai{ 
 $|\overrightarrow{c}|=\sqrt{25+4+64}=\sqrt{93}$. 
 }\end{ex}

\begin{ex}
 Trong hệ trục tọa độ ${Oxyz}$, cho hai véctơ $\overrightarrow{a}(5;4;6)$ và $\overrightarrow{v}(7;5;7)$. Tọa độ tích có hướng $\left[\overrightarrow{a},\overrightarrow{v}\right]$ là\\ 
\choice
{ $\left[\overrightarrow{a},\overrightarrow{v}\right]=(-3;10;-4)$ }
   { $\left[\overrightarrow{a},\overrightarrow{v}\right]=(-2;11;-8)$ }
     { $\left[\overrightarrow{a},\overrightarrow{v}\right]=(1;6;-2)$ }
    { \True $\left[\overrightarrow{a},\overrightarrow{v}\right]= (-2;7;-3)$ }
\loigiai{ 
 $\left[\overrightarrow{a}.\overrightarrow{v}\right]=(-2;7;-3)$. 
 }\end{ex}

\begin{ex}
 Trong không gian ${Oxyz}$, cho mặt phẳng ${(\beta)}$ có phương trình $- 8 x - 3 y - 7 z - 2=0$. Mặt phẳng ${(\beta)}$ nhận vectơ nào trong các vectơ sau làm véctơ pháp tuyến.
 
\choice
{ $\overrightarrow{n_4}=(-24;3;-7)$ }
   { $\overrightarrow{n_4}=(-3;-7;-2)$ }
     { $\overrightarrow{n_4}=(-8;3;-7)$ }
    { \True $\overrightarrow{n_4}=(-24;-9;-21)$ }
\loigiai{ 

 Mặt phẳng ${(\beta)}$ có một véctơ pháp tuyến là $\overrightarrow{n}=(-8;-3;-7)$.

 nên cũng nhận vectơ $\overrightarrow{n_4}=(-24;-9;-21)$ làm véctơ pháp tuyến. 
 }\end{ex}

\begin{ex}
 Trong không gian ${Oxyz}$, cho mặt phẳng ${(P)}$ có phương trình $24 x - 29 y - 52 z + 76=0$. Điểm nào trong các điểm sau thuộc mặt phẳng ${(P)}$?
 
\choice
{ $G(2;8;5)$ }
   { $B(6;4;-2)$ }
     { \True $D(-1;0;1)$ }
    { $K(5;-8;3)$ }
\loigiai{ 

 Thay tọa độ các điểm vào phương trình mặt phẳng ${(P)}$ta thấy chỉ có điểm $D(-1;0;1)$ thỏa mãn.
 
 }\end{ex}

\begin{ex}
 Trong không gian ${Oxyz}$, cho mặt phẳng ${(R)}$ có phương trình $- 7 x + 7 y - 3 z + 10=0$. Điểm nào trong các điểm sau không thuộc mặt phẳng ${(R)}$?
 
\choice
{ $D(6;2;-6)$ }
   { $I(8;4;-6)$ }
     { $E(-6;-7;1)$ }
    { \True $D(7;4;1)$ }
\loigiai{ 

 Thay tọa độ các điểm vào phương trình mặt phẳng ${(R)}$ta thấy điểm $D(6;2;-6)$ không thỏa mãn phương trình nên điểm ${D}$ không thuộc mặt phẳng ${(R)}$.
 
 }\end{ex}

\begin{ex}
 Trong không gian ${Oxyz}$, đường thẳng ${d}$ đi qua điểm ${D(-5;5;1)}$ và nhận vectơ $\vec{u}=(5;-2;3)$ làm véctơ chỉ phương có phương trình là
 
\choice
{ \True $\left\{ \begin{array}{l}x = -5+5t\\ y = 5-2t\\z = 1+3t\end{array} \right.$ }
   { $\left\{ \begin{array}{l}x = -5+5t\\ y = -5+2t\\z = 1+3t\end{array} \right.$ }
     { $\left\{ \begin{array}{l}x = 5-5t\\ y = -2+5t\\z = 3+t\end{array} \right.$ }
    { $\left\{ \begin{array}{l}x = 5+5t\\ y = -5-2t\\z = -1+3t\end{array} \right.$ }
\loigiai{ 

 Đường thẳng ${d}$ đi qua điểm ${D(-5;5;1)}$ nhận vectơ $\vec{u}=(5;-2;3)$ làm véctơ chỉ phương có phương có phương trình là: $\left\{ \begin{array}{l}x = -5+5t\\ y = 5-2t\\z = 1+3t\end{array} \right.$. 
 }\end{ex}

\begin{ex}
 Trong không gian ${Oxyz}$, đường thẳng ${\Delta}$ đi qua điểm ${I(-5;1;4)}$ và nhận vectơ $\overrightarrow{HK}$ làm véctơ chỉ phương với $H(-7;-5;3)$ và $K(-12;5;-5)$ có phương trình là
 
\choice
{ $\left\{ \begin{array}{l}x = -5-5t\\ y = 10+t\\z = -8+4t\end{array} \right.$ }
   { \True $\left\{ \begin{array}{l}x = -5-5t\\ y = 1+10t\\z = 4-8t\end{array} \right.$ }
     { $\left\{ \begin{array}{l}x = 5-5t\\ y = -1+10t\\z = -4-8t\end{array} \right.$ }
    { $\left\{ \begin{array}{l}x = -5-5t\\ y = -1-10t\\z = 4-8t\end{array} \right.$ }
\loigiai{ 

 Ta có: $\overrightarrow{HK}=(-5;10;-8)$.

Đường thẳng ${\Delta}$ đi qua điểm ${I(-5;1;4)}$ nhận vectơ $\overrightarrow{HK}=(-5;10;-8)$ làm véctơ chỉ phương có phương trình là: $\left\{ \begin{array}{l}x = -5-5t\\ y = 1+10t\\z = 4-8t\end{array} \right.$. 
 }\end{ex}

\begin{ex}
 Trong không gian ${Oxyz}$, đường thẳng ${d}$ đi qua điểm ${M(2;7;3)}$ và song song với đường thẳng $d_1:\dfrac{x + 1}{-6}=\dfrac{y - 8}{3}=\dfrac{z - 2}{-24}$ có phương trình là
 
\choice
{ $\left\{ \begin{array}{l}x = 2-2t\\ y = -7-t\\z = 3-8t\end{array} \right.$ }
   { $\left\{ \begin{array}{l}x = -2+2t\\ y = 1+7t\\z = -8+3t\end{array} \right.$ }
     { \True $\left\{ \begin{array}{l}x = 2-2t\\ y = 7+t\\z = 3-8t\end{array} \right.$ }
    { $\left\{ \begin{array}{l}x = -2-2t\\ y = -7+t\\z = -3-8t\end{array} \right.$ }
\loigiai{ 

 Đường thẳng ${d_1}$ có véctơ chỉ phương là $\overrightarrow{u_1}=(-6;3;-24)$.

Đường thẳng ${d}$ song song với ${d_1}$ nên có một véctơ chỉ phương là vectơ $\vec{u}=(-2;1;-8)$.

Đường thẳng ${d}$ đi qua điểm ${M(2;7;3)}$ nhận vectơ $\vec{u}=(-2;1;-8)$ làm véctơ chỉ phương có phương có phương trình là: $\left\{ \begin{array}{l}x = 2-2t\\ y = 7+t\\z = 3-8t\end{array} \right.$. 
 }\end{ex}

\begin{ex}
 Trong không gian ${Oxyz}$, cho đường thẳng ${\Delta}: \dfrac{x + 7}{-1}=\dfrac{y - 6}{3}=\dfrac{z + 5}{-1}$. Phương trình tham số của đường thẳng ${\Delta}$ là
 
\choice
{ $\left\{ \begin{array}{l}x = 7-t\\ y = -6+3t\\z = 5-t\end{array} \right.$ }
   { $\left\{ \begin{array}{l}x = -7-t\\ y = -6-3t\\z = -5-t\end{array} \right.$ }
     { \True $\left\{ \begin{array}{l}x = -7-t\\ y = 6+3t\\z = -5-t\end{array} \right.$ }
    { $\left\{ \begin{array}{l}x = -1-7t\\ y = 3+6t\\z = -1-5t\end{array} \right.$ }
\loigiai{ 

 Đường thẳng ${\Delta}$ đi qua điểm ${E(-7;6;-5)}$ nhận vectơ $\vec{u}=(-1;3;-1)$ làm véctơ chỉ phương có phương có phương trình là: $\left\{ \begin{array}{l}x = -7-t\\ y = 6+3t\\z = -5-t\end{array} \right.$. 
 }\end{ex}

\begin{ex}
 Trong không gian ${Oxyz}$, cho đường thẳng ${d}:\left\{ \begin{array}{l}x = 2-7t\\ y = 3+5t\\z = 5+9t\end{array} \right.$. Phương trình chính tắc của đường thẳng ${d}$ là
 
\choice
{ $\dfrac{x + 7}{2}=\dfrac{y - 5}{3}=\dfrac{z - 9}{5}$ }
   { $\dfrac{x - 7}{2}=\dfrac{y + 5}{3}=\dfrac{z + 9}{5}$ }
     { $\dfrac{x + 2}{-7}=\dfrac{y + 3}{5}=\dfrac{z + 5}{9}$ }
    { \True $\dfrac{x - 2}{-7}=\dfrac{y - 3}{5}=\dfrac{z - 5}{9}$ }
\loigiai{ 

 Đường thẳng ${d}$ đi qua điểm ${E(2;3;5)}$ nhận vectơ $\vec{u}=(-7;5;9)$ làm véctơ chỉ phương có phương có phương trình là: $\dfrac{x - 2}{-7}=\dfrac{y - 3}{5}=\dfrac{z - 5}{9}$. 
 }\end{ex}

\begin{ex}
 Trong không gian ${Oxyz}$, cho đường thẳng ${d}:\left\{ \begin{array}{l}x = 4-3t\\ y = -3+3t\\z = -7-8t\end{array} \right.$. Đường thẳng ${d}$ nhận vectơ nào sau đây làm véctơ chỉ phương?
 
\choice
{ $\overrightarrow{u_4}=(4;-3;-7)$ }
   { \True $\overrightarrow{u_3}=(-9;9;-24)$ }
     { $\overrightarrow{u_1}=(3;3;8)$ }
    { $\overrightarrow{u_2}=(-4;3;7)$ }
\loigiai{ 

 Đường thẳng ${d}$ nhận vectơ $\vec{u}=(-3;3;-8)$ làm véctơ chỉ phương nên cũng nhận vectơ $\overrightarrow{u_3}=(-9;9;-24)$ làm véctơ chỉ phương. 
 }\end{ex}

\begin{ex}
 Trong không gian ${Oxyz}$, cho đường thẳng ${d}:\left\{ \begin{array}{l}x = -3-3t\\ y = 5t\\z = 5-t\end{array} \right.$. Đường thẳng ${d}$ đi qua điểm nào trong các điểm sau?
 
\choice
{ $C=(3;0;-5)$ }
   { $D=(-3;5;-1)$ }
     { \True $B=(-12;15;2)$ }
    { $A=(-10;10;1)$ }
\loigiai{ 

 Tồn tại $t=3 \Rightarrow x=-12,y=15,z=2$ nên đường thẳng ${d}$ đi qua $B=(-12;15;2)$. 
 }\end{ex}

\begin{ex}
 Trong không gian ${Oxyz}$, cho đường thẳng ${d}:\dfrac{x - 7}{5}=\dfrac{y - 6}{-9}=\dfrac{z - 1}{-1}$. Đường thẳng ${d}$ nhận vectơ nào sau đây làm véctơ chỉ phương?
 
\choice
{ $\overrightarrow{u_4}=(-7;-6;-1)$ }
   { $\overrightarrow{u_3}=(7;6;1)$ }
     { \True $\overrightarrow{u_2}=(-10;18;2)$ }
    { $\overrightarrow{u_1}=(-5;-9;1)$ }
\loigiai{ 

 Đường thẳng ${d}$ nhận vectơ $\vec{u}=(5;-9;-1)$ làm véctơ chỉ phương nên cũng nhận vectơ $\overrightarrow{u_2}=(-10;18;2)$ làm véctơ chỉ phương. 
 }\end{ex}

\begin{ex}
 Trong không gian ${Oxyz}$, cho đường thẳng ${d}:\dfrac{x - 1}{-8}=\dfrac{y + 6}{-4}=\dfrac{z - 6}{-1}$. Đường thẳng ${d}$ đi qua điểm nào trong các điểm sau?
 
\choice
{ $B=(19;1;1)$ }
   { \True $C=(17;2;8)$ }
     { $A=(-1;6;-6)$ }
    { $D=(-8;-4;-1)$ }
\loigiai{ 

 Đường thẳng ${d}$ có phương trình tham số là $\left\{ \begin{array}{l}x = 1-8t\\ y = -6-4t\\z = 6-t\end{array} \right.$.

Tồn tại $t=-2 \Rightarrow x=17,y=2,z=8$ nên đường thẳng ${d}$ đi qua $C=(17;2;8)$. 
 }\end{ex}

\begin{ex}
 Trong không gian ${Oxyz}$, tọa độ giao điểm của đường thẳng ${\Delta}:\dfrac{x - 1}{8}=\dfrac{y - 2}{-1}=\dfrac{z - 1}{4}$ và mặt phẳng $(P):3 x - 6 y + 4 z + 51=0$ là điểm $H(a;b;c)$. Tính $P=a+b+c$.
 
\choice
{ ${-19}$ }
   { ${5}$ }
     { ${1}$ }
    { \True ${-7}$ }
\loigiai{ 

 Đường thẳng ${\Delta}$ có phương trình tham số là $\left\{ \begin{array}{l}x = 1+8t\\ y = 2-t\\z = 1+4t\end{array} \right.$.

Xét phương trình $3(8 t + 1)-6(2 - t)+4(4 t + 1)+51=0\Rightarrow t=-1$.

Tọa độ giao điểm của ${\Delta}$ và ${(P)}$ là $H(-7;3;-3)$.

 Vậy $P=-7+3-3=-7$. 
 }\end{ex}

\begin{ex}
 Trong không gian ${Oxyz}$, cho đường thẳng ${\Delta}:\dfrac{x - 7}{-5}=\dfrac{y - 4}{-2}=\dfrac{z - 3}{7}$ và điểm $B(-5;-6;7)$.

 Hình chiếu vuông góc của điểm $B$ trên đường thẳng ${\Delta}$ là điểm $H(a;b;c)$. Tính $P=a+b+c$.
 
\choice
{ ${- \frac{311}{13}}$ }
   { \True ${14}$ }
     { ${- \frac{148}{13}}$ }
    { ${\frac{480}{13}}$ }
\loigiai{ 

 Đường thẳng ${\Delta}$ có véctơ chỉ phương là $\vec{u}=(-5;-2;7)$.

Gọi $H(7-5t;4-2t;3+7t)$.

$\overrightarrow{BH}=(12 - 5 t;10 - 2 t;7 t - 4)$.

$\overrightarrow{BH}.\overrightarrow{u}=0\Leftrightarrow -5(12 - 5 t)-2(10 - 2 t)+7(7 t - 4)=0$$\Rightarrow t=\frac{18}{13}$. 

Tọa độ điểm $H(\frac{1}{13};\frac{16}{13};\frac{165}{13})$. Vậy $P=\frac{1}{13}+\frac{16}{13}+\frac{165}{13}=14$. 
 }\end{ex}

\begin{ex}
 Trong không gian ${Oxyz}$, cho hai điểm $A(2;2;-3),B(16;-20;15)$. Mặt cầu ${(S)}$ có đường kính ${AB}$ có phương trình là
 
\choice
{ \True $\left(x - 9\right)^{2}+\left(y + 9\right)^{2}+\left(z - 6\right)^{2}=251$ }
   { $\left(x + 9\right)^{2}+\left(y - 9\right)^{2}+\left(z + 6\right)^{2}=251$ }
     { $\left(x + 9\right)^{2}+\left(y - 9\right)^{2}+\left(z + 6\right)^{2}=\sqrt{251}$ }
    { $\left(x - 9\right)^{2}+\left(y + 9\right)^{2}+\left(z - 6\right)^{2}=1004$ }
\loigiai{ 

 Mặt cầu ${(S)}$ có tâm ${I(9;-9;6)}$ là trung điểm của đoạn thẳng ${AB}$. 

${AB}=\sqrt{\left(16-2\right)^2 + \left(-20-2\right)^2 + \left(15-(-3)\right)^2}= 2 \sqrt{251}$.

 ${(S)}$ có bán kính $R=\dfrac{AB}{2}=\sqrt{251}$.

Phương trình mặt cầu: $\left(x - 9\right)^{2}+\left(y + 9\right)^{2}+\left(z - 6\right)^{2}=251$. 
 }\end{ex}


 \begin{center}
-----HẾT-----
\end{center}

 \Closesolutionfile{ans}
% \newpage
% BẢNG ĐÁP ÁN MÃ ĐỀ 004
% \inputansbox{2}{ans/ans004}



\end{document}