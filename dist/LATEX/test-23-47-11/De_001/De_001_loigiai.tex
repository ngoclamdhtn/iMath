\documentclass[12pt,a4paper]{article}
\usepackage[top=1.5cm, bottom=1.5cm, left=2.0cm, right=1.5cm] {geometry}
\usepackage{amsmath,amssymb,txfonts}
\usepackage{tkz-euclide}
\usepackage{setspace}
\usepackage{lastpage}

\usepackage{tikz,tkz-tab}
%\usepackage[solcolor]{ex_test}
%\usepackage[dethi]{ex_test} % Chỉ hiển thị đề thi
\usepackage[loigiai]{ex_test} % Hiển thị lời giải
%\usepackage[color]{ex_test} % Khoanh các đáp án
\everymath{\displaystyle}

\def\colorEX{\color{purple}}
%\def\colorEX{}%Không tô màu đáp án đúng trong tùy chọn loigiai
\renewtheorem{ex}{\color{violet}Câu}
\renewcommand{\FalseEX}{\stepcounter{dapan}{{\bf \textcolor{blue}{\Alph{dapan}.}}}}
\renewcommand{\TrueEX}{\stepcounter{dapan}{{\bf \textcolor{blue}{\Alph{dapan}.}}}}

%---------- Khai báo viết tắt, in đáp án
\newcommand{\hoac}[1]{ %hệ hoặc
    \left[\begin{aligned}#1\end{aligned}\right.}
\newcommand{\heva}[1]{ %hệ và
    \left\{\begin{aligned}#1\end{aligned}\right.}

%Tiêu đề
\newcommand{\tenso}{iMath}
\newcommand{\tentruong}{Phần mềm tạo đề tự động}
\newcommand{\tenkythi}{ĐỀ ÔN TẬP TOÁN 12}
\newcommand{\tenmonthi}{Môn thi: Toán}
\newcommand{\thoigian}{}
\newcommand{\tieude}[2]{
    \noindent
    %Trái
    \begin{minipage}[b]{7cm}
        \centerline{\textbf{\fontsize{13}{0}\selectfont \tenso}}
        \centerline{\textbf{\fontsize{13}{0}\selectfont \tentruong}}
        \centerline{(\textit{Đề thi có #1\ trang})}
    \end{minipage}\hspace{1.5cm}
    %Phải
    \begin{minipage}[b]{9cm}
        \centerline{\textbf{\fontsize{13}{0}\selectfont \tenkythi}}
        \centerline{\textbf{\fontsize{13}{0}\selectfont \tenmonthi}}
        \centerline{\textit{\fontsize{12}{0}\selectfont Thời gian làm bài: \thoigian\ phút}}
    \end{minipage}
    \begin{minipage}[b]{10cm}
        \textbf{Họ và tên HS: }{\tiny\dotfill}
    \end{minipage}
    \begin{minipage}[b]{8cm}
        \hspace*{4cm}\fbox{\bf Mã đề: #2}
    \end{minipage}\vspace{3pt}
}
\newcommand{\chantrang}[2]{\rfoot{Trang \thepage $-$ Mã đề #2}}
\pagestyle{fancy}
\fancyhf{}
\begin{document}
%Thiết lập giãn dọng 1.5cm 
%\setlength{\lineskip}{1.5em}
%Nội dung trắc nghiệm bắt đầu ở đây


\tieude{\pageref{LastPage}}{001}

\chantrang{\pageref{LastPage}}{001}

\setcounter{page}{1}

\setcounter{ex}{0}
\Opensolutionfile{ans}[ans/ans001]
\begin{ex}
 Trong mặt phẳng ${Oxy}$, phương trình đường tròn ${(C)}$ có tâm ${A(-3;-5)}$ và đi qua điểm $M(1;10)$ là
 
\choice
{ $\left(x - 3\right)^{2}+\left(y - 5\right)^{2}=241$ }
   { $\left(x + 1\right)^{2}+\left(y + 10\right)^{2}=241$ }
     { $\left(x - 1\right)^{2}+\left(y - 10\right)^{2}=241$ }
    { \True $\left(x + 3\right)^{2}+\left(y + 5\right)^{2}=241$ }
\loigiai{ 

 Đường tròn ${(C)}$ có bán kính là ${AM}=\sqrt{(1-(-3))^2+(10-(-5))^2}=\sqrt{241}$.

Đường tròn ${(C)}$ có phương trình là: $\left(x + 3\right)^{2}+\left(y + 5\right)^{2}=241$. 
 }\end{ex}

\begin{ex}
 Trong mặt phẳng ${Oxy}$, cho đường thẳng $\Delta: 3 x + 2 y + 3=0$ và điểm ${A(-4;3)}$. Đường tròn ${(C)}$ có tâm ${A}$ và tiếp xúc với đường thẳng $\Delta$ có phương trình là
 
\choice
{ $\left(x + 4\right)^{2}+\left(y - 3\right)^{2}=\frac{3 \sqrt{13}}{13}$ }
   { $\left(x - 4\right)^{2}+\left(y + 3\right)^{2}=\frac{9}{13}$ }
     { $\left(x + 4\right)^{2}+\left(y - 3\right)^{2}=9$ }
    { \True $\left(x + 4\right)^{2}+\left(y - 3\right)^{2}=\frac{9}{13}$ }
\loigiai{ 

 Đường tròn ${(C)}$ có bán kính là: $R=d(A,\Delta)=\dfrac{|3.(-4)+2.3+3|}{ \sqrt{9+4} }=\frac{3 \sqrt{13}}{13}$.

Đường tròn ${(C)}$ có phương trình là: $\left(x + 4\right)^{2}+\left(y - 3\right)^{2}=\frac{9}{13}$. 
 }\end{ex}

\begin{ex}
 Trong không gian ${Oxyz}$, cho vectơ $\overrightarrow{v}=(-5;0;-7)$. Độ dài vectơ $\overrightarrow{v}$ bằng.\\ 
\choice
{ ${75}$ }
   { ${74}$ }
     { \True ${\sqrt{74}}$ }
    { ${12}$ }
\loigiai{ 
 $|\overrightarrow{v}|=\sqrt{25+0+49}=\sqrt{74}$. 
 }\end{ex}

\begin{ex}
 Trong hệ trục tọa độ ${Oxyz}$, cho hai véctơ $\overrightarrow{u}(-9;10;2)$ và $\overrightarrow{w}(-5;0;1)$. Tọa độ tích có hướng $\left[\overrightarrow{u},\overrightarrow{w}\right]$ là\\ 
\choice
{ $\left[\overrightarrow{u},\overrightarrow{w}\right]=(10;2;49)$ }
   { $\left[\overrightarrow{u},\overrightarrow{w}\right]=(11;-5;54)$ }
     { \True $\left[\overrightarrow{u},\overrightarrow{w}\right]= (10;-1;50)$ }
    { $\left[\overrightarrow{u},\overrightarrow{w}\right]=(8;2;47)$ }
\loigiai{ 
 $\left[\overrightarrow{u}.\overrightarrow{w}\right]=(10;-1;50)$. 
 }\end{ex}

\begin{ex}
 Trong không gian ${Oxyz}$, cho mặt phẳng ${(R)}$ có phương trình $- 3 x + 4 z + 15=0$. Mặt phẳng ${(R)}$ nhận vectơ nào trong các vectơ sau làm véctơ pháp tuyến.
 
\choice
{ $\overrightarrow{n_1}=(3;0;4)$ }
   { $\overrightarrow{n_1}=(-3;0;15)$ }
     { \True $\overrightarrow{n_1}=(-6;0;8)$ }
    { $\overrightarrow{n_1}=(-3;0;15)$ }
\loigiai{ 

 Mặt phẳng ${(R)}$ có một véctơ pháp tuyến là $\overrightarrow{n}=(-3;0;4)$.

 Do đó, mặt phẳng ${(R)}$ cũng nhận vectơ $\overrightarrow{n_1}=(-6;0;8)$ làm véctơ pháp tuyến. 
 }\end{ex}

\begin{ex}
 Trong không gian ${Oxyz}$, cho mặt phẳng ${(P)}$ có phương trình $- 9 x + 14 y - 9 z - 43=0$. Điểm nào trong các điểm sau thuộc mặt phẳng ${(P)}$?
 
\choice
{ $D(9;-4;-3)$ }
   { $N(5;5;2)$ }
     { \True $A(4;5;-1)$ }
    { $C(5;4;-6)$ }
\loigiai{ 

 Thay tọa độ các điểm vào phương trình mặt phẳng ${(P)}$ta thấy chỉ có điểm $A(4;5;-1)$ thỏa mãn.
 
 }\end{ex}

\begin{ex}
 Trong không gian ${Oxyz}$, cho mặt phẳng ${(\alpha)}$ có phương trình $- 7 x - 38 y + 36 z - 27=0$. Điểm nào trong các điểm sau không thuộc mặt phẳng ${(\alpha)}$?
 
\choice
{ $D(-3;-3;-3)$ }
   { $G(5;5;7)$ }
     { \True $A(0;5;-5)$ }
    { $B(3;-6;-5)$ }
\loigiai{ 

 Thay tọa độ các điểm vào phương trình mặt phẳng ${(\alpha)}$ta thấy điểm $A(-3;-3;-3)$ không thỏa mãn phương trình nên điểm ${A}$ không thuộc mặt phẳng ${(\alpha)}$.
 
 }\end{ex}

\begin{ex}
 Trong không gian ${Oxyz}$, đường thẳng ${\Delta}$ đi qua điểm ${C(4;4;-1)}$ và nhận vectơ $\vec{u}=(-3;-3;-4)$ làm véctơ chỉ phương có phương trình là
 
\choice
{ \True $\left\{ \begin{array}{l}x = 4-3t\\ y = 4-3t\\z = -1-4t\end{array} \right.$ }
   { $\left\{ \begin{array}{l}x = 4-3t\\ y = -4+3t\\z = -1-4t\end{array} \right.$ }
     { $\left\{ \begin{array}{l}x = -4-3t\\ y = -4-3t\\z = 1-4t\end{array} \right.$ }
    { $\left\{ \begin{array}{l}x = -3+4t\\ y = -3+4t\\z = -4-t\end{array} \right.$ }
\loigiai{ 

 Đường thẳng ${\Delta}$ đi qua điểm ${C(4;4;-1)}$ nhận vectơ $\vec{u}=(-3;-3;-4)$ làm véctơ chỉ phương có phương có phương trình là: $\left\{ \begin{array}{l}x = 4-3t\\ y = 4-3t\\z = -1-4t\end{array} \right.$. 
 }\end{ex}

\begin{ex}
 Trong không gian ${Oxyz}$, đường thẳng ${\Delta}$ đi qua điểm ${I(-8;-1;-7)}$ và nhận vectơ $\overrightarrow{DC}$ làm véctơ chỉ phương với $D(2;-1;7)$ và $C(9;-10;3)$ có phương trình là
 
\choice
{ \True $\left\{ \begin{array}{l}x = -8+7t\\ y = -1-9t\\z = -7-4t\end{array} \right.$ }
   { $\left\{ \begin{array}{l}x = 8+7t\\ y = 1-9t\\z = 7-4t\end{array} \right.$ }
     { $\left\{ \begin{array}{l}x = 7-8t\\ y = -9-t\\z = -4-7t\end{array} \right.$ }
    { $\left\{ \begin{array}{l}x = -8+7t\\ y = 1+9t\\z = -7-4t\end{array} \right.$ }
\loigiai{ 

 Ta có: $\overrightarrow{DC}=(7;-9;-4)$.

Đường thẳng ${\Delta}$ đi qua điểm ${I(-8;-1;-7)}$ nhận vectơ $\overrightarrow{DC}=(7;-9;-4)$ làm véctơ chỉ phương có phương trình là: $\left\{ \begin{array}{l}x = -8+7t\\ y = -1-9t\\z = -7-4t\end{array} \right.$. 
 }\end{ex}

\begin{ex}
 Trong không gian ${Oxyz}$, đường thẳng ${\Delta}$ đi qua điểm ${N(-6;2;6)}$ và song song với đường thẳng $d':\dfrac{x + 7}{-21}=\dfrac{y - 2}{-27}=\dfrac{z - 9}{6}$ có phương trình là
 
\choice
{ \True $\left\{ \begin{array}{l}x = -6+7t\\ y = 2+9t\\z = 6-2t\end{array} \right.$ }
   { $\left\{ \begin{array}{l}x = -6+7t\\ y = -2-9t\\z = 6-2t\end{array} \right.$ }
     { $\left\{ \begin{array}{l}x = 7-6t\\ y = 9+2t\\z = -2+6t\end{array} \right.$ }
    { $\left\{ \begin{array}{l}x = 6+7t\\ y = -2+9t\\z = -6-2t\end{array} \right.$ }
\loigiai{ 

 Đường thẳng ${d'}$ có véctơ chỉ phương là $\overrightarrow{u_1}=(-21;-27;6)$.

Đường thẳng ${\Delta}$ song song với ${d'}$ nên có một véctơ chỉ phương là vectơ $\vec{u}=(7;9;-2)$.

Đường thẳng ${\Delta}$ đi qua điểm ${N(-6;2;6)}$ nhận vectơ $\vec{u}=(7;9;-2)$ làm véctơ chỉ phương có phương có phương trình là: $\left\{ \begin{array}{l}x = -6+7t\\ y = 2+9t\\z = 6-2t\end{array} \right.$. 
 }\end{ex}

\begin{ex}
 Trong không gian ${Oxyz}$, cho đường thẳng ${\Delta}: \dfrac{x - 6}{-3}=\dfrac{y + 6}{-1}=\dfrac{z - 1}{4}$. Phương trình tham số của đường thẳng ${\Delta}$ là
 
\choice
{ $\left\{ \begin{array}{l}x = 6-3t\\ y = 6+t\\z = 1+4t\end{array} \right.$ }
   { $\left\{ \begin{array}{l}x = -6-3t\\ y = 6-t\\z = -1+4t\end{array} \right.$ }
     { \True $\left\{ \begin{array}{l}x = 6-3t\\ y = -6-t\\z = 1+4t\end{array} \right.$ }
    { $\left\{ \begin{array}{l}x = -3+6t\\ y = -1-6t\\z = 4+t\end{array} \right.$ }
\loigiai{ 

 Đường thẳng ${\Delta}$ đi qua điểm ${B(6;-6;1)}$ nhận vectơ $\vec{u}=(-3;-1;4)$ làm véctơ chỉ phương có phương có phương trình là: $\left\{ \begin{array}{l}x = 6-3t\\ y = -6-t\\z = 1+4t\end{array} \right.$. 
 }\end{ex}

\begin{ex}
 Trong không gian ${Oxyz}$, cho đường thẳng ${\Delta}:\left\{ \begin{array}{l}x = 4-3t\\ y = t\\z = -6+2t\end{array} \right.$. Phương trình chính tắc của đường thẳng ${\Delta}$ là
 
\choice
{ $\dfrac{x + 4}{3}=\dfrac{y}{-1}=\dfrac{z + 6}{-2}$ }
   { $\dfrac{x + 4}{-3}=\dfrac{y}{-1}=\dfrac{z - 6}{2}$ }
     { $\dfrac{x - 4}{-3}=\dfrac{y}{1}=\dfrac{z + 6}{2}$ }
    { \True $\dfrac{x - 4}{-3}=\dfrac{y}{-1}=\dfrac{z + 6}{2}$ }
\loigiai{ 

 Đường thẳng ${\Delta}$ đi qua điểm ${E(4;0;-6)}$ nhận vectơ $\vec{u}=(-3;-1;2)$ làm véctơ chỉ phương có phương có phương trình là: $\dfrac{x - 4}{-3}=\dfrac{y}{-1}=\dfrac{z + 6}{2}$. 
 }\end{ex}

\begin{ex}
 Trong không gian ${Oxyz}$, cho đường thẳng ${\Delta}:\left\{ \begin{array}{l}x = -1-2t\\ y = 1+7t\\z = -8-t\end{array} \right.$. Đường thẳng ${\Delta}$ nhận vectơ nào sau đây làm véctơ chỉ phương?
 
\choice
{ \True $\overrightarrow{u_2}=(-4;14;-2)$ }
   { $\overrightarrow{u_1}=(1;-1;8)$ }
     { $\overrightarrow{u_3}=(2;7;1)$ }
    { $\overrightarrow{u_4}=(-1;1;-8)$ }
\loigiai{ 

 Đường thẳng ${\Delta}$ nhận vectơ $\vec{u}=(-2;7;-1)$ làm véctơ chỉ phương nên cũng nhận vectơ $\overrightarrow{u_2}=(-4;14;-2)$ làm véctơ chỉ phương. 
 }\end{ex}

\begin{ex}
 Trong không gian ${Oxyz}$, cho đường thẳng ${d}:\left\{ \begin{array}{l}x = 3-2t\\ y = -4-t\\z = 7-3t\end{array} \right.$. Đường thẳng ${d}$ đi qua điểm nào trong các điểm sau?
 
\choice
{ $D=(-2;-1;-3)$ }
   { \True $B=(7;-2;13)$ }
     { $A=(-3;4;-7)$ }
    { $C=(10;-5;3)$ }
\loigiai{ 

 Tồn tại $t=-2 \Rightarrow x=7,y=-2,z=13$ nên đường thẳng ${d}$ đi qua $B=(7;-2;13)$. 
 }\end{ex}

\begin{ex}
 Trong không gian ${Oxyz}$, cho đường thẳng ${\Delta}:\dfrac{x - 1}{4}=\dfrac{y - 6}{1}=\dfrac{z + 4}{1}$. Đường thẳng ${\Delta}$ nhận vectơ nào sau đây làm véctơ chỉ phương?
 
\choice
{ \True $\overrightarrow{u_4}=(8;2;2)$ }
   { $\overrightarrow{u_3}=(1;6;-4)$ }
     { $\overrightarrow{u_2}=(-1;-6;4)$ }
    { $\overrightarrow{u_1}=(-4;1;-1)$ }
\loigiai{ 

 Đường thẳng ${\Delta}$ nhận vectơ $\vec{u}=(4;1;1)$ làm véctơ chỉ phương nên cũng nhận vectơ $\overrightarrow{u_4}=(8;2;2)$ làm véctơ chỉ phương. 
 }\end{ex}

\begin{ex}
 Trong không gian ${Oxyz}$, cho đường thẳng ${\Delta}:\dfrac{x - 1}{-2}=\dfrac{y - 6}{-6}=\dfrac{z - 1}{-7}$. Đường thẳng ${\Delta}$ đi qua điểm nào trong các điểm sau?
 
\choice
{ \True $D=(-1;0;-6)$ }
   { $B=(-2;-6;-7)$ }
     { $A=(0;-4;7)$ }
    { $C=(-1;-6;-1)$ }
\loigiai{ 

 Đường thẳng ${\Delta}$ có phương trình tham số là $\left\{ \begin{array}{l}x = 1-2t\\ y = 6-6t\\z = 1-7t\end{array} \right.$.

Tồn tại $t=1 \Rightarrow x=-1,y=0,z=-6$ nên đường thẳng ${\Delta}$ đi qua $D=(-1;0;-6)$. 
 }\end{ex}

\begin{ex}
 Trong không gian ${Oxyz}$, tọa độ giao điểm của đường thẳng ${\Delta}:\dfrac{x - 2}{-4}=\dfrac{y + 6}{7}=\dfrac{z - 4}{-6}$ và mặt phẳng $(\beta):- 6 x - 6 y - 4 z - 20=0$ là điểm $H(a;b;c)$. Tính $P=a+b+c$.
 
\choice
{ ${10}$ }
   { ${-22}$ }
     { \True ${-6}$ }
    { ${26}$ }
\loigiai{ 

 Đường thẳng ${\Delta}$ có phương trình tham số là $\left\{ \begin{array}{l}x = 2-4t\\ y = -6+7t\\z = 4-6t\end{array} \right.$.

Xét phương trình $-6(2 - 4 t)-6(7 t - 6)-4(4 - 6 t)-20=0\Rightarrow t=2$.

Tọa độ giao điểm của ${\Delta}$ và ${(\beta)}$ là $H(-6;8;-8)$.

 Vậy $P=-6+8-8=-6$. 
 }\end{ex}

\begin{ex}
 Trong không gian ${Oxyz}$, cho đường thẳng ${\Delta}:\dfrac{x - 6}{-1}=\dfrac{y + 5}{-2}=\dfrac{z + 2}{-6}$ và điểm $C(-2;-6;-5)$.

 Hình chiếu vuông góc của điểm $C$ trên đường thẳng ${\Delta}$ là điểm $H(a;b;c)$. Tính $P=a+b+c$.
 
\choice
{ \True ${- \frac{293}{41}}$ }
   { ${\frac{675}{41}}$ }
     { ${- \frac{271}{41}}$ }
    { ${\frac{435}{41}}$ }
\loigiai{ 

 Đường thẳng ${\Delta}$ có véctơ chỉ phương là $\vec{u}=(-1;-2;-6)$.

Gọi $H(6-1t;-5-2t;-2-6t)$.

$\overrightarrow{CH}=(8 - t;1 - 2 t;3 - 6 t)$.

$\overrightarrow{CH}.\overrightarrow{u}=0\Leftrightarrow -1(8 - t)-2(1 - 2 t)-6(3 - 6 t)=0$$\Rightarrow t=\frac{28}{41}$. 

Tọa độ điểm $H(\frac{218}{41};- \frac{261}{41};- \frac{250}{41})$. Vậy $P=\frac{218}{41}- \frac{261}{41}- \frac{250}{41}=- \frac{293}{41}$. 
 }\end{ex}

\begin{ex}
 Trong không gian ${Oxyz}$, cho hai điểm $C(2;6;-10),D(-20;-4;-6)$. Mặt cầu ${(S)}$ có đường kính ${CD}$ có phương trình là
 
\choice
{ $\left(x - 9\right)^{2}+\left(y + 1\right)^{2}+\left(z - 8\right)^{2}=5 \sqrt{6}$ }
   { \True $\left(x + 9\right)^{2}+\left(y - 1\right)^{2}+\left(z + 8\right)^{2}=150$ }
     { $\left(x - 9\right)^{2}+\left(y + 1\right)^{2}+\left(z - 8\right)^{2}=150$ }
    { $\left(x + 9\right)^{2}+\left(y - 1\right)^{2}+\left(z + 8\right)^{2}=600$ }
\loigiai{ 

 Mặt cầu ${(S)}$ có tâm ${I(-9;1;-8)}$ là trung điểm của đoạn thẳng ${CD}$. 

${CD}=\sqrt{\left(-20-2\right)^2 + \left(-4-6\right)^2 + \left(-6-(-10)\right)^2}= 10 \sqrt{6}$.

 ${(S)}$ có bán kính $R=\dfrac{CD}{2}=5 \sqrt{6}$.

Phương trình mặt cầu: $\left(x + 9\right)^{2}+\left(y - 1\right)^{2}+\left(z + 8\right)^{2}=150$. 
 }\end{ex}


 \begin{center}
-----HẾT-----
\end{center}

 \Closesolutionfile{ans}
% \newpage
% BẢNG ĐÁP ÁN MÃ ĐỀ 001
% \inputansbox{2}{ans/ans001}


\end{document}