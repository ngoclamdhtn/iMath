\documentclass[12pt,a4paper]{article}
\usepackage[top=1.5cm, bottom=1.5cm, left=2.0cm, right=1.5cm] {geometry}
\usepackage{amsmath,amssymb,txfonts}
\usepackage{tkz-euclide}
\usepackage{setspace}
\usepackage{lastpage}

\usepackage{tikz,tkz-tab}
%\usepackage[solcolor]{ex_test}
%\usepackage[dethi]{ex_test} % Chỉ hiển thị đề thi
\usepackage[loigiai]{ex_test} % Hiển thị lời giải
%\usepackage[color]{ex_test} % Khoanh các đáp án
\everymath{\displaystyle}

\def\colorEX{\color{purple}}
%\def\colorEX{}%Không tô màu đáp án đúng trong tùy chọn loigiai
\renewtheorem{ex}{\color{violet}Câu}
\renewcommand{\FalseEX}{\stepcounter{dapan}{{\bf \textcolor{blue}{\Alph{dapan}.}}}}
\renewcommand{\TrueEX}{\stepcounter{dapan}{{\bf \textcolor{blue}{\Alph{dapan}.}}}}

%---------- Khai báo viết tắt, in đáp án
\newcommand{\hoac}[1]{ %hệ hoặc
    \left[\begin{aligned}#1\end{aligned}\right.}
\newcommand{\heva}[1]{ %hệ và
    \left\{\begin{aligned}#1\end{aligned}\right.}

%Tiêu đề
\newcommand{\tenso}{iMath}
\newcommand{\tentruong}{Phần mềm Tạo đề ngẫu nhiên}
\newcommand{\tenkythi}{ĐỀ ÔN TẬP}
\newcommand{\tenmonthi}{Môn thi: Toán}
\newcommand{\thoigian}{}
\newcommand{\tieude}[1]{
   \begin{tabular}{cm{3cm}cm{3cm}cm{3cm}}
    {\bf \tenso} & & {\bf \tenkythi} \\
    {\bf \tentruong} & & {\bf \tenmonthi}\\
    && {\bf Thời gian: \bf \thoigian \, phút}\\
    && { \fbox{\bf Mã đề: #1}}
   \end{tabular}\\\\
    
   {Họ tên HS: \dotfill Số báo danh \dotfill}\\
}
\newcommand{\chantrang}[2]{\rfoot{Trang \thepage $-$ Mã đề #2}}
\pagestyle{fancy}
\fancyhf{}
\renewcommand{\headrulewidth}{0pt} 
\renewcommand{\footrulewidth}{0pt}

\begin{document}
%Thiết lập giãn dọng 1.5cm 
%\setlength{\lineskip}{1.5em}
%Nội dung trắc nghiệm bắt đầu ở đây


\tieude{007}
\chantrang{\pageref{LastPage}}{007}
\setcounter{page}{1}
{\bf PHẦN I. Câu trắc nghiệm nhiều phương án lựa chọn.}
\setcounter{ex}{0}
\Opensolutionfile{ans}[ans/ans007-1]
\begin{ex}
 Đổi số đo của góc $-710^\circ$ sang radian ta được kết quả bằng\\ 
\choice
{ $- \frac{73 \pi}{18}$ }
   { $- \frac{34 \pi}{9}$ }
     { \True $- \frac{71 \pi}{18}$ }
    { $- \frac{35 \pi}{9}$ }
\loigiai{ 
 Áp dụng công thức chuyển đổi: $-710^\circ=\dfrac{-710.\pi}{180}=- \frac{71 \pi}{18}$. 
 }\end{ex}

\begin{ex}
 Tính $\cot\frac{25 \pi}{3}$.\\ 
\choice
{ $ \frac{\sqrt{3}}{2} $ }
   { $ \sqrt{3} $ }
     { $ \frac{1}{2} $ }
    { \True $ \frac{\sqrt{3}}{3} $ }
\loigiai{ 
  
 }\end{ex}

\begin{ex}
 Cho ${x}$ là góc lượng giác. Tìm khẳng định đúng trong các khẳng định sau.\ 
\choice
{ $\cos \left(\frac{\pi}{2}-x\right)=\cos x$ }
   { $\cot (\pi+x)=\tan x$ }
     { \True $\sin (\pi+x)=-\sin x$ }
    { $\sin (\pi+x)=\cos x$ }
\loigiai{ 
 $\sin (\pi+x)=-\sin x$ là khẳng định đúng. 
 }\end{ex}

\begin{ex}
 Cho ${\alpha}$ là góc lượng giác. Tìm khẳng định đúng trong các khẳng định sau.\ 
\choice
{ $\cos 2\alpha=2\sin \alpha\cos \alpha$ }
   { $\sin 2\alpha=2\sin \alpha$ }
     { $\tan 2\alpha=\dfrac{2\tan \alpha}{1+\tan^2 \alpha}$ }
    { \True $\cos 2\alpha=\cos^2 \alpha-\sin^2 \alpha$ }
\loigiai{ 
 $\cos 2\alpha=\cos^2 \alpha-\sin^2 \alpha$ là khẳng định đúng. 
 }\end{ex}

\begin{ex}
 Cho ${\alpha,\beta}$ là các góc lượng giác. Tìm khẳng định đúng trong các khẳng định sau.\ 
\choice
{ $\cos \alpha \cos \beta=-\dfrac 1 2[\cos(\alpha+\beta) + \cos(\alpha-\beta)]$ }
   { $\sin \alpha \cos \beta=\dfrac 1 2[\cos(\alpha+\beta) - \cos(\alpha-\beta)]$ }
     { \True $\sin \alpha \cos \beta=\dfrac 1 2[\sin(\alpha+\beta) + \sin(\alpha-\beta)]$ }
    { $\sin \alpha \sin \beta=\dfrac 1 2[\cos(\alpha+\beta) - \cos(\alpha-\beta)]$ }
\loigiai{ 
 $\sin \alpha \cos \beta=\dfrac 1 2[\sin(\alpha+\beta) + \sin(\alpha-\beta)]$ là khẳng định đúng. 
 }\end{ex}

\begin{ex}
 Cho $\sin a=\frac{2}{3}$ với $a\in \left( \frac{5 \pi}{2};3\pi \right)$. Tính $\sin\left(a- \frac{2 \pi}{3}\right)$.\ 
\choice
{ $- \frac{\sqrt{15}}{6} - \frac{1}{3}$ }
   { $- \frac{\sqrt{3}}{3} + \frac{\sqrt{5}}{6}$ }
     { \True $- \frac{1}{3} + \frac{\sqrt{15}}{6}$ }
    { $\frac{2}{3} - \frac{\sqrt{5}}{3}$ }
\loigiai{ 
 Vì $a \in \left( \frac{5 \pi}{2};3\pi \right)$ nên $\cos a < 0$.

$\cos a =-\sqrt{1-\frac{4}{9}}=- \frac{\sqrt{5}}{3}$.

$\sin\left(a- \frac{2 \pi}{3}\right)=\sin a\cos (- \frac{2 \pi}{3})+\cos a \sin (- \frac{2 \pi}{3})=$$\frac{2}{3}.(- \frac{1}{2})+(- \frac{\sqrt{5}}{3}).(- \frac{\sqrt{3}}{2})=- \frac{1}{3} + \frac{\sqrt{15}}{6}$. 
 }\end{ex}

\begin{ex}
 Tìm tập xác định của hàm số $y=\tan(10x-5\pi)$.\\ 
\choice
{ \True $D=\mathbb{R}\backslash\{ \frac{11}{20}\pi + k \frac{1}{10}\pi\}$ }
   { $D=\mathbb{R}\backslash\{ \frac{3}{10}\pi + k \frac{1}{10}\pi\}$ }
     { $D=\mathbb{R}\backslash\{ \frac{3}{5}\pi + k \frac{1}{10}\pi\}$ }
    { $D=\mathbb{R}\backslash\{ \frac{11}{10}\pi + k \frac{1}{10}\pi\}$ }
\loigiai{ 
  
 }\end{ex}

\begin{ex}
 Nghiệm của phương trình $\cos\left(4 x + \frac{\pi}{2}\right)=\sin\left(- 3 x - \frac{\pi}{6}\right)$ là\ 
\choice
{ $x=\frac{5 \pi}{21}+k2 \pi, x=- \frac{\pi}{6}+k2 \pi (k\in \mathbb{Z})$ }
   { $x=\frac{\pi}{6}+k\frac{\pi}{7}, x=- \frac{\pi}{6}+k\pi (k\in \mathbb{Z})$ }
     { $x=\frac{5 \pi}{21}+k2 \pi, x=- \frac{\pi}{6}+k\frac{2 \pi}{7} (k\in \mathbb{Z})$ }
    { \True $x=\frac{\pi}{6}+k2 \pi, x=- \frac{\pi}{6}+k\frac{2 \pi}{7} (k\in \mathbb{Z})$ }
\loigiai{ 
 $\cos\left(4 x + \frac{\pi}{2}\right)=\sin\left(- 3 x - \frac{\pi}{6}\right) \Leftrightarrow \cos\left(4 x + \frac{\pi}{2}\right)=\cos\left(3 x + \frac{2 \pi}{3}\right)$

$\Leftrightarrow \left[ \begin{array}{l} 
        4 x + \frac{\pi}{2}=3 x + \frac{2 \pi}{3} +k2 \pi \\ 
        4 x + \frac{\pi}{2}=- 3 x - \frac{2 \pi}{3}+k2 \pi
        \end{array} \right.$

$\Leftrightarrow \left[ \begin{array}{l} 
        x=\frac{\pi}{6} +k2 \pi \\ 
        7 x=- \frac{7 \pi}{6}+k2 \pi
        \end{array} \right. $

$\Leftrightarrow \left[ \begin{array}{l} 
        x=\frac{\pi}{6} + k2 \pi \\ 
        x=- \frac{\pi}{6}+ k\frac{2 \pi}{7}
        \end{array} \right. , k\in \mathbb{Z} $

 
 }\end{ex}

\Closesolutionfile{ans}
{\bf PHẦN II. Câu trắc nghiệm đúng sai.}
\setcounter{ex}{0}
\Opensolutionfile{ans}[ans/ans007-2]
\begin{ex}
 Cho $\sin x=\frac{4}{7}, x\in \left( - \frac{3 \pi}{2};- \pi \right)$. Xét tính đúng-sai của các khẳng định sau.
\choiceTFt
{ \True $\cos x=- \frac{\sqrt{33}}{7}$ }
   { $\sin 2\beta=- \frac{4 \sqrt{33}}{49}$  }
     { $\cos 2\beta=- \frac{17}{49}$  }
    { \True $\sin\left(\beta+\frac{\pi}{6}\right)=- \frac{\sqrt{33}}{14} + \frac{2 \sqrt{3}}{7}$ }
\loigiai{ 
 

 a) Khẳng định đã cho là khẳng định đúng.

 Vì $x \in \left( - \frac{3 \pi}{2};- \pi \right)$ nên $\cos x < 0$.

$\cos x =-\sqrt{1-\frac{16}{49}}=- \frac{\sqrt{33}}{7}$.

b) Khẳng định đã cho là khẳng định sai.

 $\sin 2\beta=2\sin \beta \cos \beta=2.\frac{4}{7}.(- \frac{\sqrt{33}}{7})=- \frac{8 \sqrt{33}}{49}$.

c) Khẳng định đã cho là khẳng định sai.

 $\cos 2\beta=1-2\sin^2 \beta=1-2.\frac{16}{49}=\frac{17}{49}$

d) Khẳng định đã cho là khẳng định đúng.

 $\sin\left(\beta+\frac{\pi}{6}\right)=\sin \beta\cos (\frac{\pi}{6})+\cos \beta \sin (\frac{\pi}{6})=$$\frac{4}{7}.(\frac{\sqrt{3}}{2})+(- \frac{\sqrt{33}}{7}).(\frac{1}{2})=- \frac{\sqrt{33}}{14} + \frac{2 \sqrt{3}}{7}$.

 
 }\end{ex}

\begin{ex}
 Cho hàm số $y=4 \cos{\left(8 x \right)} - 6$ . Xét tính đúng-sai của các khẳng định sau. 
\choiceTFt
{ Tập xác định của hàm số là $D=[-4;4]$ }
   { Hàm số đã cho là hàm số lẻ }
     { Tập giá trị của hàm số đã cho là $T={[-14;-6]}$ }
    { \True  Đồ thị cắt trục tung tại điểm có tung độ bằng ${-2}$ }
\loigiai{ 
 

 a) Khẳng định đã cho là khẳng định sai.

 Tập xác định của hàm số là $D=\mathbb{R}$.

b) Khẳng định đã cho là khẳng định sai.

 Ta có: Với mọi $x\in \mathbb{R}$ thì $-x\in \mathbb{R}$.

$f(-x)=4 \cos{\left(8 x \right)} - 6=4 \cos{\left(8 x \right)} - 6$. Vậy hàm số $y=4 \cos{\left(8 x \right)} - 6$ là hàm số chẵn.

c) Khẳng định đã cho là khẳng định sai.

 Ta có: $-10 \le 4 \cos{\left(8 x \right)} - 6 \le -10$ nên tập giá trị là ${[-10;-10]}$

d) Khẳng định đã cho là khẳng định đúng.

 Cho $x=0\Rightarrow y=-2$. Suy ra đồ thị cắt trục tung tại điểm có tung độ bằng ${-2}$.

 
 }\end{ex}

\Closesolutionfile{ans}
{\bf PHẦN III. Câu trắc nghiệm trả lời ngắn.}
\setcounter{ex}{0}
\Opensolutionfile{ans}[ans/ans007-3]
\begin{ex}
 Một bánh xe của một loại xe có bán kính ${52}$ cm và quay được 5 vòng trong 5 giây. Tính độ dài quãng đường (theo đơn vị mét) xe đi được trong 4 giây (kết quả làm tròn đến hàng phần mười). 
\shortans[oly]{12,6}

\loigiai{ 
 Một giây bánh xe quay được số vòng là: $1$.

Một vòng quay ứng với quãng đường là $2\pi.0,5=1,0\pi$.

Sau ${4}$ giây quãng đường đi được là: $1.4.1,0\pi=12,6$:

 
 }\end{ex}

\begin{ex}
 Số nghiệm thuộc đoạn $[- 3 \pi;3 \pi]$ của phương trình $\tan \left(2 x - \frac{\pi}{3}\right)=1$ là
\shortans[oly]{12}

\loigiai{ 
 $\tan \left(2 x - \frac{\pi}{3}\right)=1 \Leftrightarrow 2 x - \frac{\pi}{3} =\frac{\pi}{4}+ k\pi \Leftrightarrow x=\frac{7 \pi}{24}+k\frac{\pi}{2}, k\in \mathbb{Z}$.

Do $x\in [- 3 \pi;3 \pi]$ nên $- 3 \pi\le \frac{7 \pi}{24}+k\frac{\pi}{2} \le 3 \pi \Rightarrow - \frac{79}{12}\le k \le \frac{65}{12}$.

Có ${12}$ số k thỏa mãn nên phương trình có ${12}$ nghiệm. 
 }\end{ex}

\Closesolutionfile{ans}

 \begin{center}
-----HẾT-----
\end{center}

 %\newpage 
%\begin{center}
%{\bf BẢNG ĐÁP ÁN MÃ ĐỀ 7 }
%\end{center}
%{\bf Phần 1 }
% \inputansbox{6}{ans007-1}
%{\bf Phần 2 }
% \inputansbox{2}{ans007-2}
%{\bf Phần 3 }
% \inputansbox{6}{ans007-3}
\newpage 



\end{document}