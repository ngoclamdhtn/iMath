\documentclass[12pt,a4paper]{article}
\usepackage[top=1.5cm, bottom=1.5cm, left=2.0cm, right=1.5cm] {geometry}
\usepackage{amsmath,amssymb,txfonts}
\usepackage{tkz-euclide}
\usepackage{setspace}
\usepackage{lastpage}

\usepackage{tikz,tkz-tab}
%\usepackage[solcolor]{ex_test}
%\usepackage[dethi]{ex_test} % Chỉ hiển thị đề thi
\usepackage[loigiai]{ex_test} % Hiển thị lời giải
%\usepackage[color]{ex_test} % Khoanh các đáp án
\everymath{\displaystyle}

\def\colorEX{\color{purple}}
%\def\colorEX{}%Không tô màu đáp án đúng trong tùy chọn loigiai
\renewtheorem{ex}{\color{violet}Câu}
\renewcommand{\FalseEX}{\stepcounter{dapan}{{\bf \textcolor{blue}{\Alph{dapan}.}}}}
\renewcommand{\TrueEX}{\stepcounter{dapan}{{\bf \textcolor{blue}{\Alph{dapan}.}}}}

%---------- Khai báo viết tắt, in đáp án
\newcommand{\hoac}[1]{ %hệ hoặc
    \left[\begin{aligned}#1\end{aligned}\right.}
\newcommand{\heva}[1]{ %hệ và
    \left\{\begin{aligned}#1\end{aligned}\right.}

%Tiêu đề
\newcommand{\tenso}{iMath}
\newcommand{\tentruong}{Phần mềm Tạo đề ngẫu nhiên}
\newcommand{\tenkythi}{ĐỀ ÔN TẬP}
\newcommand{\tenmonthi}{Môn thi: Toán}
\newcommand{\thoigian}{}
\newcommand{\tieude}[1]{
   \begin{tabular}{cm{3cm}cm{3cm}cm{3cm}}
    {\bf \tenso} & & {\bf \tenkythi} \\
    {\bf \tentruong} & & {\bf \tenmonthi}\\
    && {\bf Thời gian: \bf \thoigian \, phút}\\
    && { \fbox{\bf Mã đề: #1}}
   \end{tabular}\\\\
    
   {Họ tên HS: \dotfill Số báo danh \dotfill}\\
}
\newcommand{\chantrang}[2]{\rfoot{Trang \thepage $-$ Mã đề #2}}
\pagestyle{fancy}
\fancyhf{}
\renewcommand{\headrulewidth}{0pt} 
\renewcommand{\footrulewidth}{0pt}

\begin{document}
%Thiết lập giãn dọng 1.5cm 
%\setlength{\lineskip}{1.5em}
%Nội dung trắc nghiệm bắt đầu ở đây


\tieude{002}
\chantrang{\pageref{LastPage}}{002}
\setcounter{page}{1}
{\bf PHẦN I. Câu trắc nghiệm nhiều phương án lựa chọn.}
\setcounter{ex}{0}
\Opensolutionfile{ans}[ans/ans002-1]
\begin{ex}
 Đổi số đo của góc $720^\circ$ sang radian ta được kết quả bằng\\ 
\choice
{ \True $4 \pi$ }
   { $\frac{25 \pi}{6}$ }
     { $\frac{73 \pi}{18}$ }
    { $\frac{35 \pi}{9}$ }
\loigiai{ 
 Áp dụng công thức chuyển đổi: $720^\circ=\dfrac{720.\pi}{180}=4 \pi$. 
 }\end{ex}

\begin{ex}
 Tính $\cos\frac{2 \pi}{3}$.\\ 
\choice
{ $ - \sqrt{3} $ }
   { \True $ - \frac{1}{2} $ }
     { $ - \frac{\sqrt{3}}{3} $ }
    { $ \frac{\sqrt{3}}{2} $ }
\loigiai{ 
  
 }\end{ex}

\begin{ex}
 Cho ${x}$ là góc lượng giác. Tìm khẳng định đúng trong các khẳng định sau.\ 
\choice
{ $\tan (\pi-x)=\cot x$ }
   { $\cos (\pi-x)=\cos x$ }
     { \True $\sin (\pi-x)=\sin x$ }
    { $\sin (\pi-x)=\cos x$ }
\loigiai{ 
 $\sin (\pi-x)=\sin x$ là khẳng định đúng. 
 }\end{ex}

\begin{ex}
 Cho ${x}$ là góc lượng giác. Tìm khẳng định đúng trong các khẳng định sau.\ 
\choice
{ \True $\cos 2x=2\cos^2 x-1$ }
   { $\cos 2x=\sin^2 x-\cos^2 x$ }
     { $\sin 2x=\sin x\cos x$ }
    { $\tan 2x=\dfrac{\tan x}{1-\tan^2 x}$ }
\loigiai{ 
 $\cos 2x=2\cos^2 x-1$ là khẳng định đúng. 
 }\end{ex}

\begin{ex}
 Cho ${u,v}$ là các góc lượng giác. Tìm khẳng định đúng trong các khẳng định sau.\ 
\choice
{ $\cos u \cos v=-\dfrac 1 2[\cos(u+v) + \cos(u-v)]$ }
   { $\sin u \sin v=\dfrac 1 2[\cos(u+v) - \cos(u-v)]$ }
     { $\sin u \cos v=\dfrac 1 2[\sin(u+v) - \sin(u-v)]$ }
    { \True $\sin u \sin v=\dfrac 1 2[\cos(u-v) - \cos(u+v)]$ }
\loigiai{ 
 $\sin u \sin v=\dfrac 1 2[\cos(u-v) - \cos(u+v)]$ là khẳng định đúng. 
 }\end{ex}

\begin{ex}
 Cho $\sin \alpha=\frac{4}{7}$ với $\alpha\in \left( 0;\frac{\pi}{2} \right)$. Tính $\sin\left(\alpha- \frac{\pi}{4}\right)$.\ 
\choice
{ $\frac{4}{7} + \frac{\sqrt{33}}{7}$ }
   { \True $- \frac{\sqrt{66}}{14} + \frac{2 \sqrt{2}}{7}$ }
     { $- \frac{2 \sqrt{2}}{7} + \frac{\sqrt{66}}{14}$ }
    { $\frac{2 \sqrt{2}}{7} + \frac{\sqrt{66}}{14}$ }
\loigiai{ 
 Vì $\alpha \in \left( 0;\frac{\pi}{2} \right)$ nên $\cos \alpha > 0$.

$\cos \alpha =\sqrt{1-\frac{16}{49}}=\frac{\sqrt{33}}{7}$.

$\sin\left(\alpha- \frac{\pi}{4}\right)=\sin \alpha\cos (- \frac{\pi}{4})+\cos \alpha \sin (- \frac{\pi}{4})=$$\frac{4}{7}.(\frac{\sqrt{2}}{2})+\frac{\sqrt{33}}{7}.(- \frac{\sqrt{2}}{2})=- \frac{\sqrt{66}}{14} + \frac{2 \sqrt{2}}{7}$. 
 }\end{ex}

\begin{ex}
 Tìm tập xác định của hàm số $y=\tan(7x-5\pi)$.\\ 
\choice
{ $D=\mathbb{R}\backslash\{ \frac{6}{7}\pi + k \frac{1}{7}\pi\}$ }
   { \True $D=\mathbb{R}\backslash\{ \frac{11}{14}\pi + k \frac{1}{7}\pi\}$ }
     { $D=\mathbb{R}\backslash\{ \frac{11}{7}\pi + k \frac{1}{7}\pi\}$ }
    { $D=\mathbb{R}\backslash\{ \frac{3}{7}\pi + k \frac{1}{7}\pi\}$ }
\loigiai{ 
  
 }\end{ex}

\begin{ex}
 Nghiệm của phương trình $\cos\left(6 x - \frac{\pi}{2}\right)=\sin\left(- 3 x + \frac{5 \pi}{4}\right)$ là\ 
\choice
{ $x=- \frac{7 \pi}{36}+k\frac{2 \pi}{3}, x=\frac{\pi}{12}+k\frac{2 \pi}{9} (k\in \mathbb{Z})$ }
   { $x=- \frac{7 \pi}{36}+k2 \pi, x=\frac{\pi}{12}+k2 \pi (k\in \mathbb{Z})$ }
     { \True $x=- \frac{\pi}{12}+k\frac{2 \pi}{3}, x=\frac{5 \pi}{36}+k\frac{2 \pi}{9} (k\in \mathbb{Z})$ }
    { $x=- \frac{\pi}{12}+k\frac{\pi}{9}, x=\frac{5 \pi}{36}+k\frac{\pi}{3} (k\in \mathbb{Z})$ }
\loigiai{ 
 $\cos\left(6 x - \frac{\pi}{2}\right)=\sin\left(- 3 x + \frac{5 \pi}{4}\right) \Leftrightarrow \cos\left(6 x - \frac{\pi}{2}\right)=\cos\left(3 x - \frac{3 \pi}{4}\right)$

$\Leftrightarrow \left[ \begin{array}{l} 
        6 x - \frac{\pi}{2}=3 x - \frac{3 \pi}{4} +k2 \pi \\ 
        6 x - \frac{\pi}{2}=- 3 x + \frac{3 \pi}{4}+k2 \pi
        \end{array} \right.$

$\Leftrightarrow \left[ \begin{array}{l} 
        3 x=- \frac{\pi}{4} +k2 \pi \\ 
        9 x=\frac{5 \pi}{4}+k2 \pi
        \end{array} \right. $

$\Leftrightarrow \left[ \begin{array}{l} 
        x=- \frac{\pi}{12} + k\frac{2 \pi}{3} \\ 
        x=\frac{5 \pi}{36}+ k\frac{2 \pi}{9}
        \end{array} \right. , k\in \mathbb{Z} $

 
 }\end{ex}

\Closesolutionfile{ans}
{\bf PHẦN II. Câu trắc nghiệm đúng sai.}
\setcounter{ex}{0}
\Opensolutionfile{ans}[ans/ans002-2]
\begin{ex}
 Cho $\sin x=\frac{\sqrt{6}}{7}, x\in \left( \frac{5 \pi}{2};3\pi \right)$. Xét tính đúng-sai của các khẳng định sau.
\choiceTFt
{ $\cos x=\frac{\sqrt{43}}{7}$ }
   { \True $\sin 2x=- \frac{2 \sqrt{258}}{49}$ }
     { $\cos 2x=- \frac{37}{49}$  }
    { \True $\sin\left(x- \frac{3 \pi}{4}\right)=- \frac{\sqrt{3}}{7} + \frac{\sqrt{86}}{14}$ }
\loigiai{ 
 

 a) Khẳng định đã cho là khẳng định sai.

 Vì $x \in \left( \frac{5 \pi}{2};3\pi \right)$ nên $\cos x < 0$.

$\cos x =-\sqrt{1-\frac{6}{49}}=- \frac{\sqrt{43}}{7}$.

b) Khẳng định đã cho là khẳng định đúng.

 $\sin 2x=2\sin x \cos x=2.\frac{\sqrt{6}}{7}.(- \frac{\sqrt{43}}{7})=- \frac{2 \sqrt{258}}{49}$.

c) Khẳng định đã cho là khẳng định sai.

 $\cos 2x=1-2\sin^2 x=1-2.\frac{6}{49}=\frac{37}{49}$

d) Khẳng định đã cho là khẳng định đúng.

 $\sin\left(x- \frac{3 \pi}{4}\right)=\sin x\cos (- \frac{3 \pi}{4})+\cos x \sin (- \frac{3 \pi}{4})=$$\frac{\sqrt{6}}{7}.(- \frac{\sqrt{2}}{2})+(- \frac{\sqrt{43}}{7}).(- \frac{\sqrt{2}}{2})=- \frac{\sqrt{3}}{7} + \frac{\sqrt{86}}{14}$.

 
 }\end{ex}

\begin{ex}
 Cho hàm số $y=6 \cos{\left(4 x \right)} - 2$ . Xét tính đúng-sai của các khẳng định sau. 
\choiceTFt
{ Tập xác định của hàm số là $D=[-6;6]$ }
   { Hàm số đã cho là hàm số lẻ }
     { \True  Tập giá trị của hàm số đã cho là $T={[-8;-8]}$ }
    {  Đồ thị cắt trục tung tại điểm có tung độ bằng ${6}$ }
\loigiai{ 
 

 a) Khẳng định đã cho là khẳng định sai.

 Tập xác định của hàm số là $D=\mathbb{R}$.

b) Khẳng định đã cho là khẳng định sai.

 Ta có: Với mọi $x\in \mathbb{R}$ thì $-x\in \mathbb{R}$.

$f(-x)=6 \cos{\left(4 x \right)} - 2=6 \cos{\left(4 x \right)} - 2$. Vậy hàm số $y=6 \cos{\left(4 x \right)} - 2$ là hàm số chẵn.

c) Khẳng định đã cho là khẳng định đúng.

 Ta có: $-8 \le 6 \cos{\left(4 x \right)} - 2 \le -8$ nên tập giá trị là ${[-8;-8]}$

d) Khẳng định đã cho là khẳng định sai.

 Cho $x=0\Rightarrow y=4$. Suy ra đồ thị cắt trục tung tại điểm có tung độ bằng ${4}$.

 
 }\end{ex}

\Closesolutionfile{ans}
{\bf PHẦN III. Câu trắc nghiệm trả lời ngắn.}
\setcounter{ex}{0}
\Opensolutionfile{ans}[ans/ans002-3]
\begin{ex}
 Một bánh xe của một loại xe có bán kính ${44}$ cm và quay được 8 vòng trong 6 giây. Tính độ dài quãng đường (theo đơn vị mét) xe đi được trong 5 giây (kết quả làm tròn đến hàng phần mười). 
\shortans[oly]{16,8}

\loigiai{ 
 Một giây bánh xe quay được số vòng là: $\frac{4}{3}$.

Một vòng quay ứng với quãng đường là $2\pi.0,4=0,8\pi$.

Sau ${5}$ giây quãng đường đi được là: $\frac{4}{3}.5.0,8\pi=16,8$:

 
 }\end{ex}

\begin{ex}
 Số nghiệm thuộc khoảng $(- \pi;\pi)$ của phương trình $\tan \left(4 x - \frac{\pi}{2}\right)=0$ là
\shortans[oly]{8}

\loigiai{ 
 $\tan \left(4 x - \frac{\pi}{2}\right)=0 \Leftrightarrow 4 x - \frac{\pi}{2} =0+ k\pi \Leftrightarrow x=\frac{\pi}{8}+k\frac{\pi}{4}, k\in \mathbb{Z}$.

Do $x\in (- \pi;\pi)$ nên $- \pi< \frac{\pi}{8}+k\frac{\pi}{4} < \pi \Rightarrow - \frac{9}{2}< k < \frac{7}{2}$.

Có ${8}$ số k thỏa mãn nên phương trình có ${8}$ nghiệm. 
 }\end{ex}

\Closesolutionfile{ans}

 \begin{center}
-----HẾT-----
\end{center}

 %\newpage 
%\begin{center}
%{\bf BẢNG ĐÁP ÁN MÃ ĐỀ 2 }
%\end{center}
%{\bf Phần 1 }
% \inputansbox{6}{ans002-1}
%{\bf Phần 2 }
% \inputansbox{2}{ans002-2}
%{\bf Phần 3 }
% \inputansbox{6}{ans002-3}
\newpage 



\end{document}