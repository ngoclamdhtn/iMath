\documentclass[12pt,a4paper]{article}
\usepackage[top=1.5cm, bottom=1.5cm, left=2.0cm, right=1.5cm] {geometry}
\usepackage{amsmath,amssymb,txfonts}
\usepackage{tkz-euclide}
\usepackage{setspace}
\usepackage{lastpage}

\usepackage{tikz,tkz-tab}
%\usepackage[solcolor]{ex_test}
%\usepackage[dethi]{ex_test} % Chỉ hiển thị đề thi
\usepackage[loigiai]{ex_test} % Hiển thị lời giải
%\usepackage[color]{ex_test} % Khoanh các đáp án
\everymath{\displaystyle}

\def\colorEX{\color{purple}}
%\def\colorEX{}%Không tô màu đáp án đúng trong tùy chọn loigiai
\renewtheorem{ex}{\color{violet}Câu}
\renewcommand{\FalseEX}{\stepcounter{dapan}{{\bf \textcolor{blue}{\Alph{dapan}.}}}}
\renewcommand{\TrueEX}{\stepcounter{dapan}{{\bf \textcolor{blue}{\Alph{dapan}.}}}}

%---------- Khai báo viết tắt, in đáp án
\newcommand{\hoac}[1]{ %hệ hoặc
    \left[\begin{aligned}#1\end{aligned}\right.}
\newcommand{\heva}[1]{ %hệ và
    \left\{\begin{aligned}#1\end{aligned}\right.}

%Tiêu đề
\newcommand{\tenso}{iMath}
\newcommand{\tentruong}{Phần mềm Tạo đề ngẫu nhiên}
\newcommand{\tenkythi}{ĐỀ ÔN TẬP}
\newcommand{\tenmonthi}{Môn thi: Toán}
\newcommand{\thoigian}{}
\newcommand{\tieude}[1]{
   \begin{tabular}{cm{3cm}cm{3cm}cm{3cm}}
    {\bf \tenso} & & {\bf \tenkythi} \\
    {\bf \tentruong} & & {\bf \tenmonthi}\\
    && {\bf Thời gian: \bf \thoigian \, phút}\\
    && { \fbox{\bf Mã đề: #1}}
   \end{tabular}\\\\
    
   {Họ tên HS: \dotfill Số báo danh \dotfill}\\
}
\newcommand{\chantrang}[2]{\rfoot{Trang \thepage $-$ Mã đề #2}}
\pagestyle{fancy}
\fancyhf{}
\renewcommand{\headrulewidth}{0pt} 
\renewcommand{\footrulewidth}{0pt}

\begin{document}
%Thiết lập giãn dọng 1.5cm 
%\setlength{\lineskip}{1.5em}
%Nội dung trắc nghiệm bắt đầu ở đây


\tieude{010}
\chantrang{\pageref{LastPage}}{010}
\setcounter{page}{1}
{\bf PHẦN I. Câu trắc nghiệm nhiều phương án lựa chọn.}
\setcounter{ex}{0}
\Opensolutionfile{ans}[ans/ans010-1]
\begin{ex}
 Đổi số đo của góc $735^\circ$ sang radian ta được kết quả bằng\\ 
\choice
{ \True $\frac{49 \pi}{12}$ }
   { $\frac{17 \pi}{4}$ }
     { $\frac{143 \pi}{36}$ }
    { $\frac{149 \pi}{36}$ }
\loigiai{ 
 Áp dụng công thức chuyển đổi: $735^\circ=\dfrac{735.\pi}{180}=\frac{49 \pi}{12}$. 
 }\end{ex}

\begin{ex}
 Tính $\sin\frac{103 \pi}{3}$.\\ 
\choice
{ $ \frac{1}{2} $ }
   { $ \sqrt{3} $ }
     { $ \frac{\sqrt{3}}{3} $ }
    { \True $ \frac{\sqrt{3}}{2} $ }
\loigiai{ 
  
 }\end{ex}

\begin{ex}
 Cho ${b}$ là góc lượng giác. Tìm khẳng định đúng trong các khẳng định sau.\ 
\choice
{ $\cot (\pi+b)=-\cot b$ }
   { \True $\cos (\pi-b)=-\cos b$ }
     { $\sin \left(\frac{\pi}{2}-b\right)=\sin b$ }
    { $\cos \left(\frac{\pi}{2}-b\right)=-\cos b$ }
\loigiai{ 
 $\cos (\pi-b)=-\cos b$ là khẳng định đúng. 
 }\end{ex}

\begin{ex}
 Cho ${\alpha}$ là góc lượng giác. Tìm khẳng định đúng trong các khẳng định sau.\ 
\choice
{ $\tan 2\alpha=\dfrac{\tan \alpha}{1-2\tan^2 \alpha}$ }
   { $\cos 2\alpha=2\sin \alpha\cos \alpha$ }
     { \True $\cos 2\alpha=2\cos^2 \alpha-1$ }
    { $\sin 2\alpha=\sin \alpha+\cos \alpha$ }
\loigiai{ 
 $\cos 2\alpha=2\cos^2 \alpha-1$ là khẳng định đúng. 
 }\end{ex}

\begin{ex}
 Cho ${x,y}$ là các góc lượng giác. Tìm khẳng định đúng trong các khẳng định sau.\ 
\choice
{ $\cos x \cos y=\dfrac 1 2[\cos(x+y) - \cos(x-y)]$ }
   { $\sin x \cos y=\dfrac 1 2[\sin(x+y) - \sin(x-y)]$ }
     { \True $\sin x \sin y=\dfrac 1 2[\cos(x-y) - \cos(x+y)]$ }
    { $\sin x \sin y=\dfrac 1 2[\cos(x+y) - \cos(x-y)]$ }
\loigiai{ 
 $\sin x \sin y=\dfrac 1 2[\cos(x-y) - \cos(x+y)]$ là khẳng định đúng. 
 }\end{ex}

\begin{ex}
 Cho $\sin \beta=\frac{4}{7}$ với $\beta\in \left( - \frac{3 \pi}{2};- \pi \right)$. Tính $\sin\left(\beta+\frac{3 \pi}{4}\right)$.\ 
\choice
{ $\frac{2 \sqrt{2}}{7} + \frac{\sqrt{66}}{14}$ }
   { \True $- \frac{\sqrt{66}}{14} - \frac{2 \sqrt{2}}{7}$ }
     { $- \frac{2 \sqrt{2}}{7} + \frac{\sqrt{66}}{14}$ }
    { $\frac{4}{7} - \frac{\sqrt{33}}{7}$ }
\loigiai{ 
 Vì $\beta \in \left( - \frac{3 \pi}{2};- \pi \right)$ nên $\cos \beta < 0$.

$\cos \beta =-\sqrt{1-\frac{16}{49}}=- \frac{\sqrt{33}}{7}$.

$\sin\left(\beta+\frac{3 \pi}{4}\right)=\sin \beta\cos (\frac{3 \pi}{4})+\cos \beta \sin (\frac{3 \pi}{4})=$$\frac{4}{7}.(- \frac{\sqrt{2}}{2})+(- \frac{\sqrt{33}}{7}).(\frac{\sqrt{2}}{2})=- \frac{\sqrt{66}}{14} - \frac{2 \sqrt{2}}{7}$. 
 }\end{ex}

\begin{ex}
 Tìm tập xác định của hàm số $y=\tan(10x+5\pi)$.\\ 
\choice
{ $D=\mathbb{R}\backslash\{ - \frac{1}{5}\pi + k \frac{1}{10}\pi\}$ }
   { \True $D=\mathbb{R}\backslash\{ - \frac{9}{20}\pi + k \frac{1}{10}\pi\}$ }
     { $D=\mathbb{R}\backslash\{ - \frac{9}{10}\pi + k \frac{1}{10}\pi\}$ }
    { $D=\mathbb{R}\backslash\{ - \frac{2}{5}\pi + k \frac{1}{10}\pi\}$ }
\loigiai{ 
  
 }\end{ex}

\begin{ex}
 Nghiệm của phương trình $\cos\left(3 x + \frac{\pi}{4}\right)=\sin\left(- 2 x + \frac{5 \pi}{4}\right)$ là\ 
\choice
{ \True $x=- \pi+k2 \pi, x=\frac{\pi}{10}+k\frac{2 \pi}{5} (k\in \mathbb{Z})$ }
   { $x=- \frac{\pi}{20}+k2 \pi, x=\pi+k2 \pi (k\in \mathbb{Z})$ }
     { $x=- \frac{\pi}{20}+k2 \pi, x=\pi+k\frac{2 \pi}{5} (k\in \mathbb{Z})$ }
    { $x=- \pi+k\frac{\pi}{5}, x=\frac{\pi}{10}+k\pi (k\in \mathbb{Z})$ }
\loigiai{ 
 $\cos\left(3 x + \frac{\pi}{4}\right)=\sin\left(- 2 x + \frac{5 \pi}{4}\right) \Leftrightarrow \cos\left(3 x + \frac{\pi}{4}\right)=\cos\left(2 x - \frac{3 \pi}{4}\right)$

$\Leftrightarrow \left[ \begin{array}{l} 
        3 x + \frac{\pi}{4}=2 x - \frac{3 \pi}{4} +k2 \pi \\ 
        3 x + \frac{\pi}{4}=- 2 x + \frac{3 \pi}{4}+k2 \pi
        \end{array} \right.$

$\Leftrightarrow \left[ \begin{array}{l} 
        x=- \pi +k2 \pi \\ 
        5 x=\frac{\pi}{2}+k2 \pi
        \end{array} \right. $

$\Leftrightarrow \left[ \begin{array}{l} 
        x=- \pi + k2 \pi \\ 
        x=\frac{\pi}{10}+ k\frac{2 \pi}{5}
        \end{array} \right. , k\in \mathbb{Z} $

 
 }\end{ex}

\Closesolutionfile{ans}
{\bf PHẦN II. Câu trắc nghiệm đúng sai.}
\setcounter{ex}{0}
\Opensolutionfile{ans}[ans/ans010-2]
\begin{ex}
 Cho $\sin x=\frac{7}{8}, x\in \left( \frac{5 \pi}{2};3\pi \right)$. Xét tính đúng-sai của các khẳng định sau.
\choiceTFt
{ \True $\cos x=- \frac{\sqrt{15}}{8}$ }
   { \True $\sin 2\alpha=- \frac{7 \sqrt{15}}{32}$ }
     { $\cos 2\alpha=\frac{17}{32}$  }
    { $\sin\left(\alpha+\frac{\pi}{3}\right)=\frac{7}{8} - \frac{\sqrt{15}}{8}$ }
\loigiai{ 
 

 a) Khẳng định đã cho là khẳng định đúng.

 Vì $x \in \left( \frac{5 \pi}{2};3\pi \right)$ nên $\cos x < 0$.

$\cos x =-\sqrt{1-\frac{49}{64}}=- \frac{\sqrt{15}}{8}$.

b) Khẳng định đã cho là khẳng định đúng.

 $\sin 2\alpha=2\sin \alpha \cos \alpha=2.\frac{7}{8}.(- \frac{\sqrt{15}}{8})=- \frac{7 \sqrt{15}}{32}$.

c) Khẳng định đã cho là khẳng định sai.

 $\cos 2\alpha=1-2\sin^2 \alpha=1-2.\frac{49}{64}=- \frac{17}{32}$

d) Khẳng định đã cho là khẳng định sai.

 $\sin\left(\alpha+\frac{\pi}{3}\right)=\sin \alpha\cos (\frac{\pi}{3})+\cos \alpha \sin (\frac{\pi}{3})=$$\frac{7}{8}.(\frac{1}{2})+(- \frac{\sqrt{15}}{8}).(\frac{\sqrt{3}}{2})=\frac{7}{16} - \frac{3 \sqrt{5}}{16}$.

 
 }\end{ex}

\begin{ex}
 Cho hàm số $y=4 \sin{\left(6 x \right)} - 6$ . Xét tính đúng-sai của các khẳng định sau. 
\choiceTFt
{ Tập xác định của hàm số là $D=[-4;4]$ }
   { \True  Hàm số đã cho là hàm số không chẵn, không lẻ }
     { \True  Tập giá trị của hàm số đã cho là $T={[-10;-2]}$ }
    {  Đồ thị cắt trục tung tại điểm có tung độ bằng ${-5}$ }
\loigiai{ 
 

 a) Khẳng định đã cho là khẳng định sai.

 Tập xác định của hàm số là $D=\mathbb{R}$.

b) Khẳng định đã cho là khẳng định đúng.

 Ta có: Với mọi $x\in \mathbb{R}$ thì $-x\in \mathbb{R}$.

$f(-x)=- 4 \sin{\left(6 x \right)} - 6\ne f(x), f(-x)\ne -f(x)$.

Vậy hàm số $y=4 \sin{\left(6 x \right)} - 6$ là hàm số không chẵn, không lẻ.

c) Khẳng định đã cho là khẳng định đúng.

 Ta có: $-10 \le 4 \sin{\left(6 x \right)} - 6 \le -2$ nên tập giá trị là ${[-10;-2]}$

d) Khẳng định đã cho là khẳng định sai.

 Cho $x=0\Rightarrow y=-6$. Suy ra đồ thị cắt trục tung tại điểm có tung độ bằng ${-6}$.

 
 }\end{ex}

\Closesolutionfile{ans}
{\bf PHẦN III. Câu trắc nghiệm trả lời ngắn.}
\setcounter{ex}{0}
\Opensolutionfile{ans}[ans/ans010-3]
\begin{ex}
 Một bánh xe của một loại xe có bán kính ${57}$ cm và quay được 8 vòng trong 5 giây. Tính độ dài quãng đường (theo đơn vị mét) xe đi được trong 9 giây (kết quả làm tròn đến hàng phần mười). 
\shortans[oly]{54,3}

\loigiai{ 
 Một giây bánh xe quay được số vòng là: $\frac{8}{5}$.

Một vòng quay ứng với quãng đường là $2\pi.0,6=1,2\pi$.

Sau ${9}$ giây quãng đường đi được là: $\frac{8}{5}.9.1,2\pi=54,3$:

 
 }\end{ex}

\begin{ex}
 Số nghiệm thuộc khoảng $(- 5 \pi;5 \pi)$ của phương trình $\tan \left(x - \frac{\pi}{5}\right)=\frac{\sqrt{3}}{3}$ là
\shortans[oly]{10}

\loigiai{ 
 $\tan \left(x - \frac{\pi}{5}\right)=\frac{\sqrt{3}}{3} \Leftrightarrow x - \frac{\pi}{5} =\frac{\pi}{6}+ k\pi \Leftrightarrow x=\frac{11 \pi}{30}+k\pi, k\in \mathbb{Z}$.

Do $x\in (- 5 \pi;5 \pi)$ nên $- 5 \pi< \frac{11 \pi}{30}+k\pi < 5 \pi \Rightarrow - \frac{161}{30}< k < \frac{139}{30}$.

Có ${10}$ số k thỏa mãn nên phương trình có ${10}$ nghiệm. 
 }\end{ex}

\Closesolutionfile{ans}

 \begin{center}
-----HẾT-----
\end{center}

 %\newpage 
%\begin{center}
%{\bf BẢNG ĐÁP ÁN MÃ ĐỀ 10 }
%\end{center}
%{\bf Phần 1 }
% \inputansbox{6}{ans010-1}
%{\bf Phần 2 }
% \inputansbox{2}{ans010-2}
%{\bf Phần 3 }
% \inputansbox{6}{ans010-3}
\newpage 



\end{document}