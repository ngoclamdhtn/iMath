\documentclass[12pt,a4paper]{article}
\usepackage[top=1.5cm, bottom=1.5cm, left=2.0cm, right=1.5cm] {geometry}
\usepackage{amsmath,amssymb,txfonts}
\usepackage{tkz-euclide}
\usepackage{setspace}
\usepackage{lastpage}

\usepackage{tikz,tkz-tab}
%\usepackage[solcolor]{ex_test}
\usepackage[dethi]{ex_test} % Chỉ hiển thị đề thi
%\usepackage[loigiai]{ex_test} % Hiển thị lời giải
%\usepackage[color]{ex_test} % Khoanh các đáp án
\everymath{\displaystyle}

\def\colorEX{\color{purple}}
%\def\colorEX{}%Không tô màu đáp án đúng trong tùy chọn loigiai
\renewtheorem{ex}{\color{violet}Câu}
\renewcommand{\FalseEX}{\stepcounter{dapan}{{\bf \textcolor{blue}{\Alph{dapan}.}}}}
\renewcommand{\TrueEX}{\stepcounter{dapan}{{\bf \textcolor{blue}{\Alph{dapan}.}}}}

%---------- Khai báo viết tắt, in đáp án
\newcommand{\hoac}[1]{ %hệ hoặc
    \left[\begin{aligned}#1\end{aligned}\right.}
\newcommand{\heva}[1]{ %hệ và
    \left\{\begin{aligned}#1\end{aligned}\right.}

%Tiêu đề
\newcommand{\tenso}{iMath}
\newcommand{\tentruong}{Phần mềm Tạo đề ngẫu nhiên}
\newcommand{\tenkythi}{ĐỀ ÔN TẬP}
\newcommand{\tenmonthi}{Môn học: Toán}
\newcommand{\thoigian}{}
\newcommand{\tieude}[1]{
    \noindent
     \begin{minipage}[b]{6cm}
    \centerline{\textbf{\fontsize{11}{0}\selectfont \tenso}}
    \centerline{\fontsize{11}{0}\selectfont \tentruong}  
  \end{minipage}\hspace{1cm}
  \begin{minipage}[b]{11cm}
    \centerline{\textbf{\fontsize{11}{0}\selectfont \tenkythi}}
    \centerline{\textbf{\fontsize{11}{0}\selectfont \tenmonthi}}
    \centerline{\textit{\fontsize{11}{0}\selectfont Thời \underline{gian làm bài: \thoigian  } phút }}
  \end{minipage}
  \vspace*{3mm}
  \noindent
  \begin{minipage}[t]{12cm}
    \textbf{Họ, tên thí sinh:}\dotfill\\
    \textbf{Số báo danh:}\dotfill
  \end{minipage}\hfill
  \begin{minipage}[b]{3cm}
    \setlength\fboxrule{1pt}
    \setlength\fboxsep{3pt}
    \vspace*{3mm}\fbox{\bf Mã đề thi #1}
  \end{minipage}\\
}

\newcommand{\chantrang}[2]{\rfoot{Trang \thepage $-$ Mã đề #2}}
\pagestyle{fancy}
\fancyhf{}
\renewcommand{\headrulewidth}{0pt} 
\renewcommand{\footrulewidth}{0pt}

\begin{document}
%Thiết lập giãn dọng 1.5cm 
%\setlength{\lineskip}{1.5em}


%Nội dung trắc nghiệm bắt đầu ở đây


\tieude{009}
\chantrang{\pageref{LastPage}}{009}
\setcounter{page}{1}
{\bf PHẦN I. Câu trắc nghiệm nhiều phương án lựa chọn.}
\setcounter{ex}{0}
\Opensolutionfile{ans}[ans/ans009-1]
\begin{ex}
 Đổi số đo của góc $-345^\circ$ sang radian ta được kết quả bằng\\ 
\choice
{ $- \frac{7 \pi}{4}$ }
   { $- \frac{73 \pi}{36}$ }
     { $- \frac{67 \pi}{36}$ }
    { \True $- \frac{23 \pi}{12}$ }
\loigiai{ 
 Áp dụng công thức chuyển đổi: $-345^\circ=\dfrac{-345.\pi}{180}=- \frac{23 \pi}{12}$. 
 }\end{ex}

\begin{ex}
 Tính $\cot\frac{25 \pi}{3}$.\\ 
\choice
{ \True $ \frac{\sqrt{3}}{3} $ }
   { $ \sqrt{3} $ }
     { $ \frac{1}{2} $ }
    { $ \frac{\sqrt{3}}{2} $ }
\loigiai{ 
  
 }\end{ex}

\begin{ex}
 Cho ${x}$ là góc lượng giác. Tìm khẳng định đúng trong các khẳng định sau.\ 
\choice
{ $\cos (\pi-x)=\cos x$ }
   { \True $\cos (\pi-x)=-\cos x$ }
     { $\tan (\pi+x)=-\tan x$ }
    { $\sin (\pi+x)=\sin x$ }
\loigiai{ 
 $\cos (\pi-x)=-\cos x$ là khẳng định đúng. 
 }\end{ex}

\begin{ex}
 Cho ${b}$ là góc lượng giác. Tìm khẳng định đúng trong các khẳng định sau.\ 
\choice
{ $\tan 2b=\dfrac{\tan b}{1-2\tan^2 b}$ }
   { $\cos 2b=2\sin^2 b-1$ }
     { $\sin 2b=\sin b+\cos b$ }
    { \True $\cos 2b=\cos^2 b-\sin^2 b$ }
\loigiai{ 
 $\cos 2b=\cos^2 b-\sin^2 b$ là khẳng định đúng. 
 }\end{ex}

\begin{ex}
 Cho ${a,b}$ là các góc lượng giác. Tìm khẳng định đúng trong các khẳng định sau.\ 
\choice
{ $\sin a \cos b=\dfrac 1 2[\sin(a+b) - \sin(a-b)]$ }
   { \True $\sin a \sin b=\dfrac 1 2[\cos(a-b) - \cos(a+b)]$ }
     { $\sin a \sin b=-\dfrac 1 2[\cos(a-b) - \cos(a+b)]$ }
    { $\cos a \cos b=\dfrac 1 2[\cos(a+b) - \cos(a-b)]$ }
\loigiai{ 
 $\sin a \sin b=\dfrac 1 2[\cos(a-b) - \cos(a+b)]$ là khẳng định đúng. 
 }\end{ex}

\begin{ex}
 Cho $\sin \alpha=\frac{2}{3}$ với $\alpha\in \left( - \frac{3 \pi}{2};- \pi \right)$. Tính $\sin\left(\alpha+\frac{2 \pi}{3}\right)$.\ 
\choice
{ \True $- \frac{\sqrt{15}}{6} - \frac{1}{3}$ }
   { $- \frac{1}{3} + \frac{\sqrt{15}}{6}$ }
     { $\frac{\sqrt{5}}{6} + \frac{\sqrt{3}}{3}$ }
    { $\frac{2}{3} - \frac{\sqrt{5}}{3}$ }
\loigiai{ 
 Vì $\alpha \in \left( - \frac{3 \pi}{2};- \pi \right)$ nên $\cos \alpha < 0$.

$\cos \alpha =-\sqrt{1-\frac{4}{9}}=- \frac{\sqrt{5}}{3}$.

$\sin\left(\alpha+\frac{2 \pi}{3}\right)=\sin \alpha\cos (\frac{2 \pi}{3})+\cos \alpha \sin (\frac{2 \pi}{3})=$$\frac{2}{3}.(- \frac{1}{2})+(- \frac{\sqrt{5}}{3}).(\frac{\sqrt{3}}{2})=- \frac{\sqrt{15}}{6} - \frac{1}{3}$. 
 }\end{ex}

\begin{ex}
 Tìm tập xác định của hàm số $y=\tan(3x-5\pi)$.\\ 
\choice
{ \True $D=\mathbb{R}\backslash\{ \frac{11}{6}\pi + k \frac{1}{3}\pi\}$ }
   { $D=\mathbb{R}\backslash\{ 1\pi + k \frac{1}{3}\pi\}$ }
     { $D=\mathbb{R}\backslash\{ 2\pi + k \frac{1}{3}\pi\}$ }
    { $D=\mathbb{R}\backslash\{ \frac{11}{3}\pi + k \frac{1}{3}\pi\}$ }
\loigiai{ 
  
 }\end{ex}

\begin{ex}
 Nghiệm của phương trình $\cos\left(3 x + \frac{\pi}{6}\right)=\sin\left(- 2 x + \frac{7 \pi}{6}\right)$ là\ 
\choice
{ $x=- \frac{\pi}{15}+k2 \pi, x=\frac{5 \pi}{6}+k2 \pi (k\in \mathbb{Z})$ }
   { \True $x=- \frac{5 \pi}{6}+k2 \pi, x=\frac{\pi}{10}+k\frac{2 \pi}{5} (k\in \mathbb{Z})$ }
     { $x=- \frac{\pi}{15}+k2 \pi, x=\frac{5 \pi}{6}+k\frac{2 \pi}{5} (k\in \mathbb{Z})$ }
    { $x=- \frac{5 \pi}{6}+k\frac{\pi}{5}, x=\frac{\pi}{10}+k\pi (k\in \mathbb{Z})$ }
\loigiai{ 
 $\cos\left(3 x + \frac{\pi}{6}\right)=\sin\left(- 2 x + \frac{7 \pi}{6}\right) \Leftrightarrow \cos\left(3 x + \frac{\pi}{6}\right)=\cos\left(2 x - \frac{2 \pi}{3}\right)$

$\Leftrightarrow \left[ \begin{array}{l} 
        3 x + \frac{\pi}{6}=2 x - \frac{2 \pi}{3} +k2 \pi \\ 
        3 x + \frac{\pi}{6}=- 2 x + \frac{2 \pi}{3}+k2 \pi
        \end{array} \right.$

$\Leftrightarrow \left[ \begin{array}{l} 
        x=- \frac{5 \pi}{6} +k2 \pi \\ 
        5 x=\frac{\pi}{2}+k2 \pi
        \end{array} \right. $

$\Leftrightarrow \left[ \begin{array}{l} 
        x=- \frac{5 \pi}{6} + k2 \pi \\ 
        x=\frac{\pi}{10}+ k\frac{2 \pi}{5}
        \end{array} \right. , k\in \mathbb{Z} $

 
 }\end{ex}

\Closesolutionfile{ans}
{\bf PHẦN II. Câu trắc nghiệm đúng sai.}
\setcounter{ex}{0}
\Opensolutionfile{ans}[ans/ans009-2]
\begin{ex}
 Cho $\sin \alpha=\frac{1}{2}, \alpha\in \left( 2\pi;\frac{5 \pi}{2} \right)$. Xét tính đúng-sai của các khẳng định sau.
\choiceTFt
{ \True $\cos \alpha=\frac{\sqrt{3}}{2}$ }
   { \True $\sin 2\alpha=\frac{\sqrt{3}}{2}$ }
     { $\cos 2\alpha=- \frac{1}{2}$  }
    { $\sin\left(\alpha- \frac{\pi}{3}\right)=\frac{1}{2} + \frac{\sqrt{3}}{2}$ }
\loigiai{ 
 

 a) Khẳng định đã cho là khẳng định đúng.

 Vì $\alpha \in \left( 2\pi;\frac{5 \pi}{2} \right)$ nên $\cos \alpha > 0$.

$\cos \alpha =\sqrt{1-\frac{1}{4}}=\frac{\sqrt{3}}{2}$.

b) Khẳng định đã cho là khẳng định đúng.

 $\sin 2\alpha=2\sin \alpha \cos \alpha=2.\frac{1}{2}.\frac{\sqrt{3}}{2}=\frac{\sqrt{3}}{2}$.

c) Khẳng định đã cho là khẳng định sai.

 $\cos 2\alpha=1-2\sin^2 \alpha=1-2.\frac{1}{4}=\frac{1}{2}$

d) Khẳng định đã cho là khẳng định sai.

 $\sin\left(\alpha- \frac{\pi}{3}\right)=\sin \alpha\cos (- \frac{\pi}{3})+\cos \alpha \sin (- \frac{\pi}{3})=$$\frac{1}{2}.(\frac{1}{2})+\frac{\sqrt{3}}{2}.(- \frac{\sqrt{3}}{2})=- \frac{1}{2}$.

 
 }\end{ex}

\begin{ex}
 Cho hàm số $y=1 - 6 \cos{\left(2 x \right)}$ . Xét tính đúng-sai của các khẳng định sau. 
\choiceTFt
{ \True  Tập xác định của hàm số là $D=\mathbb{R}$ }
   { Hàm số đã cho là hàm số lẻ }
     { Tập giá trị của hàm số đã cho là $T={[-6;0]}$ }
    { \True  Đồ thị cắt trục tung tại điểm có tung độ bằng ${-5}$ }
\loigiai{ 
 

 a) Khẳng định đã cho là khẳng định đúng.

 Tập xác định của hàm số là $D=\mathbb{R}$.

b) Khẳng định đã cho là khẳng định sai.

 Ta có: Với mọi $x\in \mathbb{R}$ thì $-x\in \mathbb{R}$.

$f(-x)=1 - 6 \cos{\left(2 x \right)}=1 - 6 \cos{\left(2 x \right)}$. Vậy hàm số $y=1 - 6 \cos{\left(2 x \right)}$ là hàm số chẵn.

c) Khẳng định đã cho là khẳng định sai.

 Ta có: $-5 \le 1 - 6 \cos{\left(2 x \right)} \le -5$ nên tập giá trị là ${[-5;-5]}$

d) Khẳng định đã cho là khẳng định đúng.

 Cho $x=0\Rightarrow y=-5$. Suy ra đồ thị cắt trục tung tại điểm có tung độ bằng ${-5}$.

 
 }\end{ex}

\Closesolutionfile{ans}
{\bf PHẦN III. Câu trắc nghiệm trả lời ngắn.}
\setcounter{ex}{0}
\Opensolutionfile{ans}[ans/ans009-3]
\begin{ex}
 Một bánh xe của một loại xe có bán kính ${59}$ cm và quay được 8 vòng trong 3 giây. Tính độ dài quãng đường (theo đơn vị mét) xe đi được trong 2 giây (kết quả làm tròn đến hàng phần mười). 
\shortans[oly]{20,1}

\loigiai{ 
 Một giây bánh xe quay được số vòng là: $\frac{8}{3}$.

Một vòng quay ứng với quãng đường là $2\pi.0,6=1,2\pi$.

Sau ${2}$ giây quãng đường đi được là: $\frac{8}{3}.2.1,2\pi=20,1$:

 
 }\end{ex}

\begin{ex}
 Số nghiệm thuộc khoảng $(- 4 \pi;4 \pi)$ của phương trình $\tan \left(x + \pi\right)=-1$ là
\shortans[oly]{8}

\loigiai{ 
 $\tan \left(x + \pi\right)=-1 \Leftrightarrow x + \pi =- \frac{\pi}{4}+ k\pi \Leftrightarrow x=- \frac{5 \pi}{4}+k\pi, k\in \mathbb{Z}$.

Do $x\in (- 4 \pi;4 \pi)$ nên $- 4 \pi< - \frac{5 \pi}{4}+k\pi < 4 \pi \Rightarrow - \frac{11}{4}< k < \frac{21}{4}$.

Có ${8}$ số k thỏa mãn nên phương trình có ${8}$ nghiệm. 
 }\end{ex}

\Closesolutionfile{ans}

 \begin{center}
-----HẾT-----
\end{center}

 %\newpage 
%\begin{center}
%{\bf BẢNG ĐÁP ÁN MÃ ĐỀ 9 }
%\end{center}
%{\bf Phần 1 }
% \inputansbox{6}{ans009-1}
%{\bf Phần 2 }
% \inputansbox{2}{ans009-2}
%{\bf Phần 3 }
% \inputansbox{6}{ans009-3}
\newpage 




\end{document}