\documentclass[12pt,a4paper]{article}
\usepackage[top=1.5cm, bottom=1.5cm, left=2.0cm, right=1.5cm] {geometry}
\usepackage{amsmath,amssymb,txfonts}
\usepackage{tkz-euclide}
\usepackage{setspace}
\usepackage{lastpage}

\usepackage{tikz,tkz-tab}
%\usepackage[solcolor]{ex_test}
%\usepackage[dethi]{ex_test} % Chỉ hiển thị đề thi
\usepackage[loigiai]{ex_test} % Hiển thị lời giải
%\usepackage[color]{ex_test} % Khoanh các đáp án
\everymath{\displaystyle}

\def\colorEX{\color{purple}}
%\def\colorEX{}%Không tô màu đáp án đúng trong tùy chọn loigiai
\renewtheorem{ex}{\color{violet}Câu}
\renewcommand{\FalseEX}{\stepcounter{dapan}{{\bf \textcolor{blue}{\Alph{dapan}.}}}}
\renewcommand{\TrueEX}{\stepcounter{dapan}{{\bf \textcolor{blue}{\Alph{dapan}.}}}}

%---------- Khai báo viết tắt, in đáp án
\newcommand{\hoac}[1]{ %hệ hoặc
    \left[\begin{aligned}#1\end{aligned}\right.}
\newcommand{\heva}[1]{ %hệ và
    \left\{\begin{aligned}#1\end{aligned}\right.}

%Tiêu đề
\newcommand{\tenso}{iMath}
\newcommand{\tentruong}{Phần mềm Tạo đề ngẫu nhiên}
\newcommand{\tenkythi}{ĐỀ ÔN TẬP}
\newcommand{\tenmonthi}{Môn thi: Toán}
\newcommand{\thoigian}{}
\newcommand{\tieude}[1]{
   \begin{tabular}{cm{3cm}cm{3cm}cm{3cm}}
    {\bf \tenso} & & {\bf \tenkythi} \\
    {\bf \tentruong} & & {\bf \tenmonthi}\\
    && {\bf Thời gian: \bf \thoigian \, phút}\\
    && { \fbox{\bf Mã đề: #1}}
   \end{tabular}\\\\
    
   {Họ tên HS: \dotfill Số báo danh \dotfill}\\
}
\newcommand{\chantrang}[2]{\rfoot{Trang \thepage $-$ Mã đề #2}}
\pagestyle{fancy}
\fancyhf{}
\renewcommand{\headrulewidth}{0pt} 
\renewcommand{\footrulewidth}{0pt}

\begin{document}
%Thiết lập giãn dọng 1.5cm 
%\setlength{\lineskip}{1.5em}
%Nội dung trắc nghiệm bắt đầu ở đây


\tieude{005}
\chantrang{\pageref{LastPage}}{005}
\setcounter{page}{1}
{\bf PHẦN I. Câu trắc nghiệm nhiều phương án lựa chọn.}
\setcounter{ex}{0}
\Opensolutionfile{ans}[ans/ans005-1]
\begin{ex}
 Đổi số đo của góc $810^\circ$ sang radian ta được kết quả bằng\\ 
\choice
{ $\frac{41 \pi}{9}$ }
   { \True $\frac{9 \pi}{2}$ }
     { $\frac{14 \pi}{3}$ }
    { $\frac{79 \pi}{18}$ }
\loigiai{ 
 Áp dụng công thức chuyển đổi: $810^\circ=\dfrac{810.\pi}{180}=\frac{9 \pi}{2}$. 
 }\end{ex}

\begin{ex}
 Tính $\cot\frac{13 \pi}{6}$.\\ 
\choice
{ $ \frac{1}{2} $ }
   { $ \frac{\sqrt{3}}{3} $ }
     { $ \frac{\sqrt{3}}{2} $ }
    { \True $ \sqrt{3} $ }
\loigiai{ 
  
 }\end{ex}

\begin{ex}
 Cho ${b}$ là góc lượng giác. Tìm khẳng định đúng trong các khẳng định sau.\ 
\choice
{ $\cos (\pi+b)=\sin b$ }
   { $\tan (\pi-b)=\cot b$ }
     { $\sin (\pi+b)=\cos b$ }
    { \True $\cot (\pi+b)=\cot b$ }
\loigiai{ 
 $\cot (\pi+b)=\cot b$ là khẳng định đúng. 
 }\end{ex}

\begin{ex}
 Cho ${\beta}$ là góc lượng giác. Tìm khẳng định đúng trong các khẳng định sau.\ 
\choice
{ \True $\sin 2\beta=2\sin \beta\cos \beta$ }
   { $\cos 2\beta=1-2\cos^2 \beta$ }
     { $\sin 2\beta=2\sin \beta$ }
    { $\tan 2\beta=\dfrac{\tan \beta}{1-2\tan^2 \beta}$ }
\loigiai{ 
 $\sin 2\beta=2\sin \beta\cos \beta$ là khẳng định đúng. 
 }\end{ex}

\begin{ex}
 Cho ${\alpha,\beta}$ là các góc lượng giác. Tìm khẳng định đúng trong các khẳng định sau.\ 
\choice
{ $\sin \alpha \cos \beta=\dfrac 1 2[\cos(\alpha+\beta) - \cos(\alpha-\beta)]$ }
   { $\cos \alpha \cos \beta=-\dfrac 1 2[\cos(\alpha+\beta) + \cos(\alpha-\beta)]$ }
     { $\sin \alpha \sin \beta=\dfrac 1 2[\cos(\alpha+\beta) - \cos(\alpha-\beta)]$ }
    { \True $\cos \alpha \cos \beta=\dfrac 1 2[\cos(\alpha+\beta) + \cos(\alpha-\beta)]$ }
\loigiai{ 
 $\cos \alpha \cos \beta=\dfrac 1 2[\cos(\alpha+\beta) + \cos(\alpha-\beta)]$ là khẳng định đúng. 
 }\end{ex}

\begin{ex}
 Cho $\sin \beta=\frac{1}{2}$ với $\beta\in \left( 2\pi;\frac{5 \pi}{2} \right)$. Tính $\sin\left(\beta- \frac{\pi}{6}\right)$.\ 
\choice
{ $\frac{1}{2}$ }
   { $\frac{1}{2} + \frac{\sqrt{3}}{2}$ }
     { \True $0$ }
    { $\frac{\sqrt{3}}{2}$ }
\loigiai{ 
 Vì $\beta \in \left( 2\pi;\frac{5 \pi}{2} \right)$ nên $\cos \beta > 0$.

$\cos \beta =\sqrt{1-\frac{1}{4}}=\frac{\sqrt{3}}{2}$.

$\sin\left(\beta- \frac{\pi}{6}\right)=\sin \beta\cos (- \frac{\pi}{6})+\cos \beta \sin (- \frac{\pi}{6})=$$\frac{1}{2}.(\frac{\sqrt{3}}{2})+\frac{\sqrt{3}}{2}.(- \frac{1}{2})=0$. 
 }\end{ex}

\begin{ex}
 Tìm tập xác định của hàm số $y=\tan(9x-5\pi)$.\\ 
\choice
{ $D=\mathbb{R}\backslash\{ \frac{11}{9}\pi + k \frac{1}{9}\pi\}$ }
   { $D=\mathbb{R}\backslash\{ \frac{1}{3}\pi + k \frac{1}{9}\pi\}$ }
     { $D=\mathbb{R}\backslash\{ \frac{2}{3}\pi + k \frac{1}{9}\pi\}$ }
    { \True $D=\mathbb{R}\backslash\{ \frac{11}{18}\pi + k \frac{1}{9}\pi\}$ }
\loigiai{ 
  
 }\end{ex}

\begin{ex}
 Nghiệm của phương trình $\cos\left(3 x + \frac{\pi}{6}\right)=\sin\left(- x + \frac{5 \pi}{4}\right)$ là\ 
\choice
{ $x=- \frac{5 \pi}{48}+k\pi, x=\frac{11 \pi}{24}+k\frac{\pi}{2} (k\in \mathbb{Z})$ }
   { \True $x=- \frac{11 \pi}{24}+k\pi, x=\frac{7 \pi}{48}+k\frac{\pi}{2} (k\in \mathbb{Z})$ }
     { $x=- \frac{5 \pi}{48}+k2 \pi, x=\frac{11 \pi}{24}+k2 \pi (k\in \mathbb{Z})$ }
    { $x=- \frac{11 \pi}{24}+k\frac{\pi}{4}, x=\frac{7 \pi}{48}+k\frac{\pi}{2} (k\in \mathbb{Z})$ }
\loigiai{ 
 $\cos\left(3 x + \frac{\pi}{6}\right)=\sin\left(- x + \frac{5 \pi}{4}\right) \Leftrightarrow \cos\left(3 x + \frac{\pi}{6}\right)=\cos\left(x - \frac{3 \pi}{4}\right)$

$\Leftrightarrow \left[ \begin{array}{l} 
        3 x + \frac{\pi}{6}=x - \frac{3 \pi}{4} +k2 \pi \\ 
        3 x + \frac{\pi}{6}=- x + \frac{3 \pi}{4}+k2 \pi
        \end{array} \right.$

$\Leftrightarrow \left[ \begin{array}{l} 
        2 x=- \frac{11 \pi}{12} +k2 \pi \\ 
        4 x=\frac{7 \pi}{12}+k2 \pi
        \end{array} \right. $

$\Leftrightarrow \left[ \begin{array}{l} 
        x=- \frac{11 \pi}{24} + k\pi \\ 
        x=\frac{7 \pi}{48}+ k\frac{\pi}{2}
        \end{array} \right. , k\in \mathbb{Z} $

 
 }\end{ex}

\Closesolutionfile{ans}
{\bf PHẦN II. Câu trắc nghiệm đúng sai.}
\setcounter{ex}{0}
\Opensolutionfile{ans}[ans/ans005-2]
\begin{ex}
 Cho $\sin x=\frac{\sqrt{3}}{6}, x\in \left( \frac{5 \pi}{2};3\pi \right)$. Xét tính đúng-sai của các khẳng định sau.
\choiceTFt
{ \True $\cos x=- \frac{\sqrt{33}}{6}$ }
   { \True $\sin 2\alpha=- \frac{\sqrt{11}}{6}$ }
     { \True $\cos 2\alpha=\frac{5}{6}$ }
    { \True $\sin\left(\alpha+\frac{\pi}{6}\right)=\frac{1}{4} - \frac{\sqrt{33}}{12}$ }
\loigiai{ 
 

 a) Khẳng định đã cho là khẳng định đúng.

 Vì $x \in \left( \frac{5 \pi}{2};3\pi \right)$ nên $\cos x < 0$.

$\cos x =-\sqrt{1-\frac{1}{12}}=- \frac{\sqrt{33}}{6}$.

b) Khẳng định đã cho là khẳng định đúng.

 $\sin 2\alpha=2\sin \alpha \cos \alpha=2.\frac{\sqrt{3}}{6}.(- \frac{\sqrt{33}}{6})=- \frac{\sqrt{11}}{6}$.

c) Khẳng định đã cho là khẳng định đúng.

 $\cos 2\alpha=1-2\sin^2 \alpha=1-2.\frac{1}{12}=\frac{5}{6}$

d) Khẳng định đã cho là khẳng định đúng.

 $\sin\left(\alpha+\frac{\pi}{6}\right)=\sin \alpha\cos (\frac{\pi}{6})+\cos \alpha \sin (\frac{\pi}{6})=$$\frac{\sqrt{3}}{6}.(\frac{\sqrt{3}}{2})+(- \frac{\sqrt{33}}{6}).(\frac{1}{2})=\frac{1}{4} - \frac{\sqrt{33}}{12}$.

 
 }\end{ex}

\begin{ex}
 Cho hàm số $y=- 4 \cos{\left(3 x \right)} - 7$ . Xét tính đúng-sai của các khẳng định sau. 
\choiceTFt
{ \True  Tập xác định của hàm số là $D=\mathbb{R}$ }
   { Hàm số đã cho là hàm số lẻ }
     { \True  Tập giá trị của hàm số đã cho là $T={[-11;-11]}$ }
    {  Đồ thị cắt trục tung tại điểm có tung độ bằng ${-10}$ }
\loigiai{ 
 

 a) Khẳng định đã cho là khẳng định đúng.

 Tập xác định của hàm số là $D=\mathbb{R}$.

b) Khẳng định đã cho là khẳng định sai.

 Ta có: Với mọi $x\in \mathbb{R}$ thì $-x\in \mathbb{R}$.

$f(-x)=- 4 \cos{\left(3 x \right)} - 7=- 4 \cos{\left(3 x \right)} - 7$. Vậy hàm số $y=- 4 \cos{\left(3 x \right)} - 7$ là hàm số chẵn.

c) Khẳng định đã cho là khẳng định đúng.

 Ta có: $-11 \le - 4 \cos{\left(3 x \right)} - 7 \le -11$ nên tập giá trị là ${[-11;-11]}$

d) Khẳng định đã cho là khẳng định sai.

 Cho $x=0\Rightarrow y=-11$. Suy ra đồ thị cắt trục tung tại điểm có tung độ bằng ${-11}$.

 
 }\end{ex}

\Closesolutionfile{ans}
{\bf PHẦN III. Câu trắc nghiệm trả lời ngắn.}
\setcounter{ex}{0}
\Opensolutionfile{ans}[ans/ans005-3]
\begin{ex}
 Một bánh xe của một loại xe có bán kính ${47}$ cm và quay được 4 vòng trong 5 giây. Tính độ dài quãng đường (theo đơn vị mét) xe đi được trong 8 giây (kết quả làm tròn đến hàng phần mười). 
\shortans[oly]{20,1}

\loigiai{ 
 Một giây bánh xe quay được số vòng là: $\frac{4}{5}$.

Một vòng quay ứng với quãng đường là $2\pi.0,5=1,0\pi$.

Sau ${8}$ giây quãng đường đi được là: $\frac{4}{5}.8.1,0\pi=20,1$:

 
 }\end{ex}

\begin{ex}
 Số nghiệm thuộc khoảng $(- \pi;\pi)$ của phương trình $\tan \left(4 x + \frac{3 \pi}{4}\right)=0$ là
\shortans[oly]{7}

\loigiai{ 
 $\tan \left(4 x + \frac{3 \pi}{4}\right)=0 \Leftrightarrow 4 x + \frac{3 \pi}{4} =0+ k\pi \Leftrightarrow x=- \frac{3 \pi}{16}+k\frac{\pi}{4}, k\in \mathbb{Z}$.

Do $x\in (- \pi;\pi)$ nên $- \pi< - \frac{3 \pi}{16}+k\frac{\pi}{4} < \pi \Rightarrow - \frac{13}{4}< k < \frac{19}{4}$.

Có ${7}$ số k thỏa mãn nên phương trình có ${7}$ nghiệm. 
 }\end{ex}

\Closesolutionfile{ans}

 \begin{center}
-----HẾT-----
\end{center}

 %\newpage 
%\begin{center}
%{\bf BẢNG ĐÁP ÁN MÃ ĐỀ 5 }
%\end{center}
%{\bf Phần 1 }
% \inputansbox{6}{ans005-1}
%{\bf Phần 2 }
% \inputansbox{2}{ans005-2}
%{\bf Phần 3 }
% \inputansbox{6}{ans005-3}
\newpage 



\end{document}