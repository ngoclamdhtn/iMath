\documentclass[12pt,a4paper]{article}
\usepackage[top=1.5cm, bottom=1.5cm, left=2.0cm, right=1.5cm] {geometry}
\usepackage{amsmath,amssymb,txfonts}
\usepackage{tkz-euclide}
\usepackage{setspace}
\usepackage{lastpage}

\usepackage{tikz,tkz-tab}
%\usepackage[solcolor]{ex_test}
\usepackage[dethi]{ex_test} % Chỉ hiển thị đề thi
%\usepackage[loigiai]{ex_test} % Hiển thị lời giải
%\usepackage[color]{ex_test} % Khoanh các đáp án
\everymath{\displaystyle}

\def\colorEX{\color{purple}}
%\def\colorEX{}%Không tô màu đáp án đúng trong tùy chọn loigiai
\renewtheorem{ex}{\color{violet}Câu}
\renewcommand{\FalseEX}{\stepcounter{dapan}{{\bf \textcolor{blue}{\Alph{dapan}.}}}}
\renewcommand{\TrueEX}{\stepcounter{dapan}{{\bf \textcolor{blue}{\Alph{dapan}.}}}}

%---------- Khai báo viết tắt, in đáp án
\newcommand{\hoac}[1]{ %hệ hoặc
    \left[\begin{aligned}#1\end{aligned}\right.}
\newcommand{\heva}[1]{ %hệ và
    \left\{\begin{aligned}#1\end{aligned}\right.}

%Tiêu đề
\newcommand{\tenso}{iMath}
\newcommand{\tentruong}{Phần mềm Tạo đề ngẫu nhiên}
\newcommand{\tenkythi}{ĐỀ ÔN TẬP}
\newcommand{\tenmonthi}{Môn học: Toán}
\newcommand{\thoigian}{}
\newcommand{\tieude}[1]{
    \noindent
     \begin{minipage}[b]{6cm}
    \centerline{\textbf{\fontsize{11}{0}\selectfont \tenso}}
    \centerline{\fontsize{11}{0}\selectfont \tentruong}  
  \end{minipage}\hspace{1cm}
  \begin{minipage}[b]{11cm}
    \centerline{\textbf{\fontsize{11}{0}\selectfont \tenkythi}}
    \centerline{\textbf{\fontsize{11}{0}\selectfont \tenmonthi}}
    \centerline{\textit{\fontsize{11}{0}\selectfont Thời \underline{gian làm bài: \thoigian  } phút }}
  \end{minipage}
  \vspace*{3mm}
  \noindent
  \begin{minipage}[t]{12cm}
    \textbf{Họ, tên thí sinh:}\dotfill\\
    \textbf{Số báo danh:}\dotfill
  \end{minipage}\hfill
  \begin{minipage}[b]{3cm}
    \setlength\fboxrule{1pt}
    \setlength\fboxsep{3pt}
    \vspace*{3mm}\fbox{\bf Mã đề thi #1}
  \end{minipage}\\
}

\newcommand{\chantrang}[2]{\rfoot{Trang \thepage $-$ Mã đề #2}}
\pagestyle{fancy}
\fancyhf{}
\renewcommand{\headrulewidth}{0pt} 
\renewcommand{\footrulewidth}{0pt}

\begin{document}
%Thiết lập giãn dọng 1.5cm 
%\setlength{\lineskip}{1.5em}


%Nội dung trắc nghiệm bắt đầu ở đây


\tieude{006}
\chantrang{\pageref{LastPage}}{006}
\setcounter{page}{1}
{\bf PHẦN I. Câu trắc nghiệm nhiều phương án lựa chọn.}
\setcounter{ex}{0}
\Opensolutionfile{ans}[ans/ans006-1]
\begin{ex}
 Đổi số đo của góc $330^\circ$ sang radian ta được kết quả bằng\\ 
\choice
{ $\frac{17 \pi}{9}$ }
   { $\frac{31 \pi}{18}$ }
     { \True $\frac{11 \pi}{6}$ }
    { $2 \pi$ }
\loigiai{ 
 Áp dụng công thức chuyển đổi: $330^\circ=\dfrac{330.\pi}{180}=\frac{11 \pi}{6}$. 
 }\end{ex}

\begin{ex}
 Tính $\sin\frac{103 \pi}{3}$.\\ 
\choice
{ $ \frac{\sqrt{3}}{3} $ }
   { $ \sqrt{3} $ }
     { \True $ \frac{\sqrt{3}}{2} $ }
    { $ \frac{1}{2} $ }
\loigiai{ 
  
 }\end{ex}

\begin{ex}
 Cho ${\alpha}$ là góc lượng giác. Tìm khẳng định đúng trong các khẳng định sau.\ 
\choice
{ \True $\sin \left(\frac{\pi}{2}-\alpha\right)=\cos \alpha$ }
   { $\cos (-\alpha)=\sin \alpha$ }
     { $\sin (\pi-\alpha)=-\sin \alpha$ }
    { $\cot \left(\frac{\pi}{2}-\alpha\right)=\cot \alpha$ }
\loigiai{ 
 $\sin \left(\frac{\pi}{2}-\alpha\right)=\cos \alpha$ là khẳng định đúng. 
 }\end{ex}

\begin{ex}
 Cho ${\gamma}$ là góc lượng giác. Tìm khẳng định đúng trong các khẳng định sau.\ 
\choice
{ $\tan 2\gamma=\dfrac{\tan \gamma}{1-2\tan^2 \gamma}$ }
   { $\sin 2\gamma=2\sin \gamma$ }
     { \True $\cos 2\gamma=2\cos^2 \gamma-1$ }
    { $\cos 2\gamma=2\sin^2 \gamma-1$ }
\loigiai{ 
 $\cos 2\gamma=2\cos^2 \gamma-1$ là khẳng định đúng. 
 }\end{ex}

\begin{ex}
 Cho ${\alpha,\beta}$ là các góc lượng giác. Tìm khẳng định đúng trong các khẳng định sau.\ 
\choice
{ $\sin \alpha \cos \beta=\dfrac 1 2[\cos(\alpha+\beta) - \cos(\alpha-\beta)]$ }
   { $\cos \alpha \cos \beta=-\dfrac 1 2[\cos(\alpha+\beta) + \cos(\alpha-\beta)]$ }
     { $\sin \alpha \sin \beta=\dfrac 1 2[\cos(\alpha+\beta) - \cos(\alpha-\beta)]$ }
    { \True $\sin \alpha \sin \beta=\dfrac 1 2[\cos(\alpha-\beta) - \cos(\alpha+\beta)]$ }
\loigiai{ 
 $\sin \alpha \sin \beta=\dfrac 1 2[\cos(\alpha-\beta) - \cos(\alpha+\beta)]$ là khẳng định đúng. 
 }\end{ex}

\begin{ex}
 Cho $\sin \alpha=\frac{9}{10}$ với $\alpha\in \left( 2\pi;\frac{5 \pi}{2} \right)$. Tính $\sin\left(\alpha- \frac{5 \pi}{6}\right)$.\ 
\choice
{ \True $- \frac{9 \sqrt{3}}{20} - \frac{\sqrt{19}}{20}$ }
   { $\frac{\sqrt{19}}{10} + \frac{9}{10}$ }
     { $- \frac{9}{20} - \frac{\sqrt{57}}{20}$ }
    { $- \frac{9 \sqrt{3}}{20} + \frac{\sqrt{19}}{20}$ }
\loigiai{ 
 Vì $\alpha \in \left( 2\pi;\frac{5 \pi}{2} \right)$ nên $\cos \alpha > 0$.

$\cos \alpha =\sqrt{1-\frac{81}{100}}=\frac{\sqrt{19}}{10}$.

$\sin\left(\alpha- \frac{5 \pi}{6}\right)=\sin \alpha\cos (- \frac{5 \pi}{6})+\cos \alpha \sin (- \frac{5 \pi}{6})=$$\frac{9}{10}.(- \frac{\sqrt{3}}{2})+\frac{\sqrt{19}}{10}.(- \frac{1}{2})=- \frac{9 \sqrt{3}}{20} - \frac{\sqrt{19}}{20}$. 
 }\end{ex}

\begin{ex}
 Tìm tập xác định của hàm số $y=\tan(8x-5\pi)$.\\ 
\choice
{ \True $D=\mathbb{R}\backslash\{ \frac{11}{16}\pi + k \frac{1}{8}\pi\}$ }
   { $D=\mathbb{R}\backslash\{ \frac{11}{8}\pi + k \frac{1}{8}\pi\}$ }
     { $D=\mathbb{R}\backslash\{ \frac{3}{8}\pi + k \frac{1}{8}\pi\}$ }
    { $D=\mathbb{R}\backslash\{ \frac{3}{4}\pi + k \frac{1}{8}\pi\}$ }
\loigiai{ 
  
 }\end{ex}

\begin{ex}
 Nghiệm của phương trình $\cos\left(4 x + \frac{\pi}{3}\right)=\sin\left(- 2 x - \frac{\pi}{4}\right)$ là\ 
\choice
{ $x=\frac{17 \pi}{72}+k2 \pi, x=- \frac{5 \pi}{24}+k2 \pi (k\in \mathbb{Z})$ }
   { \True $x=\frac{5 \pi}{24}+k\pi, x=- \frac{13 \pi}{72}+k\frac{\pi}{3} (k\in \mathbb{Z})$ }
     { $x=\frac{17 \pi}{72}+k\pi, x=- \frac{5 \pi}{24}+k\frac{\pi}{3} (k\in \mathbb{Z})$ }
    { $x=\frac{5 \pi}{24}+k\frac{\pi}{6}, x=- \frac{13 \pi}{72}+k\frac{\pi}{2} (k\in \mathbb{Z})$ }
\loigiai{ 
 $\cos\left(4 x + \frac{\pi}{3}\right)=\sin\left(- 2 x - \frac{\pi}{4}\right) \Leftrightarrow \cos\left(4 x + \frac{\pi}{3}\right)=\cos\left(2 x + \frac{3 \pi}{4}\right)$

$\Leftrightarrow \left[ \begin{array}{l} 
        4 x + \frac{\pi}{3}=2 x + \frac{3 \pi}{4} +k2 \pi \\ 
        4 x + \frac{\pi}{3}=- 2 x - \frac{3 \pi}{4}+k2 \pi
        \end{array} \right.$

$\Leftrightarrow \left[ \begin{array}{l} 
        2 x=\frac{5 \pi}{12} +k2 \pi \\ 
        6 x=- \frac{13 \pi}{12}+k2 \pi
        \end{array} \right. $

$\Leftrightarrow \left[ \begin{array}{l} 
        x=\frac{5 \pi}{24} + k\pi \\ 
        x=- \frac{13 \pi}{72}+ k\frac{\pi}{3}
        \end{array} \right. , k\in \mathbb{Z} $

 
 }\end{ex}

\Closesolutionfile{ans}
{\bf PHẦN II. Câu trắc nghiệm đúng sai.}
\setcounter{ex}{0}
\Opensolutionfile{ans}[ans/ans006-2]
\begin{ex}
 Cho $\sin x=\frac{\sqrt{7}}{9}, x\in \left( \frac{\pi}{2};\pi \right)$. Xét tính đúng-sai của các khẳng định sau.
\choiceTFt
{ $\cos x=\frac{\sqrt{74}}{9}$ }
   { $\sin 2\gamma=- \frac{\sqrt{518}}{81}$  }
     { $\cos 2\gamma=- \frac{67}{81}$  }
    { \True $\sin\left(\gamma+\frac{3 \pi}{4}\right)=- \frac{\sqrt{37}}{9} - \frac{\sqrt{14}}{18}$ }
\loigiai{ 
 

 a) Khẳng định đã cho là khẳng định sai.

 Vì $x \in \left( \frac{\pi}{2};\pi \right)$ nên $\cos x < 0$.

$\cos x =-\sqrt{1-\frac{7}{81}}=- \frac{\sqrt{74}}{9}$.

b) Khẳng định đã cho là khẳng định sai.

 $\sin 2\gamma=2\sin \gamma \cos \gamma=2.\frac{\sqrt{7}}{9}.(- \frac{\sqrt{74}}{9})=- \frac{2 \sqrt{518}}{81}$.

c) Khẳng định đã cho là khẳng định sai.

 $\cos 2\gamma=1-2\sin^2 \gamma=1-2.\frac{7}{81}=\frac{67}{81}$

d) Khẳng định đã cho là khẳng định đúng.

 $\sin\left(\gamma+\frac{3 \pi}{4}\right)=\sin \gamma\cos (\frac{3 \pi}{4})+\cos \gamma \sin (\frac{3 \pi}{4})=$$\frac{\sqrt{7}}{9}.(- \frac{\sqrt{2}}{2})+(- \frac{\sqrt{74}}{9}).(\frac{\sqrt{2}}{2})=- \frac{\sqrt{37}}{9} - \frac{\sqrt{14}}{18}$.

 
 }\end{ex}

\begin{ex}
 Cho hàm số $y=6 \cos{\left(8 x \right)} - 2$ . Xét tính đúng-sai của các khẳng định sau. 
\choiceTFt
{ \True  Tập xác định của hàm số là $D=\mathbb{R}$ }
   { Hàm số đã cho là hàm số lẻ }
     { \True  Tập giá trị của hàm số đã cho là $T={[-8;-8]}$ }
    { \True  Đồ thị cắt trục tung tại điểm có tung độ bằng ${4}$ }
\loigiai{ 
 

 a) Khẳng định đã cho là khẳng định đúng.

 Tập xác định của hàm số là $D=\mathbb{R}$.

b) Khẳng định đã cho là khẳng định sai.

 Ta có: Với mọi $x\in \mathbb{R}$ thì $-x\in \mathbb{R}$.

$f(-x)=6 \cos{\left(8 x \right)} - 2=6 \cos{\left(8 x \right)} - 2$. Vậy hàm số $y=6 \cos{\left(8 x \right)} - 2$ là hàm số chẵn.

c) Khẳng định đã cho là khẳng định đúng.

 Ta có: $-8 \le 6 \cos{\left(8 x \right)} - 2 \le -8$ nên tập giá trị là ${[-8;-8]}$

d) Khẳng định đã cho là khẳng định đúng.

 Cho $x=0\Rightarrow y=4$. Suy ra đồ thị cắt trục tung tại điểm có tung độ bằng ${4}$.

 
 }\end{ex}

\Closesolutionfile{ans}
{\bf PHẦN III. Câu trắc nghiệm trả lời ngắn.}
\setcounter{ex}{0}
\Opensolutionfile{ans}[ans/ans006-3]
\begin{ex}
 Một bánh xe của một loại xe có bán kính ${56}$ cm và quay được 7 vòng trong 3 giây. Tính độ dài quãng đường (theo đơn vị mét) xe đi được trong 4 giây (kết quả làm tròn đến hàng phần mười). 
\shortans[oly]{35,2}

\loigiai{ 
 Một giây bánh xe quay được số vòng là: $\frac{7}{3}$.

Một vòng quay ứng với quãng đường là $2\pi.0,6=1,2\pi$.

Sau ${4}$ giây quãng đường đi được là: $\frac{7}{3}.4.1,2\pi=35,2$:

 
 }\end{ex}

\begin{ex}
 Số nghiệm thuộc đoạn $[- 5 \pi;5 \pi]$ của phương trình $\tan \left(2 x + \frac{\pi}{2}\right)=\sqrt{3}$ là
\shortans[oly]{20}

\loigiai{ 
 $\tan \left(2 x + \frac{\pi}{2}\right)=\sqrt{3} \Leftrightarrow 2 x + \frac{\pi}{2} =\frac{\pi}{3}+ k\pi \Leftrightarrow x=- \frac{\pi}{12}+k\frac{\pi}{2}, k\in \mathbb{Z}$.

Do $x\in [- 5 \pi;5 \pi]$ nên $- 5 \pi\le - \frac{\pi}{12}+k\frac{\pi}{2} \le 5 \pi \Rightarrow - \frac{59}{6}\le k \le \frac{61}{6}$.

Có ${20}$ số k thỏa mãn nên phương trình có ${20}$ nghiệm. 
 }\end{ex}

\Closesolutionfile{ans}

 \begin{center}
-----HẾT-----
\end{center}

 %\newpage 
%\begin{center}
%{\bf BẢNG ĐÁP ÁN MÃ ĐỀ 6 }
%\end{center}
%{\bf Phần 1 }
% \inputansbox{6}{ans006-1}
%{\bf Phần 2 }
% \inputansbox{2}{ans006-2}
%{\bf Phần 3 }
% \inputansbox{6}{ans006-3}
\newpage 




\end{document}