\documentclass[12pt,a4paper]{article}
\usepackage[top=1.5cm, bottom=1.5cm, left=2.0cm, right=1.5cm] {geometry}
\usepackage{amsmath,amssymb,txfonts}
\usepackage{tkz-euclide}
\usepackage{setspace}
\usepackage{lastpage}

\usepackage{tikz,tkz-tab}
%\usepackage[solcolor]{ex_test}
%\usepackage[dethi]{ex_test} % Chỉ hiển thị đề thi
\usepackage[loigiai]{ex_test} % Hiển thị lời giải
%\usepackage[color]{ex_test} % Khoanh các đáp án
\everymath{\displaystyle}

\def\colorEX{\color{purple}}
%\def\colorEX{}%Không tô màu đáp án đúng trong tùy chọn loigiai
\renewtheorem{ex}{\color{violet}Câu}
\renewcommand{\FalseEX}{\stepcounter{dapan}{{\bf \textcolor{blue}{\Alph{dapan}.}}}}
\renewcommand{\TrueEX}{\stepcounter{dapan}{{\bf \textcolor{blue}{\Alph{dapan}.}}}}

%---------- Khai báo viết tắt, in đáp án
\newcommand{\hoac}[1]{ %hệ hoặc
    \left[\begin{aligned}#1\end{aligned}\right.}
\newcommand{\heva}[1]{ %hệ và
    \left\{\begin{aligned}#1\end{aligned}\right.}

%Tiêu đề
\newcommand{\tenso}{iMath}
\newcommand{\tentruong}{Phần mềm Tạo đề ngẫu nhiên}
\newcommand{\tenkythi}{ĐỀ ÔN TẬP}
\newcommand{\tenmonthi}{Môn thi: Toán}
\newcommand{\thoigian}{}
\newcommand{\tieude}[1]{
   \begin{tabular}{cm{3cm}cm{3cm}cm{3cm}}
    {\bf \tenso} & & {\bf \tenkythi} \\
    {\bf \tentruong} & & {\bf \tenmonthi}\\
    && {\bf Thời gian: \bf \thoigian \, phút}\\
    && { \fbox{\bf Mã đề: #1}}
   \end{tabular}\\\\
    
   {Họ tên HS: \dotfill Số báo danh \dotfill}\\
}
\newcommand{\chantrang}[2]{\rfoot{Trang \thepage $-$ Mã đề #2}}
\pagestyle{fancy}
\fancyhf{}
\renewcommand{\headrulewidth}{0pt} 
\renewcommand{\footrulewidth}{0pt}

\begin{document}
%Thiết lập giãn dọng 1.5cm 
%\setlength{\lineskip}{1.5em}
%Nội dung trắc nghiệm bắt đầu ở đây


\tieude{004}
\chantrang{\pageref{LastPage}}{004}
\setcounter{page}{1}
{\bf PHẦN I. Câu trắc nghiệm nhiều phương án lựa chọn.}
\setcounter{ex}{0}
\Opensolutionfile{ans}[ans/ans004-1]
\begin{ex}
 Đổi số đo của góc $-465^\circ$ sang radian ta được kết quả bằng\\ 
\choice
{ $- \frac{91 \pi}{36}$ }
   { \True $- \frac{31 \pi}{12}$ }
     { $- \frac{97 \pi}{36}$ }
    { $- \frac{29 \pi}{12}$ }
\loigiai{ 
 Áp dụng công thức chuyển đổi: $-465^\circ=\dfrac{-465.\pi}{180}=- \frac{31 \pi}{12}$. 
 }\end{ex}

\begin{ex}
 Tính $\sin\frac{103 \pi}{3}$.\\ 
\choice
{ \True $ \frac{\sqrt{3}}{2} $ }
   { $ \frac{1}{2} $ }
     { $ \frac{\sqrt{3}}{3} $ }
    { $ \sqrt{3} $ }
\loigiai{ 
  
 }\end{ex}

\begin{ex}
 Cho ${b}$ là góc lượng giác. Tìm khẳng định đúng trong các khẳng định sau.\ 
\choice
{ \True $\tan (-b)=-\tan b$ }
   { $\cos (-b)=-\cos b$ }
     { $\tan (-b)=\cot b$ }
    { $\sin (\pi-b)=-\sin b$ }
\loigiai{ 
 $\tan (-b)=-\tan b$ là khẳng định đúng. 
 }\end{ex}

\begin{ex}
 Cho ${\beta}$ là góc lượng giác. Tìm khẳng định đúng trong các khẳng định sau.\ 
\choice
{ \True $\cos 2\beta=\cos^2 \beta-\sin^2 \beta$ }
   { $\cos 2\beta=1-2\cos^2 \beta$ }
     { $\tan 2\beta=\dfrac{\tan \beta}{1-\tan^2 \beta}$ }
    { $\sin 2\beta=\sin \beta+\cos \beta$ }
\loigiai{ 
 $\cos 2\beta=\cos^2 \beta-\sin^2 \beta$ là khẳng định đúng. 
 }\end{ex}

\begin{ex}
 Cho ${\alpha,\beta}$ là các góc lượng giác. Tìm khẳng định đúng trong các khẳng định sau.\ 
\choice
{ \True $\sin \alpha \cos \beta=\dfrac 1 2[\sin(\alpha+\beta) + \sin(\alpha-\beta)]$ }
   { $\sin \alpha \sin \beta=-\dfrac 1 2[\cos(\alpha-\beta) - \cos(\alpha+\beta)]$ }
     { $\sin \alpha \cos \beta=\dfrac 1 2[\sin(\alpha+\beta) - \sin(\alpha-\beta)]$ }
    { $\cos \alpha \cos \beta=-\dfrac 1 2[\cos(\alpha+\beta) + \cos(\alpha-\beta)]$ }
\loigiai{ 
 $\sin \alpha \cos \beta=\dfrac 1 2[\sin(\alpha+\beta) + \sin(\alpha-\beta)]$ là khẳng định đúng. 
 }\end{ex}

\begin{ex}
 Cho $\sin x=\frac{5}{7}$ với $x\in \left( 0;\frac{\pi}{2} \right)$. Tính $\sin\left(x- \frac{\pi}{6}\right)$.\ 
\choice
{ $- \frac{5}{14} + \frac{3 \sqrt{2}}{7}$ }
   { $\frac{\sqrt{6}}{7} + \frac{5 \sqrt{3}}{14}$ }
     { \True $- \frac{\sqrt{6}}{7} + \frac{5 \sqrt{3}}{14}$ }
    { $\frac{2 \sqrt{6}}{7} + \frac{5}{7}$ }
\loigiai{ 
 Vì $x \in \left( 0;\frac{\pi}{2} \right)$ nên $\cos x > 0$.

$\cos x =\sqrt{1-\frac{25}{49}}=\frac{2 \sqrt{6}}{7}$.

$\sin\left(x- \frac{\pi}{6}\right)=\sin x\cos (- \frac{\pi}{6})+\cos x \sin (- \frac{\pi}{6})=$$\frac{5}{7}.(\frac{\sqrt{3}}{2})+\frac{2 \sqrt{6}}{7}.(- \frac{1}{2})=- \frac{\sqrt{6}}{7} + \frac{5 \sqrt{3}}{14}$. 
 }\end{ex}

\begin{ex}
 Tìm tập xác định của hàm số $y=\tan(3x-5\pi)$.\\ 
\choice
{ $D=\mathbb{R}\backslash\{ 1\pi + k \frac{1}{3}\pi\}$ }
   { \True $D=\mathbb{R}\backslash\{ \frac{11}{6}\pi + k \frac{1}{3}\pi\}$ }
     { $D=\mathbb{R}\backslash\{ 2\pi + k \frac{1}{3}\pi\}$ }
    { $D=\mathbb{R}\backslash\{ \frac{11}{3}\pi + k \frac{1}{3}\pi\}$ }
\loigiai{ 
  
 }\end{ex}

\begin{ex}
 Nghiệm của phương trình $\cos\left(2 x + \frac{\pi}{4}\right)=\sin\left(- x - \frac{\pi}{6}\right)$ là\ 
\choice
{ $x=\frac{5 \pi}{12}+k\frac{\pi}{3}, x=- \frac{11 \pi}{36}+k\pi (k\in \mathbb{Z})$ }
   { $x=\frac{7 \pi}{18}+k2 \pi, x=- \frac{5 \pi}{12}+k\frac{2 \pi}{3} (k\in \mathbb{Z})$ }
     { $x=\frac{7 \pi}{18}+k2 \pi, x=- \frac{5 \pi}{12}+k2 \pi (k\in \mathbb{Z})$ }
    { \True $x=\frac{5 \pi}{12}+k2 \pi, x=- \frac{11 \pi}{36}+k\frac{2 \pi}{3} (k\in \mathbb{Z})$ }
\loigiai{ 
 $\cos\left(2 x + \frac{\pi}{4}\right)=\sin\left(- x - \frac{\pi}{6}\right) \Leftrightarrow \cos\left(2 x + \frac{\pi}{4}\right)=\cos\left(x + \frac{2 \pi}{3}\right)$

$\Leftrightarrow \left[ \begin{array}{l} 
        2 x + \frac{\pi}{4}=x + \frac{2 \pi}{3} +k2 \pi \\ 
        2 x + \frac{\pi}{4}=- x - \frac{2 \pi}{3}+k2 \pi
        \end{array} \right.$

$\Leftrightarrow \left[ \begin{array}{l} 
        x=\frac{5 \pi}{12} +k2 \pi \\ 
        3 x=- \frac{11 \pi}{12}+k2 \pi
        \end{array} \right. $

$\Leftrightarrow \left[ \begin{array}{l} 
        x=\frac{5 \pi}{12} + k2 \pi \\ 
        x=- \frac{11 \pi}{36}+ k\frac{2 \pi}{3}
        \end{array} \right. , k\in \mathbb{Z} $

 
 }\end{ex}

\Closesolutionfile{ans}
{\bf PHẦN II. Câu trắc nghiệm đúng sai.}
\setcounter{ex}{0}
\Opensolutionfile{ans}[ans/ans004-2]
\begin{ex}
 Cho $\sin x=\frac{1}{4}, x\in \left( - \frac{3 \pi}{2};- \pi \right)$. Xét tính đúng-sai của các khẳng định sau.
\choiceTFt
{ $\cos x=\frac{\sqrt{15}}{4}$ }
   { $\sin 2\gamma=- \frac{\sqrt{15}}{16}$  }
     { $\cos 2\gamma=- \frac{7}{8}$  }
    { $\sin\left(\gamma- \frac{2 \pi}{3}\right)=\frac{1}{4} - \frac{\sqrt{15}}{4}$ }
\loigiai{ 
 

 a) Khẳng định đã cho là khẳng định sai.

 Vì $x \in \left( - \frac{3 \pi}{2};- \pi \right)$ nên $\cos x < 0$.

$\cos x =-\sqrt{1-\frac{1}{16}}=- \frac{\sqrt{15}}{4}$.

b) Khẳng định đã cho là khẳng định sai.

 $\sin 2\gamma=2\sin \gamma \cos \gamma=2.\frac{1}{4}.(- \frac{\sqrt{15}}{4})=- \frac{\sqrt{15}}{8}$.

c) Khẳng định đã cho là khẳng định sai.

 $\cos 2\gamma=1-2\sin^2 \gamma=1-2.\frac{1}{16}=\frac{7}{8}$

d) Khẳng định đã cho là khẳng định sai.

 $\sin\left(\gamma- \frac{2 \pi}{3}\right)=\sin \gamma\cos (- \frac{2 \pi}{3})+\cos \gamma \sin (- \frac{2 \pi}{3})=$$\frac{1}{4}.(- \frac{1}{2})+(- \frac{\sqrt{15}}{4}).(- \frac{\sqrt{3}}{2})=- \frac{1}{8} + \frac{3 \sqrt{5}}{8}$.

 
 }\end{ex}

\begin{ex}
 Cho hàm số $y=\cos{\left(7 x \right)} + 6$ . Xét tính đúng-sai của các khẳng định sau. 
\choiceTFt
{ Tập xác định của hàm số là $D=[-1;1]$ }
   { \True  Hàm số đã cho là hàm số chẵn }
     { \True  Tập giá trị của hàm số đã cho là $T={[5;5]}$ }
    {  Đồ thị cắt trục tung tại điểm có tung độ bằng ${10}$ }
\loigiai{ 
 

 a) Khẳng định đã cho là khẳng định sai.

 Tập xác định của hàm số là $D=\mathbb{R}$.

b) Khẳng định đã cho là khẳng định đúng.

 Ta có: Với mọi $x\in \mathbb{R}$ thì $-x\in \mathbb{R}$.

$f(-x)=\cos{\left(7 x \right)} + 6=\cos{\left(7 x \right)} + 6$. Vậy hàm số $y=\cos{\left(7 x \right)} + 6$ là hàm số chẵn.

c) Khẳng định đã cho là khẳng định đúng.

 Ta có: $5 \le \cos{\left(7 x \right)} + 6 \le 5$ nên tập giá trị là ${[5;5]}$

d) Khẳng định đã cho là khẳng định sai.

 Cho $x=0\Rightarrow y=7$. Suy ra đồ thị cắt trục tung tại điểm có tung độ bằng ${7}$.

 
 }\end{ex}

\Closesolutionfile{ans}
{\bf PHẦN III. Câu trắc nghiệm trả lời ngắn.}
\setcounter{ex}{0}
\Opensolutionfile{ans}[ans/ans004-3]
\begin{ex}
 Một bánh xe của một loại xe có bán kính ${47}$ cm và quay được 11 vòng trong 3 giây. Tính độ dài quãng đường (theo đơn vị mét) xe đi được trong 7 giây (kết quả làm tròn đến hàng phần mười). 
\shortans[oly]{80,6}

\loigiai{ 
 Một giây bánh xe quay được số vòng là: $\frac{11}{3}$.

Một vòng quay ứng với quãng đường là $2\pi.0,5=1,0\pi$.

Sau ${7}$ giây quãng đường đi được là: $\frac{11}{3}.7.1,0\pi=80,6$:

 
 }\end{ex}

\begin{ex}
 Số nghiệm thuộc khoảng $(- 4 \pi;4 \pi)$ của phương trình $\tan \left(x - \frac{\pi}{6}\right)=\sqrt{3}$ là
\shortans[oly]{8}

\loigiai{ 
 $\tan \left(x - \frac{\pi}{6}\right)=\sqrt{3} \Leftrightarrow x - \frac{\pi}{6} =\frac{\pi}{3}+ k\pi \Leftrightarrow x=\frac{\pi}{2}+k\pi, k\in \mathbb{Z}$.

Do $x\in (- 4 \pi;4 \pi)$ nên $- 4 \pi< \frac{\pi}{2}+k\pi < 4 \pi \Rightarrow - \frac{9}{2}< k < \frac{7}{2}$.

Có ${8}$ số k thỏa mãn nên phương trình có ${8}$ nghiệm. 
 }\end{ex}

\Closesolutionfile{ans}

 \begin{center}
-----HẾT-----
\end{center}

 %\newpage 
%\begin{center}
%{\bf BẢNG ĐÁP ÁN MÃ ĐỀ 4 }
%\end{center}
%{\bf Phần 1 }
% \inputansbox{6}{ans004-1}
%{\bf Phần 2 }
% \inputansbox{2}{ans004-2}
%{\bf Phần 3 }
% \inputansbox{6}{ans004-3}
\newpage 



\end{document}