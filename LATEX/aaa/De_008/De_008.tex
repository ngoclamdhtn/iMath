\documentclass[12pt,a4paper]{article}
\usepackage[top=1.5cm, bottom=1.5cm, left=2.0cm, right=1.5cm] {geometry}
\usepackage{amsmath,amssymb,txfonts}
\usepackage{tkz-euclide}
\usepackage{setspace}
\usepackage{lastpage}

\usepackage{tikz,tkz-tab}
%\usepackage[solcolor]{ex_test}
\usepackage[dethi]{ex_test} % Chỉ hiển thị đề thi
%\usepackage[loigiai]{ex_test} % Hiển thị lời giải
%\usepackage[color]{ex_test} % Khoanh các đáp án
\everymath{\displaystyle}

\def\colorEX{\color{purple}}
%\def\colorEX{}%Không tô màu đáp án đúng trong tùy chọn loigiai
\renewtheorem{ex}{\color{violet}Câu}
\renewcommand{\FalseEX}{\stepcounter{dapan}{{\bf \textcolor{blue}{\Alph{dapan}.}}}}
\renewcommand{\TrueEX}{\stepcounter{dapan}{{\bf \textcolor{blue}{\Alph{dapan}.}}}}

%---------- Khai báo viết tắt, in đáp án
\newcommand{\hoac}[1]{ %hệ hoặc
    \left[\begin{aligned}#1\end{aligned}\right.}
\newcommand{\heva}[1]{ %hệ và
    \left\{\begin{aligned}#1\end{aligned}\right.}

%Tiêu đề
\newcommand{\tenso}{iMath}
\newcommand{\tentruong}{Phần mềm Tạo đề ngẫu nhiên}
\newcommand{\tenkythi}{ĐỀ ÔN TẬP}
\newcommand{\tenmonthi}{Môn học: Toán}
\newcommand{\thoigian}{}
\newcommand{\tieude}[1]{
    \noindent
     \begin{minipage}[b]{6cm}
    \centerline{\textbf{\fontsize{11}{0}\selectfont \tenso}}
    \centerline{\fontsize{11}{0}\selectfont \tentruong}  
  \end{minipage}\hspace{1cm}
  \begin{minipage}[b]{11cm}
    \centerline{\textbf{\fontsize{11}{0}\selectfont \tenkythi}}
    \centerline{\textbf{\fontsize{11}{0}\selectfont \tenmonthi}}
    \centerline{\textit{\fontsize{11}{0}\selectfont Thời \underline{gian làm bài: \thoigian  } phút }}
  \end{minipage}
  \vspace*{3mm}
  \noindent
  \begin{minipage}[t]{12cm}
    \textbf{Họ, tên thí sinh:}\dotfill\\
    \textbf{Số báo danh:}\dotfill
  \end{minipage}\hfill
  \begin{minipage}[b]{3cm}
    \setlength\fboxrule{1pt}
    \setlength\fboxsep{3pt}
    \vspace*{3mm}\fbox{\bf Mã đề thi #1}
  \end{minipage}\\
}

\newcommand{\chantrang}[2]{\rfoot{Trang \thepage $-$ Mã đề #2}}
\pagestyle{fancy}
\fancyhf{}
\renewcommand{\headrulewidth}{0pt} 
\renewcommand{\footrulewidth}{0pt}

\begin{document}
%Thiết lập giãn dọng 1.5cm 
%\setlength{\lineskip}{1.5em}


%Nội dung trắc nghiệm bắt đầu ở đây


\tieude{008}
\chantrang{\pageref{LastPage}}{008}
\setcounter{page}{1}
{\bf PHẦN I. Câu trắc nghiệm nhiều phương án lựa chọn.}
\setcounter{ex}{0}
\Opensolutionfile{ans}[ans/ans008-1]
\begin{ex}
 Đổi số đo của góc $-160^\circ$ sang radian ta được kết quả bằng\\ 
\choice
{ $- \frac{13 \pi}{18}$ }
   { $- \pi$ }
     { \True $- \frac{8 \pi}{9}$ }
    { $- \frac{5 \pi}{6}$ }
\loigiai{ 
 Áp dụng công thức chuyển đổi: $-160^\circ=\dfrac{-160.\pi}{180}=- \frac{8 \pi}{9}$. 
 }\end{ex}

\begin{ex}
 Tính $\cos\frac{2 \pi}{3}$.\\ 
\choice
{ \True $ - \frac{1}{2} $ }
   { $ - \frac{\sqrt{3}}{3} $ }
     { $ \frac{\sqrt{3}}{2} $ }
    { $ - \sqrt{3} $ }
\loigiai{ 
  
 }\end{ex}

\begin{ex}
 Cho ${\gamma}$ là góc lượng giác. Tìm khẳng định đúng trong các khẳng định sau.\ 
\choice
{ $\tan (-\gamma)=\cot \gamma$ }
   { \True $\cos (\pi+\gamma)=-\cos \gamma$ }
     { $\cos (-\gamma)=\sin \gamma$ }
    { $\sin (-\gamma)=\sin \gamma$ }
\loigiai{ 
 $\cos (\pi+\gamma)=-\cos \gamma$ là khẳng định đúng. 
 }\end{ex}

\begin{ex}
 Cho ${x}$ là góc lượng giác. Tìm khẳng định đúng trong các khẳng định sau.\ 
\choice
{ $\cos 2x=1-2\cos^2 x$ }
   { $\tan 2x=\dfrac{2\tan x}{1+\tan^2 x}$ }
     { $\sin 2x=\sin x\cos x$ }
    { \True $\sin 2x=2\sin x\cos x$ }
\loigiai{ 
 $\sin 2x=2\sin x\cos x$ là khẳng định đúng. 
 }\end{ex}

\begin{ex}
 Cho ${\alpha,\beta}$ là các góc lượng giác. Tìm khẳng định đúng trong các khẳng định sau.\ 
\choice
{ $\sin \alpha \cos \beta=\dfrac 1 2[\cos(\alpha+\beta) - \cos(\alpha-\beta)]$ }
   { \True $\cos \alpha \cos \beta=\dfrac 1 2[\cos(\alpha+\beta) + \cos(\alpha-\beta)]$ }
     { $\sin \alpha \sin \beta=\dfrac 1 2[\cos(\alpha+\beta) - \cos(\alpha-\beta)]$ }
    { $\cos \alpha \cos \beta=\dfrac 1 2[\cos(\alpha+\beta) - \cos(\alpha-\beta)]$ }
\loigiai{ 
 $\cos \alpha \cos \beta=\dfrac 1 2[\cos(\alpha+\beta) + \cos(\alpha-\beta)]$ là khẳng định đúng. 
 }\end{ex}

\begin{ex}
 Cho $\sin \gamma=\frac{7}{10}$ với $\gamma\in \left( \frac{\pi}{2};\pi \right)$. Tính $\sin\left(\gamma- \frac{\pi}{4}\right)$.\ 
\choice
{ \True $\frac{7 \sqrt{2}}{20} + \frac{\sqrt{102}}{20}$ }
   { $- \frac{\sqrt{102}}{20} - \frac{7 \sqrt{2}}{20}$ }
     { $- \frac{\sqrt{102}}{20} + \frac{7 \sqrt{2}}{20}$ }
    { $\frac{7}{10} - \frac{\sqrt{51}}{10}$ }
\loigiai{ 
 Vì $\gamma \in \left( \frac{\pi}{2};\pi \right)$ nên $\cos \gamma < 0$.

$\cos \gamma =-\sqrt{1-\frac{49}{100}}=- \frac{\sqrt{51}}{10}$.

$\sin\left(\gamma- \frac{\pi}{4}\right)=\sin \gamma\cos (- \frac{\pi}{4})+\cos \gamma \sin (- \frac{\pi}{4})=$$\frac{7}{10}.(\frac{\sqrt{2}}{2})+(- \frac{\sqrt{51}}{10}).(- \frac{\sqrt{2}}{2})=\frac{7 \sqrt{2}}{20} + \frac{\sqrt{102}}{20}$. 
 }\end{ex}

\begin{ex}
 Tìm tập xác định của hàm số $y=\tan(3x+5\pi)$.\\ 
\choice
{ \True $D=\mathbb{R}\backslash\{ - \frac{3}{2}\pi + k \frac{1}{3}\pi\}$ }
   { $D=\mathbb{R}\backslash\{ -3\pi + k \frac{1}{3}\pi\}$ }
     { $D=\mathbb{R}\backslash\{ - \frac{2}{3}\pi + k \frac{1}{3}\pi\}$ }
    { $D=\mathbb{R}\backslash\{ - \frac{4}{3}\pi + k \frac{1}{3}\pi\}$ }
\loigiai{ 
  
 }\end{ex}

\begin{ex}
 Nghiệm của phương trình $\cos\left(4 x - \frac{\pi}{3}\right)=\sin\left(- x + \frac{5 \pi}{4}\right)$ là\ 
\choice
{ \True $x=- \frac{5 \pi}{36}+k\frac{2 \pi}{3}, x=\frac{13 \pi}{60}+k\frac{2 \pi}{5} (k\in \mathbb{Z})$ }
   { $x=- \frac{17 \pi}{60}+k2 \pi, x=\frac{5 \pi}{36}+k2 \pi (k\in \mathbb{Z})$ }
     { $x=- \frac{17 \pi}{60}+k\frac{2 \pi}{3}, x=\frac{5 \pi}{36}+k\frac{2 \pi}{5} (k\in \mathbb{Z})$ }
    { $x=- \frac{5 \pi}{36}+k\frac{\pi}{5}, x=\frac{13 \pi}{60}+k\frac{\pi}{3} (k\in \mathbb{Z})$ }
\loigiai{ 
 $\cos\left(4 x - \frac{\pi}{3}\right)=\sin\left(- x + \frac{5 \pi}{4}\right) \Leftrightarrow \cos\left(4 x - \frac{\pi}{3}\right)=\cos\left(x - \frac{3 \pi}{4}\right)$

$\Leftrightarrow \left[ \begin{array}{l} 
        4 x - \frac{\pi}{3}=x - \frac{3 \pi}{4} +k2 \pi \\ 
        4 x - \frac{\pi}{3}=- x + \frac{3 \pi}{4}+k2 \pi
        \end{array} \right.$

$\Leftrightarrow \left[ \begin{array}{l} 
        3 x=- \frac{5 \pi}{12} +k2 \pi \\ 
        5 x=\frac{13 \pi}{12}+k2 \pi
        \end{array} \right. $

$\Leftrightarrow \left[ \begin{array}{l} 
        x=- \frac{5 \pi}{36} + k\frac{2 \pi}{3} \\ 
        x=\frac{13 \pi}{60}+ k\frac{2 \pi}{5}
        \end{array} \right. , k\in \mathbb{Z} $

 
 }\end{ex}

\Closesolutionfile{ans}
{\bf PHẦN II. Câu trắc nghiệm đúng sai.}
\setcounter{ex}{0}
\Opensolutionfile{ans}[ans/ans008-2]
\begin{ex}
 Cho $\sin x=\frac{1}{4}, x\in \left( - \frac{3 \pi}{2};- \pi \right)$. Xét tính đúng-sai của các khẳng định sau.
\choiceTFt
{ $\cos x=\frac{\sqrt{15}}{4}$ }
   { $\sin 2a=- \frac{\sqrt{15}}{16}$  }
     { $\cos 2a=- \frac{7}{8}$  }
    { \True $\sin\left(a+\frac{\pi}{6}\right)=- \sqrt{\frac{9}{32} - \frac{3 \sqrt{5}}{32}}$ }
\loigiai{ 
 

 a) Khẳng định đã cho là khẳng định sai.

 Vì $x \in \left( - \frac{3 \pi}{2};- \pi \right)$ nên $\cos x < 0$.

$\cos x =-\sqrt{1-\frac{1}{16}}=- \frac{\sqrt{15}}{4}$.

b) Khẳng định đã cho là khẳng định sai.

 $\sin 2a=2\sin a \cos a=2.\frac{1}{4}.(- \frac{\sqrt{15}}{4})=- \frac{\sqrt{15}}{8}$.

c) Khẳng định đã cho là khẳng định sai.

 $\cos 2a=1-2\sin^2 a=1-2.\frac{1}{16}=\frac{7}{8}$

d) Khẳng định đã cho là khẳng định đúng.

 $\sin\left(a+\frac{\pi}{6}\right)=\sin a\cos (\frac{\pi}{6})+\cos a \sin (\frac{\pi}{6})=$$\frac{1}{4}.(\frac{\sqrt{3}}{2})+(- \frac{\sqrt{15}}{4}).(\frac{1}{2})=- \sqrt{\frac{9}{32} - \frac{3 \sqrt{5}}{32}}$.

 
 }\end{ex}

\begin{ex}
 Cho hàm số $y=5 \sin{\left(9 x \right)} + 1$ . Xét tính đúng-sai của các khẳng định sau. 
\choiceTFt
{ Tập xác định của hàm số là $D=[-5;5]$ }
   { Hàm số đã cho là hàm số lẻ }
     { Tập giá trị của hàm số đã cho là $T={[-6;8]}$ }
    { \True  Đồ thị cắt trục tung tại điểm có tung độ bằng ${1}$ }
\loigiai{ 
 

 a) Khẳng định đã cho là khẳng định sai.

 Tập xác định của hàm số là $D=\mathbb{R}$.

b) Khẳng định đã cho là khẳng định sai.

 Ta có: Với mọi $x\in \mathbb{R}$ thì $-x\in \mathbb{R}$.

$f(-x)=1 - 5 \sin{\left(9 x \right)}\ne f(x), f(-x)\ne -f(x)$.

Vậy hàm số $y=5 \sin{\left(9 x \right)} + 1$ là hàm số không chẵn, không lẻ.

c) Khẳng định đã cho là khẳng định sai.

 Ta có: $-4 \le 5 \sin{\left(9 x \right)} + 1 \le 6$ nên tập giá trị là ${[-4;6]}$

d) Khẳng định đã cho là khẳng định đúng.

 Cho $x=0\Rightarrow y=1$. Suy ra đồ thị cắt trục tung tại điểm có tung độ bằng ${1}$.

 
 }\end{ex}

\Closesolutionfile{ans}
{\bf PHẦN III. Câu trắc nghiệm trả lời ngắn.}
\setcounter{ex}{0}
\Opensolutionfile{ans}[ans/ans008-3]
\begin{ex}
 Một bánh xe của một loại xe có bán kính ${45}$ cm và quay được 9 vòng trong 5 giây. Tính độ dài quãng đường (theo đơn vị mét) xe đi được trong 7 giây (kết quả làm tròn đến hàng phần mười). 
\shortans[oly]{39,6}

\loigiai{ 
 Một giây bánh xe quay được số vòng là: $\frac{9}{5}$.

Một vòng quay ứng với quãng đường là $2\pi.0,5=1,0\pi$.

Sau ${7}$ giây quãng đường đi được là: $\frac{9}{5}.7.1,0\pi=39,6$:

 
 }\end{ex}

\begin{ex}
 Số nghiệm thuộc đoạn $[- 6 \pi;6 \pi]$ của phương trình $\tan \left(3 x + \frac{\pi}{6}\right)=1$ là
\shortans[oly]{36}

\loigiai{ 
 $\tan \left(3 x + \frac{\pi}{6}\right)=1 \Leftrightarrow 3 x + \frac{\pi}{6} =\frac{\pi}{4}+ k\pi \Leftrightarrow x=\frac{\pi}{36}+k\frac{\pi}{3}, k\in \mathbb{Z}$.

Do $x\in [- 6 \pi;6 \pi]$ nên $- 6 \pi\le \frac{\pi}{36}+k\frac{\pi}{3} \le 6 \pi \Rightarrow - \frac{217}{12}\le k \le \frac{215}{12}$.

Có ${36}$ số k thỏa mãn nên phương trình có ${36}$ nghiệm. 
 }\end{ex}

\Closesolutionfile{ans}

 \begin{center}
-----HẾT-----
\end{center}

 %\newpage 
%\begin{center}
%{\bf BẢNG ĐÁP ÁN MÃ ĐỀ 8 }
%\end{center}
%{\bf Phần 1 }
% \inputansbox{6}{ans008-1}
%{\bf Phần 2 }
% \inputansbox{2}{ans008-2}
%{\bf Phần 3 }
% \inputansbox{6}{ans008-3}
\newpage 




\end{document}