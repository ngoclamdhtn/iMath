\documentclass[12pt,a4paper]{article}
\usepackage[top=1.5cm, bottom=1.5cm, left=2.0cm, right=1.5cm] {geometry}
\usepackage{amsmath,amssymb,txfonts}
\usepackage{tkz-euclide}
\usepackage{setspace}
\usepackage{lastpage}

\usepackage{tikz,tkz-tab}
%\usepackage[solcolor]{ex_test}
%\usepackage[dethi]{ex_test} % Chỉ hiển thị đề thi
\usepackage[loigiai]{ex_test} % Hiển thị lời giải
%\usepackage[color]{ex_test} % Khoanh các đáp án
\everymath{\displaystyle}

\def\colorEX{\color{purple}}
%\def\colorEX{}%Không tô màu đáp án đúng trong tùy chọn loigiai
\renewtheorem{ex}{\color{violet}Câu}
\renewcommand{\FalseEX}{\stepcounter{dapan}{{\bf \textcolor{blue}{\Alph{dapan}.}}}}
\renewcommand{\TrueEX}{\stepcounter{dapan}{{\bf \textcolor{blue}{\Alph{dapan}.}}}}

%---------- Khai báo viết tắt, in đáp án
\newcommand{\hoac}[1]{ %hệ hoặc
    \left[\begin{aligned}#1\end{aligned}\right.}
\newcommand{\heva}[1]{ %hệ và
    \left\{\begin{aligned}#1\end{aligned}\right.}

%Tiêu đề
\newcommand{\tenso}{iMath}
\newcommand{\tentruong}{Phần mềm Tạo đề ngẫu nhiên}
\newcommand{\tenkythi}{ĐỀ ÔN TẬP}
\newcommand{\tenmonthi}{Môn thi: Toán}
\newcommand{\thoigian}{}
\newcommand{\tieude}[1]{
   \begin{tabular}{cm{3cm}cm{3cm}cm{3cm}}
    {\bf \tenso} & & {\bf \tenkythi} \\
    {\bf \tentruong} & & {\bf \tenmonthi}\\
    && {\bf Thời gian: \bf \thoigian \, phút}\\
    && { \fbox{\bf Mã đề: #1}}
   \end{tabular}\\\\
    
   {Họ tên HS: \dotfill Số báo danh \dotfill}\\
}
\newcommand{\chantrang}[2]{\rfoot{Trang \thepage $-$ Mã đề #2}}
\pagestyle{fancy}
\fancyhf{}
\renewcommand{\headrulewidth}{0pt} 
\renewcommand{\footrulewidth}{0pt}

\begin{document}
%Thiết lập giãn dọng 1.5cm 
%\setlength{\lineskip}{1.5em}
%Nội dung trắc nghiệm bắt đầu ở đây


\tieude{003}
\chantrang{\pageref{LastPage}}{003}
\setcounter{page}{1}
{\bf PHẦN I. Câu trắc nghiệm nhiều phương án lựa chọn.}
\setcounter{ex}{0}
\Opensolutionfile{ans}[ans/ans003-1]
\begin{ex}
 Đổi số đo của góc $-530^\circ$ sang radian ta được kết quả bằng\\ 
\choice
{ \True $- \frac{53 \pi}{18}$ }
   { $- \frac{55 \pi}{18}$ }
     { $- \frac{25 \pi}{9}$ }
    { $- \frac{26 \pi}{9}$ }
\loigiai{ 
 Áp dụng công thức chuyển đổi: $-530^\circ=\dfrac{-530.\pi}{180}=- \frac{53 \pi}{18}$. 
 }\end{ex}

\begin{ex}
 Tính $\cot\frac{2 \pi}{3}$.\\ 
\choice
{ $ \frac{\sqrt{3}}{2} $ }
   { $ - \sqrt{3} $ }
     { $ - \frac{1}{2} $ }
    { \True $ - \frac{\sqrt{3}}{3} $ }
\loigiai{ 
  
 }\end{ex}

\begin{ex}
 Cho ${b}$ là góc lượng giác. Tìm khẳng định đúng trong các khẳng định sau.\ 
\choice
{ $\sin (\pi-b)=\cos b$ }
   { $\cos (\pi-b)=\sin b$ }
     { \True $\cot (\pi-b)=-\cot b$ }
    { $\tan \left(\frac{\pi}{2}-b\right)=-\tan b$ }
\loigiai{ 
 $\cot (\pi-b)=-\cot b$ là khẳng định đúng. 
 }\end{ex}

\begin{ex}
 Cho ${\gamma}$ là góc lượng giác. Tìm khẳng định đúng trong các khẳng định sau.\ 
\choice
{ $\tan 2\gamma=\dfrac{\tan \gamma}{1-2\tan^2 \gamma}$ }
   { $\cos 2\gamma=1-2\cos^2 \gamma$ }
     { \True $\cos 2\gamma=2\cos^2 \gamma-1$ }
    { $\sin 2\gamma=\sin \gamma+\cos \gamma$ }
\loigiai{ 
 $\cos 2\gamma=2\cos^2 \gamma-1$ là khẳng định đúng. 
 }\end{ex}

\begin{ex}
 Cho ${u,v}$ là các góc lượng giác. Tìm khẳng định đúng trong các khẳng định sau.\ 
\choice
{ $\sin u \cos v=\dfrac 1 2[\sin(u+v) - \sin(u-v)]$ }
   { $\sin u \sin v=\dfrac 1 2[\cos(u+v) - \cos(u-v)]$ }
     { $\cos u \cos v=\dfrac 1 2[\cos(u+v) - \cos(u-v)]$ }
    { \True $\sin u \cos v=\dfrac 1 2[\sin(u+v) + \sin(u-v)]$ }
\loigiai{ 
 $\sin u \cos v=\dfrac 1 2[\sin(u+v) + \sin(u-v)]$ là khẳng định đúng. 
 }\end{ex}

\begin{ex}
 Cho $\sin x=\frac{9}{11}$ với $x\in \left( 0;\frac{\pi}{2} \right)$. Tính $\sin\left(x- \frac{5 \pi}{6}\right)$.\ 
\choice
{ \True $- \frac{9 \sqrt{3}}{22} - \frac{\sqrt{10}}{11}$ }
   { $- \frac{\sqrt{30}}{11} - \frac{9}{22}$ }
     { $\frac{2 \sqrt{10}}{11} + \frac{9}{11}$ }
    { $- \frac{9 \sqrt{3}}{22} + \frac{\sqrt{10}}{11}$ }
\loigiai{ 
 Vì $x \in \left( 0;\frac{\pi}{2} \right)$ nên $\cos x > 0$.

$\cos x =\sqrt{1-\frac{81}{121}}=\frac{2 \sqrt{10}}{11}$.

$\sin\left(x- \frac{5 \pi}{6}\right)=\sin x\cos (- \frac{5 \pi}{6})+\cos x \sin (- \frac{5 \pi}{6})=$$\frac{9}{11}.(- \frac{\sqrt{3}}{2})+\frac{2 \sqrt{10}}{11}.(- \frac{1}{2})=- \frac{9 \sqrt{3}}{22} - \frac{\sqrt{10}}{11}$. 
 }\end{ex}

\begin{ex}
 Tìm tập xác định của hàm số $y=\tan(4x+5\pi)$.\\ 
\choice
{ $D=\mathbb{R}\backslash\{ - \frac{1}{2}\pi + k \frac{1}{4}\pi\}$ }
   { $D=\mathbb{R}\backslash\{ -1\pi + k \frac{1}{4}\pi\}$ }
     { \True $D=\mathbb{R}\backslash\{ - \frac{9}{8}\pi + k \frac{1}{4}\pi\}$ }
    { $D=\mathbb{R}\backslash\{ - \frac{9}{4}\pi + k \frac{1}{4}\pi\}$ }
\loigiai{ 
  
 }\end{ex}

\begin{ex}
 Nghiệm của phương trình $\cos\left(2 x + \frac{\pi}{6}\right)=\sin\left(- x - \frac{\pi}{6}\right)$ là\ 
\choice
{ $x=\frac{\pi}{3}+k2 \pi, x=- \frac{\pi}{2}+k\frac{2 \pi}{3} (k\in \mathbb{Z})$ }
   { \True $x=\frac{\pi}{2}+k2 \pi, x=- \frac{5 \pi}{18}+k\frac{2 \pi}{3} (k\in \mathbb{Z})$ }
     { $x=\frac{\pi}{3}+k2 \pi, x=- \frac{\pi}{2}+k2 \pi (k\in \mathbb{Z})$ }
    { $x=\frac{\pi}{2}+k\frac{\pi}{3}, x=- \frac{5 \pi}{18}+k\pi (k\in \mathbb{Z})$ }
\loigiai{ 
 $\cos\left(2 x + \frac{\pi}{6}\right)=\sin\left(- x - \frac{\pi}{6}\right) \Leftrightarrow \cos\left(2 x + \frac{\pi}{6}\right)=\cos\left(x + \frac{2 \pi}{3}\right)$

$\Leftrightarrow \left[ \begin{array}{l} 
        2 x + \frac{\pi}{6}=x + \frac{2 \pi}{3} +k2 \pi \\ 
        2 x + \frac{\pi}{6}=- x - \frac{2 \pi}{3}+k2 \pi
        \end{array} \right.$

$\Leftrightarrow \left[ \begin{array}{l} 
        x=\frac{\pi}{2} +k2 \pi \\ 
        3 x=- \frac{5 \pi}{6}+k2 \pi
        \end{array} \right. $

$\Leftrightarrow \left[ \begin{array}{l} 
        x=\frac{\pi}{2} + k2 \pi \\ 
        x=- \frac{5 \pi}{18}+ k\frac{2 \pi}{3}
        \end{array} \right. , k\in \mathbb{Z} $

 
 }\end{ex}

\Closesolutionfile{ans}
{\bf PHẦN II. Câu trắc nghiệm đúng sai.}
\setcounter{ex}{0}
\Opensolutionfile{ans}[ans/ans003-2]
\begin{ex}
 Cho $\sin \gamma=\frac{\sqrt{7}}{9}, \gamma\in \left( 0;\frac{\pi}{2} \right)$. Xét tính đúng-sai của các khẳng định sau.
\choiceTFt
{ $\cos \gamma=- \frac{\sqrt{74}}{9}$ }
   { $\sin 2\gamma=\frac{\sqrt{518}}{81}$  }
     { $\cos 2\gamma=- \frac{67}{81}$  }
    { \True $\sin\left(\gamma+\frac{3 \pi}{4}\right)=- \frac{\sqrt{14}}{18} + \frac{\sqrt{37}}{9}$ }
\loigiai{ 
 

 a) Khẳng định đã cho là khẳng định sai.

 Vì $\gamma \in \left( 0;\frac{\pi}{2} \right)$ nên $\cos \gamma > 0$.

$\cos \gamma =\sqrt{1-\frac{7}{81}}=\frac{\sqrt{74}}{9}$.

b) Khẳng định đã cho là khẳng định sai.

 $\sin 2\gamma=2\sin \gamma \cos \gamma=2.\frac{\sqrt{7}}{9}.\frac{\sqrt{74}}{9}=\frac{2 \sqrt{518}}{81}$.

c) Khẳng định đã cho là khẳng định sai.

 $\cos 2\gamma=1-2\sin^2 \gamma=1-2.\frac{7}{81}=\frac{67}{81}$

d) Khẳng định đã cho là khẳng định đúng.

 $\sin\left(\gamma+\frac{3 \pi}{4}\right)=\sin \gamma\cos (\frac{3 \pi}{4})+\cos \gamma \sin (\frac{3 \pi}{4})=$$\frac{\sqrt{7}}{9}.(- \frac{\sqrt{2}}{2})+\frac{\sqrt{74}}{9}.(\frac{\sqrt{2}}{2})=- \frac{\sqrt{14}}{18} + \frac{\sqrt{37}}{9}$.

 
 }\end{ex}

\begin{ex}
 Cho hàm số $y=- 6 \cos{\left(5 x \right)} - 2$ . Xét tính đúng-sai của các khẳng định sau. 
\choiceTFt
{ \True  Tập xác định của hàm số là $D=\mathbb{R}$ }
   { \True  Hàm số đã cho là hàm số chẵn }
     { Tập giá trị của hàm số đã cho là $T={[-12;-5]}$ }
    { \True  Đồ thị cắt trục tung tại điểm có tung độ bằng ${-8}$ }
\loigiai{ 
 

 a) Khẳng định đã cho là khẳng định đúng.

 Tập xác định của hàm số là $D=\mathbb{R}$.

b) Khẳng định đã cho là khẳng định đúng.

 Ta có: Với mọi $x\in \mathbb{R}$ thì $-x\in \mathbb{R}$.

$f(-x)=- 6 \cos{\left(5 x \right)} - 2=- 6 \cos{\left(5 x \right)} - 2$. Vậy hàm số $y=- 6 \cos{\left(5 x \right)} - 2$ là hàm số chẵn.

c) Khẳng định đã cho là khẳng định sai.

 Ta có: $-8 \le - 6 \cos{\left(5 x \right)} - 2 \le -8$ nên tập giá trị là ${[-8;-8]}$

d) Khẳng định đã cho là khẳng định đúng.

 Cho $x=0\Rightarrow y=-8$. Suy ra đồ thị cắt trục tung tại điểm có tung độ bằng ${-8}$.

 
 }\end{ex}

\Closesolutionfile{ans}
{\bf PHẦN III. Câu trắc nghiệm trả lời ngắn.}
\setcounter{ex}{0}
\Opensolutionfile{ans}[ans/ans003-3]
\begin{ex}
 Một bánh xe của một loại xe có bán kính ${52}$ cm và quay được 9 vòng trong 4 giây. Tính độ dài quãng đường (theo đơn vị mét) xe đi được trong 3 giây (kết quả làm tròn đến hàng phần mười). 
\shortans[oly]{21,2}

\loigiai{ 
 Một giây bánh xe quay được số vòng là: $\frac{9}{4}$.

Một vòng quay ứng với quãng đường là $2\pi.0,5=1,0\pi$.

Sau ${3}$ giây quãng đường đi được là: $\frac{9}{4}.3.1,0\pi=21,2$:

 
 }\end{ex}

\begin{ex}
 Số nghiệm thuộc đoạn $[- 4 \pi;4 \pi]$ của phương trình $\tan \left(x + \frac{3 \pi}{4}\right)=\sqrt{3}$ là
\shortans[oly]{8}

\loigiai{ 
 $\tan \left(x + \frac{3 \pi}{4}\right)=\sqrt{3} \Leftrightarrow x + \frac{3 \pi}{4} =\frac{\pi}{3}+ k\pi \Leftrightarrow x=- \frac{5 \pi}{12}+k\pi, k\in \mathbb{Z}$.

Do $x\in [- 4 \pi;4 \pi]$ nên $- 4 \pi\le - \frac{5 \pi}{12}+k\pi \le 4 \pi \Rightarrow - \frac{43}{12}\le k \le \frac{53}{12}$.

Có ${8}$ số k thỏa mãn nên phương trình có ${8}$ nghiệm. 
 }\end{ex}

\Closesolutionfile{ans}

 \begin{center}
-----HẾT-----
\end{center}

 %\newpage 
%\begin{center}
%{\bf BẢNG ĐÁP ÁN MÃ ĐỀ 3 }
%\end{center}
%{\bf Phần 1 }
% \inputansbox{6}{ans003-1}
%{\bf Phần 2 }
% \inputansbox{2}{ans003-2}
%{\bf Phần 3 }
% \inputansbox{6}{ans003-3}
\newpage 



\end{document}