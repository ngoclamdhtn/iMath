\documentclass[12pt,a4paper]{article}
\usepackage[top=1.5cm, bottom=1.5cm, left=2.0cm, right=1.5cm] {geometry}
\usepackage{amsmath,amssymb,txfonts}
\usepackage{tkz-euclide}
\usepackage{setspace}
\usepackage{lastpage}

\usepackage{tikz,tkz-tab}
%\usepackage[solcolor]{ex_test}
%\usepackage[dethi]{ex_test} % Chỉ hiển thị đề thi
\usepackage[loigiai]{ex_test} % Hiển thị lời giải
%\usepackage[color]{ex_test} % Khoanh các đáp án
\everymath{\displaystyle}

\def\colorEX{\color{purple}}
%\def\colorEX{}%Không tô màu đáp án đúng trong tùy chọn loigiai
\renewtheorem{ex}{\color{violet}Câu}
\renewcommand{\FalseEX}{\stepcounter{dapan}{{\bf \textcolor{blue}{\Alph{dapan}.}}}}
\renewcommand{\TrueEX}{\stepcounter{dapan}{{\bf \textcolor{blue}{\Alph{dapan}.}}}}

%---------- Khai báo viết tắt, in đáp án
\newcommand{\hoac}[1]{ %hệ hoặc
    \left[\begin{aligned}#1\end{aligned}\right.}
\newcommand{\heva}[1]{ %hệ và
    \left\{\begin{aligned}#1\end{aligned}\right.}

%Tiêu đề
\newcommand{\tenso}{iMath}
\newcommand{\tentruong}{Phần mềm Tạo đề ngẫu nhiên}
\newcommand{\tenkythi}{ĐỀ ÔN TẬP}
\newcommand{\tenmonthi}{Môn thi: Toán}
\newcommand{\thoigian}{}
\newcommand{\tieude}[1]{
   \begin{tabular}{cm{3cm}cm{3cm}cm{3cm}}
    {\bf \tenso} & & {\bf \tenkythi} \\
    {\bf \tentruong} & & {\bf \tenmonthi}\\
    && {\bf Thời gian: \bf \thoigian \, phút}\\
    && { \fbox{\bf Mã đề: #1}}
   \end{tabular}\\\\
    
   {Họ tên HS: \dotfill Số báo danh \dotfill}\\
}
\newcommand{\chantrang}[2]{\rfoot{Trang \thepage $-$ Mã đề #2}}
\pagestyle{fancy}
\fancyhf{}
\renewcommand{\headrulewidth}{0pt} 
\renewcommand{\footrulewidth}{0pt}

\begin{document}
%Thiết lập giãn dọng 1.5cm 
%\setlength{\lineskip}{1.5em}
%Nội dung trắc nghiệm bắt đầu ở đây


\tieude{001}
\chantrang{\pageref{LastPage}}{001}
\setcounter{page}{1}
{\bf PHẦN I. Câu trắc nghiệm nhiều phương án lựa chọn.}
\setcounter{ex}{0}
\Opensolutionfile{ans}[ans/ans001-1]
\begin{ex}
 Đổi số đo của góc $-630^\circ$ sang radian ta được kết quả bằng\\ 
\choice
{ \True $- \frac{7 \pi}{2}$ }
   { $- \frac{65 \pi}{18}$ }
     { $- \frac{10 \pi}{3}$ }
    { $- \frac{31 \pi}{9}$ }
\loigiai{ 
 Áp dụng công thức chuyển đổi: $-630^\circ=\dfrac{-630.\pi}{180}=- \frac{7 \pi}{2}$. 
 }\end{ex}

\begin{ex}
 Tính $\cot\frac{13 \pi}{6}$.\\ 
\choice
{ $ \frac{1}{2} $ }
   { \True $ \sqrt{3} $ }
     { $ \frac{\sqrt{3}}{2} $ }
    { $ \frac{\sqrt{3}}{3} $ }
\loigiai{ 
  
 }\end{ex}

\begin{ex}
 Cho ${b}$ là góc lượng giác. Tìm khẳng định đúng trong các khẳng định sau.\ 
\choice
{ $\cot \left(\frac{\pi}{2}-b\right)=-\cot b$ }
   { $\cos (\pi-b)=\sin b$ }
     { \True $\tan \left(\frac{\pi}{2}-b\right)=\cot b$ }
    { $\sin (\pi-b)=-\sin b$ }
\loigiai{ 
 $\tan \left(\frac{\pi}{2}-b\right)=\cot b$ là khẳng định đúng. 
 }\end{ex}

\begin{ex}
 Cho ${\beta}$ là góc lượng giác. Tìm khẳng định đúng trong các khẳng định sau.\ 
\choice
{ $\tan 2\beta=\dfrac{\tan \beta}{1-2\tan^2 \beta}$ }
   { $\cos 2\beta=2\sin \beta\cos \beta$ }
     { \True $\cos 2\beta=2\cos^2 \beta-1$ }
    { $\sin 2\beta=\sin \beta\cos \beta$ }
\loigiai{ 
 $\cos 2\beta=2\cos^2 \beta-1$ là khẳng định đúng. 
 }\end{ex}

\begin{ex}
 Cho ${u,v}$ là các góc lượng giác. Tìm khẳng định đúng trong các khẳng định sau.\ 
\choice
{ $\cos u \cos v=-\dfrac 1 2[\cos(u+v) + \cos(u-v)]$ }
   { $\sin u \cos v=\dfrac 1 2[\sin(u+v) - \sin(u-v)]$ }
     { \True $\sin u \sin v=\dfrac 1 2[\cos(u-v) - \cos(u+v)]$ }
    { $\sin u \sin v=-\dfrac 1 2[\cos(u-v) - \cos(u+v)]$ }
\loigiai{ 
 $\sin u \sin v=\dfrac 1 2[\cos(u-v) - \cos(u+v)]$ là khẳng định đúng. 
 }\end{ex}

\begin{ex}
 Cho $\sin x=\frac{5}{6}$ với $x\in \left( 0;\frac{\pi}{2} \right)$. Tính $\sin\left(x- \frac{3 \pi}{4}\right)$.\ 
\choice
{ $- \frac{5 \sqrt{2}}{12} + \frac{\sqrt{22}}{12}$ }
   { $\frac{\sqrt{11}}{6} + \frac{11}{6}$ }
     { $\frac{\sqrt{11}}{6} + \frac{5}{6}$ }
    { \True $- \frac{5 \sqrt{2}}{12} - \frac{\sqrt{22}}{12}$ }
\loigiai{ 
 Vì $x \in \left( 0;\frac{\pi}{2} \right)$ nên $\cos x > 0$.

$\cos x =\sqrt{1-\frac{25}{36}}=\frac{\sqrt{11}}{6}$.

$\sin\left(x- \frac{3 \pi}{4}\right)=\sin x\cos (- \frac{3 \pi}{4})+\cos x \sin (- \frac{3 \pi}{4})=$$\frac{5}{6}.(- \frac{\sqrt{2}}{2})+\frac{\sqrt{11}}{6}.(- \frac{\sqrt{2}}{2})=- \frac{5 \sqrt{2}}{12} - \frac{\sqrt{22}}{12}$. 
 }\end{ex}

\begin{ex}
 Tìm tập xác định của hàm số $y=\tan(6x+5\pi)$.\\ 
\choice
{ $D=\mathbb{R}\backslash\{ - \frac{1}{3}\pi + k \frac{1}{6}\pi\}$ }
   { $D=\mathbb{R}\backslash\{ - \frac{2}{3}\pi + k \frac{1}{6}\pi\}$ }
     { $D=\mathbb{R}\backslash\{ - \frac{3}{2}\pi + k \frac{1}{6}\pi\}$ }
    { \True $D=\mathbb{R}\backslash\{ - \frac{3}{4}\pi + k \frac{1}{6}\pi\}$ }
\loigiai{ 
  
 }\end{ex}

\begin{ex}
 Nghiệm của phương trình $\cos\left(3 x + \frac{\pi}{6}\right)=\sin\left(- x - \frac{\pi}{4}\right)$ là\ 
\choice
{ $x=\frac{7 \pi}{24}+k\frac{\pi}{4}, x=- \frac{11 \pi}{48}+k\frac{\pi}{2} (k\in \mathbb{Z})$ }
   { $x=\frac{13 \pi}{48}+k2 \pi, x=- \frac{7 \pi}{24}+k2 \pi (k\in \mathbb{Z})$ }
     { $x=\frac{13 \pi}{48}+k\pi, x=- \frac{7 \pi}{24}+k\frac{\pi}{2} (k\in \mathbb{Z})$ }
    { \True $x=\frac{7 \pi}{24}+k\pi, x=- \frac{11 \pi}{48}+k\frac{\pi}{2} (k\in \mathbb{Z})$ }
\loigiai{ 
 $\cos\left(3 x + \frac{\pi}{6}\right)=\sin\left(- x - \frac{\pi}{4}\right) \Leftrightarrow \cos\left(3 x + \frac{\pi}{6}\right)=\cos\left(x + \frac{3 \pi}{4}\right)$

$\Leftrightarrow \left[ \begin{array}{l} 
        3 x + \frac{\pi}{6}=x + \frac{3 \pi}{4} +k2 \pi \\ 
        3 x + \frac{\pi}{6}=- x - \frac{3 \pi}{4}+k2 \pi
        \end{array} \right.$

$\Leftrightarrow \left[ \begin{array}{l} 
        2 x=\frac{7 \pi}{12} +k2 \pi \\ 
        4 x=- \frac{11 \pi}{12}+k2 \pi
        \end{array} \right. $

$\Leftrightarrow \left[ \begin{array}{l} 
        x=\frac{7 \pi}{24} + k\pi \\ 
        x=- \frac{11 \pi}{48}+ k\frac{\pi}{2}
        \end{array} \right. , k\in \mathbb{Z} $

 
 }\end{ex}

\Closesolutionfile{ans}
{\bf PHẦN II. Câu trắc nghiệm đúng sai.}
\setcounter{ex}{0}
\Opensolutionfile{ans}[ans/ans001-2]
\begin{ex}
 Cho $\sin \alpha=\frac{9}{11}, \alpha\in \left( 0;\frac{\pi}{2} \right)$. Xét tính đúng-sai của các khẳng định sau.
\choiceTFt
{ \True $\cos \alpha=\frac{2 \sqrt{10}}{11}$ }
   { \True $\sin 2\alpha=\frac{36 \sqrt{10}}{121}$ }
     { \True $\cos 2\alpha=- \frac{41}{121}$ }
    { $\sin\left(\alpha+\frac{\pi}{2}\right)=\frac{2 \sqrt{10}}{11} + \frac{9}{11}$ }
\loigiai{ 
 

 a) Khẳng định đã cho là khẳng định đúng.

 Vì $\alpha \in \left( 0;\frac{\pi}{2} \right)$ nên $\cos \alpha > 0$.

$\cos \alpha =\sqrt{1-\frac{81}{121}}=\frac{2 \sqrt{10}}{11}$.

b) Khẳng định đã cho là khẳng định đúng.

 $\sin 2\alpha=2\sin \alpha \cos \alpha=2.\frac{9}{11}.\frac{2 \sqrt{10}}{11}=\frac{36 \sqrt{10}}{121}$.

c) Khẳng định đã cho là khẳng định đúng.

 $\cos 2\alpha=1-2\sin^2 \alpha=1-2.\frac{81}{121}=- \frac{41}{121}$

d) Khẳng định đã cho là khẳng định sai.

 $\sin\left(\alpha+\frac{\pi}{2}\right)=\sin \alpha\cos (\frac{\pi}{2})+\cos \alpha \sin (\frac{\pi}{2})=$$\frac{9}{11}.(0)+\frac{2 \sqrt{10}}{11}.(1)=\frac{2 \sqrt{10}}{11}$.

 
 }\end{ex}

\begin{ex}
 Cho hàm số $y=3 - \cos{\left(7 x \right)}$ . Xét tính đúng-sai của các khẳng định sau. 
\choiceTFt
{ \True  Tập xác định của hàm số là $D=\mathbb{R}$ }
   { Hàm số đã cho là hàm số lẻ }
     { Tập giá trị của hàm số đã cho là $T={[0;3]}$ }
    { \True  Đồ thị cắt trục tung tại điểm có tung độ bằng ${2}$ }
\loigiai{ 
 

 a) Khẳng định đã cho là khẳng định đúng.

 Tập xác định của hàm số là $D=\mathbb{R}$.

b) Khẳng định đã cho là khẳng định sai.

 Ta có: Với mọi $x\in \mathbb{R}$ thì $-x\in \mathbb{R}$.

$f(-x)=3 - \cos{\left(7 x \right)}=3 - \cos{\left(7 x \right)}$. Vậy hàm số $y=3 - \cos{\left(7 x \right)}$ là hàm số chẵn.

c) Khẳng định đã cho là khẳng định sai.

 Ta có: $2 \le 3 - \cos{\left(7 x \right)} \le 2$ nên tập giá trị là ${[2;2]}$

d) Khẳng định đã cho là khẳng định đúng.

 Cho $x=0\Rightarrow y=2$. Suy ra đồ thị cắt trục tung tại điểm có tung độ bằng ${2}$.

 
 }\end{ex}

\Closesolutionfile{ans}
{\bf PHẦN III. Câu trắc nghiệm trả lời ngắn.}
\setcounter{ex}{0}
\Opensolutionfile{ans}[ans/ans001-3]
\begin{ex}
 Một bánh xe của một loại xe có bán kính ${49}$ cm và quay được 7 vòng trong 5 giây. Tính độ dài quãng đường (theo đơn vị mét) xe đi được trong 3 giây (kết quả làm tròn đến hàng phần mười). 
\shortans[oly]{13,2}

\loigiai{ 
 Một giây bánh xe quay được số vòng là: $\frac{7}{5}$.

Một vòng quay ứng với quãng đường là $2\pi.0,5=1,0\pi$.

Sau ${3}$ giây quãng đường đi được là: $\frac{7}{5}.3.1,0\pi=13,2$:

 
 }\end{ex}

\begin{ex}
 Số nghiệm thuộc đoạn $[- 10 \pi;10 \pi]$ của phương trình $\tan \left(x + \frac{\pi}{4}\right)=\frac{\sqrt{3}}{3}$ là
\shortans[oly]{20}

\loigiai{ 
 $\tan \left(x + \frac{\pi}{4}\right)=\frac{\sqrt{3}}{3} \Leftrightarrow x + \frac{\pi}{4} =\frac{\pi}{6}+ k\pi \Leftrightarrow x=- \frac{\pi}{12}+k\pi, k\in \mathbb{Z}$.

Do $x\in [- 10 \pi;10 \pi]$ nên $- 10 \pi\le - \frac{\pi}{12}+k\pi \le 10 \pi \Rightarrow - \frac{119}{12}\le k \le \frac{121}{12}$.

Có ${20}$ số k thỏa mãn nên phương trình có ${20}$ nghiệm. 
 }\end{ex}

\Closesolutionfile{ans}

 \begin{center}
-----HẾT-----
\end{center}

 %\newpage 
%\begin{center}
%{\bf BẢNG ĐÁP ÁN MÃ ĐỀ 1 }
%\end{center}
%{\bf Phần 1 }
% \inputansbox{6}{ans001-1}
%{\bf Phần 2 }
% \inputansbox{2}{ans001-2}
%{\bf Phần 3 }
% \inputansbox{6}{ans001-3}
\newpage 



\end{document}