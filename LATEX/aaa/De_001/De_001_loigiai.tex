\documentclass[12pt,a4paper]{article}
\usepackage[top=1.5cm, bottom=1.5cm, left=2.0cm, right=1.5cm] {geometry}
\usepackage{amsmath,amssymb,txfonts}
\usepackage{tkz-euclide}
\usepackage{setspace}
\usepackage{lastpage}

\usepackage{tikz,tkz-tab}
%\usepackage[solcolor]{ex_test}
%\usepackage[dethi]{ex_test} % Chỉ hiển thị đề thi
\usepackage[loigiai]{ex_test} % Hiển thị lời giải
%\usepackage[color]{ex_test} % Khoanh các đáp án
\everymath{\displaystyle}

\def\colorEX{\color{purple}}
%\def\colorEX{}%Không tô màu đáp án đúng trong tùy chọn loigiai
\renewtheorem{ex}{\color{violet}Câu}
\renewcommand{\FalseEX}{\stepcounter{dapan}{{\bf \textcolor{blue}{\Alph{dapan}.}}}}
\renewcommand{\TrueEX}{\stepcounter{dapan}{{\bf \textcolor{blue}{\Alph{dapan}.}}}}

%---------- Khai báo viết tắt, in đáp án
\newcommand{\hoac}[1]{ %hệ hoặc
    \left[\begin{aligned}#1\end{aligned}\right.}
\newcommand{\heva}[1]{ %hệ và
    \left\{\begin{aligned}#1\end{aligned}\right.}

%Tiêu đề
\newcommand{\tenso}{iMath}
\newcommand{\tentruong}{Phần mềm Tạo đề ngẫu nhiên}
\newcommand{\tenkythi}{ĐỀ ÔN TẬP}
\newcommand{\tenmonthi}{Môn thi: Toán}
\newcommand{\thoigian}{}
\newcommand{\tieude}[1]{
   \begin{tabular}{cm{3cm}cm{3cm}cm{3cm}}
    {\bf \tenso} & & {\bf \tenkythi} \\
    {\bf \tentruong} & & {\bf \tenmonthi}\\
    && {\bf Thời gian: \bf \thoigian \, phút}\\
    && { \fbox{\bf Mã đề: #1}}
   \end{tabular}\\\\
    
   {Họ tên HS: \dotfill Số báo danh \dotfill}\\
}
\newcommand{\chantrang}[2]{\rfoot{Trang \thepage $-$ Mã đề #2}}
\pagestyle{fancy}
\fancyhf{}
\renewcommand{\headrulewidth}{0pt} 
\renewcommand{\footrulewidth}{0pt}

\begin{document}
%Thiết lập giãn dọng 1.5cm 
%\setlength{\lineskip}{1.5em}
%Nội dung trắc nghiệm bắt đầu ở đây


\tieude{001}
\chantrang{\pageref{LastPage}}{001}
\setcounter{page}{1}
\begin{ex}
	Cho các đường thẳng ${\Delta,b}$ và các mặt phẳng ${(\beta), (R)}$. Khẳng định nào sau đây là khẳng định đúng? 
	\choice
	{ Qua một điểm ${A}$ nằm ngoài mặt phẳng ${(\beta)}$ có một và chỉ một đường thẳng song song với mặt phẳng ${(\beta)}$ }
	{ \True Nếu đường thẳng ${\Delta}$ không nằm trong mặt phẳng ${(\beta)}$ và song song với một đường thẳng nào đó nằm trong ${(\beta)}$ thì ${\Delta}$ song song với ${(\beta)}$ }
	{ Nếu ${\Delta}$ song song với ${b}$ và $b\subset (\beta)$ thì ${\Delta}$ song song với ${(\beta)}$ }
	{ Nếu ${\Delta}$ song song với mặt phẳng ${(\beta)}$ thì ${\Delta}$ song song với mọi đường thẳng nằm trên mặt phẳng ${(\beta)}$ }
	\loigiai{ 
		Nếu đường thẳng ${\Delta}$ không nằm trong mặt phẳng ${(\beta)}$ và song song với một đường thẳng nào đó nằm trong ${(\beta)}$ thì ${\Delta}$ song song với ${(\beta)}$ là khẳng định đúng. 
}\end{ex}

\begin{ex}
	Cho các đường thẳng ${n,d}$ và các mặt phẳng ${(Q), (R)}$. Khẳng định nào sau đây là khẳng định đúng? 
	\choice
	{ Nếu ${n}$ song song với mặt phẳng ${(Q)}$ thì ${n}$ song song với mọi đường thẳng nằm trên mặt phẳng ${(Q)}$ }
	{ Nếu đường thẳng ${n}$ không nằm trong mặt phẳng ${(Q)}$ thì ${n}$ song song với ${(Q)}$ }
	{ Nếu ${n}$ song song với mặt phẳng ${(Q)}$ thì mọi mặt phẳng chứa ${n}$ đều song song với mặt phẳng ${(Q)}$ }
	{ \True ${n}$ song song với mặt phẳng ${(Q)}$ khi và chỉ khi ${n}$ và ${(Q)}$ không có điểm chung }
	\loigiai{ 
		${n}$ song song với mặt phẳng ${(Q)}$ khi và chỉ khi ${n}$ và ${(Q)}$ không có điểm chung là khẳng định đúng. 
}\end{ex}

\begin{ex}
	Cho các đường thẳng ${b,n}$ và các mặt phẳng ${(Q), (\alpha)}$. Khẳng định nào sau đây là khẳng định đúng? 
	\choice
	{ Nếu ${b}$ song song với ${n}$ và ${n}$ song song với mặt phẳng ${(Q)}$ thì ${b}$ song song với mặt phẳng ${(Q)}$ }
	{ \True Nếu ${b}$ song song với mặt phẳng ${(Q)}$ và $n\subset (Q)$ thì ${b}$ song song ${n}$ hoặc ${b,n}$ chéo nhau }
	{ Nếu ${b}$ song song với mặt phẳng ${(Q)}$ và $n\subset (Q)$ thì ${b,n}$ chéo nhau }
	{ Qua một điểm ${N}$ nằm ngoài mặt phẳng ${(Q)}$ có một và chỉ một đường thẳng song song với mặt phẳng ${(Q)}$ }
	\loigiai{ 
		Nếu ${b}$ song song với mặt phẳng ${(Q)}$ và $n\subset (Q)$ thì ${b}$ song song ${n}$ hoặc ${b,n}$ chéo nhau là khẳng định đúng. 
}\end{ex}

\begin{ex}
	Cho các đường thẳng ${n,a}$ và các mặt phẳng ${(R), (\beta)}$. Khẳng định nào sau đây là khẳng định đúng? 
	\choice
	{ Nếu đường thẳng ${n}$ không nằm trong mặt phẳng ${(R)}$ thì ${n}$ song song với ${(R)}$ }
	{ Nếu ${n}$ song song với mặt phẳng ${(R)}$ thì ${n}$ song song với mọi đường thẳng nằm trên mặt phẳng ${(R)}$ }
	{ \True Nếu đường thẳng ${n}$ không nằm trong mặt phẳng ${(R)}$ và song song với một đường thẳng nào đó nằm trong ${(R)}$ thì ${n}$ song song với ${(R)}$ }
	{ Nếu ${n}$ song song với ${a}$ và ${a}$ song song với ${d}$ thì ${n}$ song song với ${d}$ }
	\loigiai{ 
		Nếu đường thẳng ${n}$ không nằm trong mặt phẳng ${(R)}$ và song song với một đường thẳng nào đó nằm trong ${(R)}$ thì ${n}$ song song với ${(R)}$ là khẳng định đúng. 
}\end{ex}

\begin{ex}
	Cho các đường thẳng ${a,m}$ và các mặt phẳng ${(R), (\alpha)}$. Khẳng định nào sau đây là khẳng định đúng? 
	\choice
	{ Nếu ${a}$ song song với ${m}$ và $m\subset (R)$ thì ${a}$ song song với ${(R)}$ }
	{ Nếu đường thẳng ${a}$ song song với một đường thẳng ${m}$ nào đó nằm trong ${(R)}$ thì ${a}$ song song với ${(R)}$ }
	{ Nếu đường thẳng ${a}$ không nằm trong mặt phẳng ${(R)}$ thì ${a}$ song song với ${(R)}$ }
	{ \True ${a}$ song song với mặt phẳng ${(R)}$ khi và chỉ khi ${a}$ và ${(R)}$ không có điểm chung }
	\loigiai{ 
		${a}$ song song với mặt phẳng ${(R)}$ khi và chỉ khi ${a}$ và ${(R)}$ không có điểm chung là khẳng định đúng. 
}\end{ex}

\begin{ex}
	Cho các đường thẳng ${a,m}$ và các mặt phẳng ${(\alpha), (R)}$. Khẳng định nào sau đây là khẳng định đúng? 
	\choice
	{ Nếu ${a}$ song song với ${m}$ và ${m}$ song song với mặt phẳng ${(\alpha)}$ thì ${a}$ song song với mặt phẳng ${(\alpha)}$ }
	{ Nếu hai đường thẳng phân biệt ${a,m}$ cùng song song với mặt phẳng ${(\alpha)}$ thì ${a,m}$ song song nhau }
	{ Nếu đường thẳng ${a}$ song song với một đường thẳng ${m}$ nào đó nằm trong ${(\alpha)}$ thì ${a}$ song song với ${(\alpha)}$ }
	{ \True Nếu ${a}$ song song với mặt phẳng ${(\alpha)}$ thì có một đường thẳng ${b}$ nằm trong ${(\alpha)}$ sao ${a}$ và ${a}$ đồng phẳng }
	\loigiai{ 
		Nếu ${a}$ song song với mặt phẳng ${(\alpha)}$ thì có một đường thẳng ${b}$ nằm trong ${(\alpha)}$ sao ${a}$ và ${a}$ đồng phẳng là khẳng định đúng. 
}\end{ex}

\begin{ex}
	Cho các đường thẳng ${a,b}$ và các mặt phẳng ${(Q), (\gamma)}$. Khẳng định nào sau đây là khẳng định đúng? 
	\choice
	{ Nếu ${a}$ song song với ${b}$ và ${b}$ song song với ${d}$ thì ${a}$ song song với ${d}$ }
	{ Nếu ${a}$ song song với mặt phẳng ${(Q)}$ thì mọi mặt phẳng chứa ${a}$ đều song song với mặt phẳng ${(Q)}$ }
	{ Nếu ${a}$ song song với ${b}$ và ${b}$ song song với mặt phẳng ${(Q)}$ thì ${a}$ song song với mặt phẳng ${(Q)}$ }
	{ \True Nếu ${a}$ song song với mặt phẳng ${(Q)}$ thì có một đường thẳng ${d}$ nằm trong ${(Q)}$ sao ${a}$ và ${a}$ đồng phẳng }
	\loigiai{ 
		Nếu ${a}$ song song với mặt phẳng ${(Q)}$ thì có một đường thẳng ${d}$ nằm trong ${(Q)}$ sao ${a}$ và ${a}$ đồng phẳng là khẳng định đúng. 
}\end{ex}

\begin{ex}
	Cho các đường thẳng ${n,b}$ và các mặt phẳng ${(\beta), (P)}$. Khẳng định nào sau đây là khẳng định đúng? 
	\choice
	{ \True Nếu ${n}$ song song với mặt phẳng ${(\beta)}$ và $b\subset (\beta)$ thì ${n}$ và ${b}$ không có điểm chung }
	{ Nếu ${n}$ song song với mặt phẳng ${(\beta)}$ thì mọi mặt phẳng chứa ${n}$ đều song song với mặt phẳng ${(\beta)}$ }
	{ Nếu ${n}$ song song với mặt phẳng ${(\beta)}$ thì ${n}$ song song với mọi đường thẳng nằm trên mặt phẳng ${(\beta)}$ }
	{ Nếu ${n}$ song song với ${b}$ và $b\subset (\beta)$ thì ${n}$ song song với ${(\beta)}$ }
	\loigiai{ 
		Nếu ${n}$ song song với mặt phẳng ${(\beta)}$ và $b\subset (\beta)$ thì ${n}$ và ${b}$ không có điểm chung là khẳng định đúng. 
}\end{ex}

\begin{ex}
	Cho các đường thẳng ${a,d}$ và các mặt phẳng ${(R), (\beta)}$. Khẳng định nào sau đây là khẳng định đúng? 
	\choice
	{ \True ${a}$ song song với mặt phẳng ${(R)}$ khi và chỉ khi ${a}$ và ${(R)}$ không có điểm chung }
	{ Nếu ${a}$ song song với ${d}$ và ${d}$ song song với mặt phẳng ${(R)}$ thì ${a}$ song song với mặt phẳng ${(R)}$ }
	{ Nếu đường thẳng ${a}$ không nằm trong mặt phẳng ${(R)}$ thì ${a}$ song song với ${(R)}$ }
	{ Qua một điểm ${B}$ nằm ngoài mặt phẳng ${(R)}$ có một và chỉ một đường thẳng song song với mặt phẳng ${(R)}$ }
	\loigiai{ 
		${a}$ song song với mặt phẳng ${(R)}$ khi và chỉ khi ${a}$ và ${(R)}$ không có điểm chung là khẳng định đúng. 
}\end{ex}

\begin{ex}
	Cho các đường thẳng ${a,\Delta}$ và các mặt phẳng ${(Q), (\gamma)}$. Khẳng định nào sau đây là khẳng định đúng? 
	\choice
	{ Nếu ${a}$ song song với ${\Delta}$ và ${\Delta}$ song song với mặt phẳng ${(Q)}$ thì ${a}$ song song với mặt phẳng ${(Q)}$ }
	{ \True Nếu ${a}$ song song với mặt phẳng ${(Q)}$ thì có một đường thẳng ${d}$ nằm trong ${(Q)}$ sao ${a}$ và ${a}$ đồng phẳng }
	{ Nếu đường thẳng ${a}$ không nằm trong mặt phẳng ${(Q)}$ thì ${a}$ song song với ${(Q)}$ }
	{ Nếu ${a}$ song song với mặt phẳng ${(Q)}$ thì ${a}$ song song với mọi đường thẳng nằm trên mặt phẳng ${(Q)}$ }
	\loigiai{ 
		Nếu ${a}$ song song với mặt phẳng ${(Q)}$ thì có một đường thẳng ${d}$ nằm trong ${(Q)}$ sao ${a}$ và ${a}$ đồng phẳng là khẳng định đúng. 
}\end{ex}







\newpage 



\end{document}