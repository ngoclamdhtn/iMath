\documentclass[12pt,a4paper]{article}
\usepackage[top=1.5cm, bottom=1.5cm, left=2.0cm, right=1.5cm] {geometry}
\usepackage{amsmath,amssymb,txfonts}
\usepackage{tkz-euclide}
\usepackage{setspace}
\usepackage{lastpage}

\usepackage{tikz,tkz-tab}
%\usepackage[solcolor]{ex_test}
%\usepackage[dethi]{ex_test} % Chỉ hiển thị đề thi
\usepackage[loigiai]{ex_test} % Hiển thị lời giải
%\usepackage[color]{ex_test} % Khoanh các đáp án
\everymath{\displaystyle}

\def\colorEX{\color{purple}}
%\def\colorEX{}%Không tô màu đáp án đúng trong tùy chọn loigiai
\renewtheorem{ex}{\color{violet}Câu}
\renewcommand{\FalseEX}{\stepcounter{dapan}{{\bf \textcolor{blue}{\Alph{dapan}.}}}}
\renewcommand{\TrueEX}{\stepcounter{dapan}{{\bf \textcolor{blue}{\Alph{dapan}.}}}}

%---------- Khai báo viết tắt, in đáp án
\newcommand{\hoac}[1]{ %hệ hoặc
    \left[\begin{aligned}#1\end{aligned}\right.}
\newcommand{\heva}[1]{ %hệ và
    \left\{\begin{aligned}#1\end{aligned}\right.}

%Tiêu đề
\newcommand{\tenso}{iMath}
\newcommand{\tentruong}{Phần mềm Tạo đề ngẫu nhiên}
\newcommand{\tenkythi}{ĐỀ ÔN TẬP}
\newcommand{\tenmonthi}{Môn thi: Toán}
\newcommand{\thoigian}{}
\newcommand{\tieude}[1]{
   \begin{tabular}{cm{3cm}cm{3cm}cm{3cm}}
    {\bf \tenso} & & {\bf \tenkythi} \\
    {\bf \tentruong} & & {\bf \tenmonthi}\\
    && {\bf Thời gian: \bf \thoigian \, phút}\\
    && { \fbox{\bf Mã đề: #1}}
   \end{tabular}\\\\
    
   {Họ tên HS: \dotfill Số báo danh \dotfill}\\
}
\newcommand{\chantrang}[2]{\rfoot{Trang \thepage $-$ Mã đề #2}}
\pagestyle{fancy}
\fancyhf{}
\renewcommand{\headrulewidth}{0pt} 
\renewcommand{\footrulewidth}{0pt}

\begin{document}
%Thiết lập giãn dọng 1.5cm 
%\setlength{\lineskip}{1.5em}
%Nội dung trắc nghiệm bắt đầu ở đây


\tieude{001}
\chantrang{\pageref{LastPage}}{001}
\setcounter{page}{1}
\begin{ex}
	Trong không gian ${Oxyz}$, cho tam giác ${CMP}$ với $P(2 m + 4;-1;-1)$, C(-2;1;1),M(-5;0;-4). Tìm giá trị của ${m}$ để tam giác ${CMP}$ vuông tại ${M}$.
	\shortans[oly]{-5}
	
	\loigiai{ 
		$\overrightarrow{MC}=(3;1;5)$.
		
		$\overrightarrow{MP}=(2 m + 9;-1;3)$.
		
		Tam giác ${CMP}$ vuông tại ${M}$ khi $\overrightarrow{MC}.\overrightarrow{MP}=0$
		
		$\Rightarrow 3.(2 m + 9)+1.(-1)+5.3=0$
		
		$\Rightarrow 6 m + 41=0$
		
		$\Rightarrow m=-5$.
		
		Đáp án: -5 
}\end{ex}

\begin{ex}
	Trong không gian ${Oxyz}$, cho tam giác ${ADM}$ với $M(m + 5;-1;2)$, A(6;5;-1),D(4;-5;-5). Tìm giá trị của ${m}$ để tam giác ${ADM}$ vuông tại ${D}$.
	\shortans[oly]{-2}
	
	\loigiai{ 
		$\overrightarrow{DA}=(2;10;4)$.
		
		$\overrightarrow{DM}=(m + 1;4;7)$.
		
		Tam giác ${ADM}$ vuông tại ${D}$ khi $\overrightarrow{DA}.\overrightarrow{DM}=0$
		
		$\Rightarrow 2.(m + 1)+10.4+4.7=0$
		
		$\Rightarrow 2 m + 70=0$
		
		$\Rightarrow m=-2$.
		
		Đáp án: -2 
}\end{ex}

\begin{ex}
	Trong không gian ${Oxyz}$, cho tam giác ${NMP}$ với $N(5;6;6),P(m - 3;-4;-6)$, M(-6;2;-4). Tìm giá trị của ${m}$ để tam giác ${NMP}$ vuông tại ${M}$.
	\shortans[oly]{1}
	
	\loigiai{ 
		$\overrightarrow{MN}=(11;4;10)$.
		
		$\overrightarrow{MP}=(m + 3;-6;-2)$.
		
		Tam giác ${NMP}$ vuông tại ${M}$ khi $\overrightarrow{MN}.\overrightarrow{MP}=0$
		
		$\Rightarrow 11.(m + 3)+4.(-6)+10.(-2)=0$
		
		$\Rightarrow 11 m - 11=0$
		
		$\Rightarrow m=1$.
		
		Đáp án: 1 
}\end{ex}

\begin{ex}
	Trong không gian ${Oxyz}$, cho tam giác ${CPE}$ với $C(5;-2;1), P(3;5;3), E(1 - m;-6;4)$. Tìm giá trị của ${m}$ để tam giác ${CPE}$ vuông tại ${C}$.
	\shortans[oly]{7}
	
	\loigiai{ 
		$\overrightarrow{CP}=(-2;7;2)$.
		
		$\overrightarrow{CE}=(- m - 4;-4;3)$.
		
		Tam giác ${CPE}$ vuông tại ${C}$ khi $\overrightarrow{CP}.\overrightarrow{CE}=0$
		
		$\Rightarrow -2.(- m - 4)+7.(-4)+2.3=0$
		
		$\Rightarrow 2 m - 14=0$
		
		$\Rightarrow m=7$.
		
		Đáp án: 7 
}\end{ex}

\begin{ex}
	Trong không gian ${Oxyz}$, cho tam giác ${AEM}$ với $A(5;3;-5), E(-2;6;-4), M(5 - 4 m;4;-1)$. Tìm giá trị của ${m}$ để tam giác ${AEM}$ vuông tại ${E}$ (kết quả làm tròn đến hàng phần mười).
	\shortans[oly]{1,2}
	
	\loigiai{ 
		$\overrightarrow{EA}=(7;-3;-1)$.
		
		$\overrightarrow{EM}=(7 - 4 m;-2;3)$.
		
		Tam giác ${AEM}$ vuông tại ${E}$ khi $\overrightarrow{EA}.\overrightarrow{EM}=0$
		
		$\Rightarrow 7.(7 - 4 m)-3.(-2)-1.3=0$
		
		$\Rightarrow 52 - 28 m=0$
		
		$\Rightarrow m=\frac{5}{4}$.
		
		Đáp án: 1,2 
}\end{ex}






\newpage 



\end{document}