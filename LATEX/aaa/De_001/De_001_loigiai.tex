\documentclass[12pt,a4paper]{article}
\usepackage[top=1.5cm, bottom=1.5cm, left=2.0cm, right=1.5cm] {geometry}
\usepackage{amsmath,amssymb,txfonts}
\usepackage{tkz-euclide}
\usepackage{setspace}
\usepackage{lastpage}

\usepackage{tikz,tkz-tab}
%\usepackage[solcolor]{ex_test}
%\usepackage[dethi]{ex_test} % Chỉ hiển thị đề thi
\usepackage[loigiai]{ex_test} % Hiển thị lời giải
%\usepackage[color]{ex_test} % Khoanh các đáp án
\everymath{\displaystyle}

\def\colorEX{\color{purple}}
%\def\colorEX{}%Không tô màu đáp án đúng trong tùy chọn loigiai
\renewtheorem{ex}{\color{violet}Câu}
\renewcommand{\FalseEX}{\stepcounter{dapan}{{\bf \textcolor{blue}{\Alph{dapan}.}}}}
\renewcommand{\TrueEX}{\stepcounter{dapan}{{\bf \textcolor{blue}{\Alph{dapan}.}}}}

%---------- Khai báo viết tắt, in đáp án
\newcommand{\hoac}[1]{ %hệ hoặc
    \left[\begin{aligned}#1\end{aligned}\right.}
\newcommand{\heva}[1]{ %hệ và
    \left\{\begin{aligned}#1\end{aligned}\right.}

%Tiêu đề
\newcommand{\tenso}{iMath}
\newcommand{\tentruong}{Phần mềm Tạo đề ngẫu nhiên}
\newcommand{\tenkythi}{ĐỀ ÔN TẬP}
\newcommand{\tenmonthi}{Môn thi: Toán}
\newcommand{\thoigian}{}
\newcommand{\tieude}[1]{
   \begin{tabular}{cm{3cm}cm{3cm}cm{3cm}}
    {\bf \tenso} & & {\bf \tenkythi} \\
    {\bf \tentruong} & & {\bf \tenmonthi}\\
    && {\bf Thời gian: \bf \thoigian \, phút}\\
    && { \fbox{\bf Mã đề: #1}}
   \end{tabular}\\\\
    
   {Họ tên HS: \dotfill Số báo danh \dotfill}\\
}
\newcommand{\chantrang}[2]{\rfoot{Trang \thepage $-$ Mã đề #2}}
\pagestyle{fancy}
\fancyhf{}
\renewcommand{\headrulewidth}{0pt} 
\renewcommand{\footrulewidth}{0pt}

\begin{document}
%Thiết lập giãn dọng 1.5cm 
%\setlength{\lineskip}{1.5em}
%Nội dung trắc nghiệm bắt đầu ở đây


\tieude{001}
\chantrang{\pageref{LastPage}}{001}
\setcounter{page}{1}
\begin{ex}
	Cho các điểm $A(5;2;-1), P(1;-4;2), E(2;6;3)$. Xét tính đúng-sai của các khẳng định sau:
	\choiceTFt
	{ Độ dài đoạn thẳng ${AP}$ bằng $3 \sqrt{7}$ }
	{ Tọa độ vectơ $\overrightarrow{b}=2 \overrightarrow{AP} + 2 \overrightarrow{AE}$ là $(-15;-4;16)$ }
	{ \True  Vectơ $\overrightarrow{w}$ thỏa mãn $\overrightarrow{AP} - \overrightarrow{AE} - 2 \overrightarrow{w}=- 3 \overrightarrow{PE}$ thì tọa độ vectơ $\overrightarrow{w}$ là $(1;10;1)$ }
	{ Gọi ${I}$ là chân đường cao hạ từ của ${A}$ của tam giác ${APE}$. Độ dài ${AI}$ bằng ${\frac{\sqrt{257603}}{103}}$ }
	\loigiai{ 
		
		
		a) Khẳng định đã cho là khẳng định sai.
		
		$\overrightarrow{AP}=(-4;-6;3)\Rightarrow $$AP=\sqrt{16+36+9}=\sqrt{61}$.
		
		b) Khẳng định đã cho là khẳng định sai.
		
		$\overrightarrow{AP}=(-4;-6;3)$, $\overrightarrow{AE}=(-3;4;4)$
		
		$\overrightarrow{b}=2 \overrightarrow{AP} + 2 \overrightarrow{AE}=(-14;-4;14)$.
		
		c) Khẳng định đã cho là khẳng định đúng.
		
		$\overrightarrow{AP} - \overrightarrow{AE} - 2 \overrightarrow{w}=- 3 \overrightarrow{PE}\Rightarrow \overrightarrow{w}=- \frac{1}{2}(- \overrightarrow{AP} + \overrightarrow{AE} - 3 \overrightarrow{PE})=(1;10;1)$
		
		d) Khẳng định đã cho là khẳng định sai.
		
		$\overrightarrow{AP}=(-4;-6;3),\overrightarrow{AE}=(-3;4;4) \Rightarrow AP=\sqrt{61},AE=\sqrt{41}$.
		
		$\overrightarrow{AP}.\overrightarrow{AE}=0\Rightarrow $ tam giác ${APE}$ vuông tại ${A}$.
		
		$AI=\dfrac{AP.AE}{\sqrt{AP^2+AE^2}}=\dfrac{\sqrt{61}.\sqrt{41}}{\sqrt{61+61}}=\frac{\sqrt{255102}}{102}$.
		
		
}\end{ex}


	







\newpage 



\end{document}