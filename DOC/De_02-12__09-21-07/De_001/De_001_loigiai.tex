\documentclass[12pt,a4paper]{article}
\usepackage[top=1.5cm, bottom=1.5cm, left=2.0cm, right=1.5cm] {geometry}
\usepackage{amsmath,amssymb,txfonts}
\usepackage{tkz-euclide}
\usepackage{setspace}
\usepackage{lastpage}

\usepackage{tikz,tkz-tab}
%\usepackage[solcolor]{ex_test}
%\usepackage[dethi]{ex_test} % Chỉ hiển thị đề thi
\usepackage[loigiai]{ex_test} % Hiển thị lời giải
%\usepackage[color]{ex_test} % Khoanh các đáp án
\everymath{\displaystyle}

\def\colorEX{\color{purple}}
%\def\colorEX{}%Không tô màu đáp án đúng trong tùy chọn loigiai
\renewtheorem{ex}{\color{violet}Câu}
\renewcommand{\FalseEX}{\stepcounter{dapan}{{\bf \textcolor{blue}{\Alph{dapan}.}}}}
\renewcommand{\TrueEX}{\stepcounter{dapan}{{\bf \textcolor{blue}{\Alph{dapan}.}}}}

%---------- Khai báo viết tắt, in đáp án
\newcommand{\hoac}[1]{ %hệ hoặc
    \left[\begin{aligned}#1\end{aligned}\right.}
\newcommand{\heva}[1]{ %hệ và
    \left\{\begin{aligned}#1\end{aligned}\right.}

%Tiêu đề
\newcommand{\tenso}{}
\newcommand{\tentruong}{}
\newcommand{\tenkythi}{ĐỀ ÔN TẬP}
\newcommand{\tenmonthi}{Môn thi: }
\newcommand{\thoigian}{}
\newcommand{\tieude}[1]{
   \begin{tabular}{cm{3cm}cm{3cm}cm{3cm}}
    {\bf \tenso} & & {\bf \tenkythi} \\
    {\bf \tentruong} & & {\bf \tenmonthi}\\
    && {\bf Thời gian: \bf \thoigian \, phút}\\
    && { \fbox{\bf Mã đề: #1}}
   \end{tabular}\\\\
    
   {Họ tên HS: \dotfill Số báo danh \dotfill}\\
}
\newcommand{\chantrang}[2]{\rfoot{Trang \thepage $-$ Mã đề #2}}
\pagestyle{fancy}
\fancyhf{}
\renewcommand{\headrulewidth}{0pt} 
\renewcommand{\footrulewidth}{0pt}
\usetikzlibrary{shapes.geometric,arrows,calc,intersections,angles,quotes,patterns,snakes,positioning}

\begin{document}
%Thiết lập giãn dọng 1.5cm 
%\setlength{\lineskip}{1.5em}
%Nội dung trắc nghiệm bắt đầu ở đây


\tieude{001}
\chantrang{\pageref{LastPage}}{001}
\setcounter{page}{1}
{\bf PHẦN I. Câu trắc nghiệm nhiều phương án lựa chọn.}
\setcounter{ex}{0}
\Opensolutionfile{ans}[ans/ans001-1]
\begin{ex}
 Số các phần tử của tập hợp $D=\{n\in \mathbb{N}| 11< n <36, n \text{ là số nguyên tố} \}$ 
\choice
{ ${8}$ }
   { \True ${6}$ }
     { ${5}$ }
    { ${7}$ }
\loigiai{ 
 $D=\{13, 17, 19, 23, 29, 31\}$.

Số phần tử là: 6. 
 }\end{ex}

\begin{ex}
 Liệt kê các phần tử của tập hợp $A=\{x\in \mathbb{Z}|-5 \le x \le -2 \}$. \\ 
\choice
{ $A=\{-4, -3, -2\}$ }
   { $A=\{-5, -4, -3\}$ }
     { $A=\{-4, -3\}$ }
    { \True $A=\{-5, -4, -3, -2\}$ }
\loigiai{ 
  
 }\end{ex}

\begin{ex}
 Số phần tử của tập hợp $M=\{x\in \mathbb{Z}|4 x^{2} - 27 x + 18=0 \}$.\\ 
\choice
{ ${2}$ }
   { ${3}$ }
     { \True ${1}$ }
    { ${0}$ }
\loigiai{ 
 Phương trình $4 x^{2} - 27 x + 18=0$ có nghiệm $x_1=\frac{3}{4}\notin \mathbb{Z}, x_2=6 \in \mathbb{Z}$.

Vậy $M=\{6\}$ có 1 phần tử. 
 }\end{ex}

\begin{ex}
 Liệt kê các phần tử của tập hợp $A=\{x\in \mathbb{N}|x >6, x$ là ước của $14\}$. \\ 
\choice
{ $\{1, 2, 3, 6\}$ }
   { $\{0, 1, 2, 3, 4, 5\}$ }
     { \True $\{7, 14\}$ }
    { $\{1, 2\}$ }
\loigiai{ 
  
 }\end{ex}

\begin{ex}
 Liệt kê các phần tử của tập hợp $A=\{x\in \mathbb{N}|x< 35, x$ là bội của $3\}$. \\ 
\choice
{ \True $\{0, 3, 6, 9, 12, 15, 18, 21, 24, 27, 30, 33\}$ }
   { $\{1, 3\}$ }
     { $\{0, 1, 2\}$ }
    { $\{3, 9, 15\}$ }
\loigiai{ 
  
 }\end{ex}

\begin{ex}
 Số các phần tử của tập hợp $B=\{n\in \mathbb{N}| 7< n <34, n \text{ là số nguyên tố} \}$ 
\choice
{ ${6}$ }
   { ${5}$ }
     { ${8}$ }
    { \True ${7}$ }
\loigiai{ 
 $B=\{11, 13, 17, 19, 23, 29, 31\}$.

Số phần tử là: 7. 
 }\end{ex}

\begin{ex}
 Số các phần tử của tập hợp $N=\{n\in \mathbb{N}| 4< n \le 18, n \text{ là số chẵn} \}$ 
\choice
{ ${10}$ }
   { ${8}$ }
     { ${6}$ }
    { \True ${7}$ }
\loigiai{ 
 $N=\{6, 8, 10, 12, 14, 16, 18\}$.

Số phần tử là: 7. 
 }\end{ex}

\begin{ex}
 Cho tập hợp $P=\{{8, 10, 6}\}$. Viết tập hợp ${P}$ dưới dạng chỉ ra tính chất đặc trưng của các phần tử.\\ 
\choice
{ $P=\left\{x\in \mathbb{N}|5 \le x < 12 \right\}$ }
   { \True $P=\left\{x\in \mathbb{Z}|5 \le x < 12, x \text{ là số chẵn } \right\}$ }
     { $P=\left\{x\in \mathbb{Z}|3 \le x < 14, x \text{ là số chẵn } \right\}$ }
    { $P=\left\{x\in \mathbb{Z}|7 < x \le 14,, x \text{ là số nguyên tố } \right\}$ }
\loigiai{ 
 Ta có:$P=\left\{x\in \mathbb{Z}|5 \le x < 12, x \text{ là số chẵn } \right\}$. 
 }\end{ex}

\Closesolutionfile{ans}

 \begin{center}
-----HẾT-----
\end{center}

 %\newpage 
%\begin{center}
%{\bf BẢNG ĐÁP ÁN MÃ ĐỀ 1 }
%\end{center}
%{\bf Phần 1 }
% \inputansbox{6}{ans001-1}



\end{document}