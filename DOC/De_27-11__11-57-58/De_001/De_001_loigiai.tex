\documentclass[12pt,a4paper]{article}
\usepackage[top=1.5cm, bottom=1.5cm, left=2.0cm, right=1.5cm] {geometry}
\usepackage{amsmath,amssymb,txfonts}
\usepackage{tkz-euclide}
\usepackage{setspace}
\usepackage{lastpage}

\usepackage{tikz,tkz-tab}
%\usepackage[solcolor]{ex_test}
%\usepackage[dethi]{ex_test} % Chỉ hiển thị đề thi
\usepackage[loigiai]{ex_test} % Hiển thị lời giải
%\usepackage[color]{ex_test} % Khoanh các đáp án
\everymath{\displaystyle}

\def\colorEX{\color{purple}}
%\def\colorEX{}%Không tô màu đáp án đúng trong tùy chọn loigiai
\renewtheorem{ex}{\color{violet}Câu}
\renewcommand{\FalseEX}{\stepcounter{dapan}{{\bf \textcolor{blue}{\Alph{dapan}.}}}}
\renewcommand{\TrueEX}{\stepcounter{dapan}{{\bf \textcolor{blue}{\Alph{dapan}.}}}}

%---------- Khai báo viết tắt, in đáp án
\newcommand{\hoac}[1]{ %hệ hoặc
    \left[\begin{aligned}#1\end{aligned}\right.}
\newcommand{\heva}[1]{ %hệ và
    \left\{\begin{aligned}#1\end{aligned}\right.}

%Tiêu đề
\newcommand{\tenso}{}
\newcommand{\tentruong}{}
\newcommand{\tenkythi}{ĐỀ ÔN TẬP}
\newcommand{\tenmonthi}{Môn thi: }
\newcommand{\thoigian}{}
\newcommand{\tieude}[1]{
   \begin{tabular}{cm{3cm}cm{3cm}cm{3cm}}
    {\bf \tenso} & & {\bf \tenkythi} \\
    {\bf \tentruong} & & {\bf \tenmonthi}\\
    && {\bf Thời gian: \bf \thoigian \, phút}\\
    && { \fbox{\bf Mã đề: #1}}
   \end{tabular}\\\\
    
   {Họ tên HS: \dotfill Số báo danh \dotfill}\\
}
\newcommand{\chantrang}[2]{\rfoot{Trang \thepage $-$ Mã đề #2}}
\pagestyle{fancy}
\fancyhf{}
\renewcommand{\headrulewidth}{0pt} 
\renewcommand{\footrulewidth}{0pt}
\usetikzlibrary{shapes.geometric,arrows,calc,intersections,angles,quotes,patterns,snakes,positioning}

\begin{document}
%Thiết lập giãn dọng 1.5cm 
%\setlength{\lineskip}{1.5em}
%Nội dung trắc nghiệm bắt đầu ở đây


\tieude{001}
\chantrang{\pageref{LastPage}}{001}
\setcounter{page}{1}
\begin{ex}
	Một đội thợ công nhân dùng gạch cỡ 50x50cm để lát nền cho một toà tháp gồm 6 tầng theo cấu trúc diện tích mặt sàn của tầng trên bằng 70\% diện tích mặt sàn của tầng dưới. Biết diện tích mặt đáy của tháp là 45 ${m^2}$. Hỏi đội công nhân dự định dùng tối thiểu khoảng bao nhiêu viên gạch?
	
	
	\shortans[4]{530}
	
	\loigiai{ 
		Giả sử diện tích mặt sàn tầng 1 là $S_1(m^2)$.
		
		Suy ra, diện tích mặt sàn tầng 2 là $S_2=\frac{7}{10}S_1$.
		
		Diện tích mặt sàn tầng 3 là $S_3=\frac{7}{10}S_2=\frac{49}{100}S_1$.
		
		..........Diện tích mặt sàn tầng 6 là $S_6=\left(\frac{7}{10}\right)^5S_1$.
		
		Tổng diện tích mặt sàn của toà tháp là:
		
		$S=S_1+S_2+...+s_6=45.\dfrac{1-\left(\frac{7}{10}\right)^6 }{1-\frac{7}{10}}=132,353(m^2)$.
		
		Số viên gạch cần dùng là: $132,353:(0,5.0,5)=530$. 
}\end{ex}

\begin{ex}
	Một đội thợ công nhân dùng gạch cỡ 80x80cm để lát nền cho một toà tháp gồm 5 tầng theo cấu trúc diện tích mặt sàn của tầng trên bằng 65\% diện tích mặt sàn của tầng dưới. Biết diện tích mặt đáy của tháp là 50 ${m^2}$. Hỏi đội công nhân dự định dùng tối thiểu khoảng bao nhiêu viên gạch?
	
	
	\shortans[4]{198}
	
	\loigiai{ 
		Giả sử diện tích mặt sàn tầng 1 là $S_1(m^2)$.
		
		Suy ra, diện tích mặt sàn tầng 2 là $S_2=\frac{13}{20}S_1$.
		
		Diện tích mặt sàn tầng 3 là $S_3=\frac{13}{20}S_2=\frac{169}{400}S_1$.
		
		..........Diện tích mặt sàn tầng 5 là $S_5=\left(\frac{13}{20}\right)^4S_1$.
		
		Tổng diện tích mặt sàn của toà tháp là:
		
		$S=S_1+S_2+...+s_5=50.\dfrac{1-\left(\frac{13}{20}\right)^5 }{1-\frac{13}{20}}=126,282(m^2)$.
		
		Số viên gạch cần dùng là: $126,282:(0,8.0,8)=198$. 
}\end{ex}

\begin{ex}
	Một đội thợ công nhân dùng gạch cỡ 50x50cm để lát nền cho một toà tháp gồm 7 tầng theo cấu trúc diện tích mặt sàn của tầng trên bằng 75\% diện tích mặt sàn của tầng dưới. Biết diện tích mặt đáy của tháp là 45 ${m^2}$. Hỏi đội công nhân dự định dùng tối thiểu khoảng bao nhiêu viên gạch?
	
	
	\shortans[4]{624}
	
	\loigiai{ 
		Giả sử diện tích mặt sàn tầng 1 là $S_1(m^2)$.
		
		Suy ra, diện tích mặt sàn tầng 2 là $S_2=\frac{3}{4}S_1$.
		
		Diện tích mặt sàn tầng 3 là $S_3=\frac{3}{4}S_2=\frac{9}{16}S_1$.
		
		..........Diện tích mặt sàn tầng 7 là $S_7=\left(\frac{3}{4}\right)^6S_1$.
		
		Tổng diện tích mặt sàn của toà tháp là:
		
		$S=S_1+S_2+...+s_7=45.\dfrac{1-\left(\frac{3}{4}\right)^7 }{1-\frac{3}{4}}=155,973(m^2)$.
		
		Số viên gạch cần dùng là: $155,973:(0,5.0,5)=624$. 
}\end{ex}
















\end{document}