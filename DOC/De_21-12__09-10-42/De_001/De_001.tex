\documentclass[12pt,a4paper]{article}
\usepackage[top=1.5cm, bottom=1.5cm, left=2.0cm, right=1.5cm] {geometry}
\usepackage{amsmath,amssymb,fontawesome}
\usepackage{tkz-euclide}
\usepackage{setspace}
\usepackage{lastpage}

\usepackage{tikz,tkz-tab}
%\usepackage[solcolor]{ex_test}
\usepackage[dethi]{ex_test} % Chỉ hiển thị đề thi
%\usepackage[loigiai]{ex_test} % Hiển thị lời giải
%\usepackage[color]{ex_test} % Khoanh các đáp án
\usetikzlibrary{shapes.geometric,arrows,calc,intersections,angles,quotes,patterns,snakes,positioning}
\everymath{\displaystyle}

\def\colorEX{\color{purple}}
%\def\colorEX{}%Không tô màu đáp án đúng trong tùy chọn loigiai
\renewtheorem{ex}{\color{violet}Câu}
\renewcommand{\FalseEX}{\stepcounter{dapan}{{\bf \textcolor{blue}{\Alph{dapan}.}}}}
\renewcommand{\TrueEX}{\stepcounter{dapan}{{\bf \textcolor{blue}{\Alph{dapan}.}}}}

%---------- Khai báo viết tắt, in đáp án
\newcommand{\hoac}[1]{ %hệ hoặc
    \left[\begin{aligned}#1\end{aligned}\right.}
\newcommand{\heva}[1]{ %hệ và
    \left\{\begin{aligned}#1\end{aligned}\right.}

%Tiêu đề
\newcommand{\tenso}{iMath}
\newcommand{\tentruong}{Phần mềm Tạo đề ngẫu nhiên}
\newcommand{\tenkythi}{ĐỀ ÔN TẬP}
\newcommand{\tenmonthi}{Môn học: Toán 12}
\newcommand{\thoigian}{}
\newcommand{\tieude}[1]{
    \noindent
     \begin{minipage}[b]{6cm}
    \centerline{\textbf{\fontsize{11}{0}\selectfont \tenso}}
    \centerline{\fontsize{11}{0}\selectfont \tentruong}  
  \end{minipage}\hspace{1cm}
  \begin{minipage}[b]{11cm}
    \centerline{\textbf{\fontsize{11}{0}\selectfont \tenkythi}}
    \centerline{\textbf{\fontsize{11}{0}\selectfont \tenmonthi}}
    \centerline{\textit{\fontsize{11}{0}\selectfont Thời \underline{gian làm bài: \thoigian  } phút }}
  \end{minipage}
  \vspace*{3mm}
  \noindent
  \begin{minipage}[t]{12cm}
    \textbf{Họ, tên thí sinh:}\dotfill\\
    \textbf{Số báo danh:}\dotfill
  \end{minipage}\hfill
  \begin{minipage}[b]{3cm}
    \setlength\fboxrule{1pt}
    \setlength\fboxsep{3pt}
    \vspace*{3mm}\fbox{\bf Mã đề thi #1}
  \end{minipage}\\
}

\newcommand{\chantrang}[2]{\rfoot{Trang \thepage $-$ Mã đề #2}}
\pagestyle{fancy}
\fancyhf{}
\renewcommand{\headrulewidth}{0pt} 
\renewcommand{\footrulewidth}{0pt}

\begin{document}
%Thiết lập giãn dọng 1.5cm 
%\setlength{\lineskip}{1.5em}



%Nội dung trắc nghiệm bắt đầu ở đây


\tieude{001}
\chantrang{\pageref{LastPage}}{001}
\setcounter{page}{1}
{\bf PHẦN III. Câu trắc nghiệm trả lời ngắn.}
\setcounter{ex}{0}
\Opensolutionfile{ans}[ans/ans001-3]
\begin{ex}
 Tìm giá trị của tham số ${m}$ để hàm số $f(x)=\left\{ \begin{array}{l} 
    \dfrac{- 2 x^{2} - 2 x + 12}{8 - 4 x} \text{ khi } x\ne 2\\ 
    m x + 5 \text{ khi } x = 2
    \end{array} \right.$ liên tục tại $x=2$ (kết quả làm tròn đến hàng phần mười).


\shortans[4]{-1,2}

\loigiai{ 
 Ta có: $\mathop{\lim}\limits_{x \to  2}f(x)=\frac{5}{2}$.

$f(2)=2 m + 5$.

Hàm số liên tục tại $x=2$ khi $2 m + 5=\frac{5}{2}$.

Suy ra $m=- \frac{5}{4}=-1,2$. 
 }\end{ex}

\begin{ex}
 Tìm giá trị của tham số ${m}$ để hàm số $f(x)=\left\{ \begin{array}{l} 
    \dfrac{x^{2} - 5 x + 6}{8 - 4 x} \text{ khi } x\ne 2\\ 
    - 2 m x - 4 \text{ khi } x = 2
    \end{array} \right.$ liên tục tại $x=2$ (kết quả làm tròn đến hàng phần mười).


\shortans[4]{-1,1}

\loigiai{ 
 Ta có: $\mathop{\lim}\limits_{x \to  2}f(x)=\frac{1}{4}$.

$f(2)=- 4 m - 4$.

Hàm số liên tục tại $x=2$ khi $- 4 m - 4=\frac{1}{4}$.

Suy ra $m=- \frac{17}{16}=-1,1$. 
 }\end{ex}

\begin{ex}
 Tìm giá trị của tham số ${m}$ để hàm số $f(x)=\left\{ \begin{array}{l} 
    \dfrac{x^{2} + 2 x - 3}{- 3 x - 9} \text{ khi } x\ne -3\\ 
    5 m x + 3 \text{ khi } x = -3
    \end{array} \right.$ liên tục tại $x=-3$ (kết quả làm tròn đến hàng phần mười).


\shortans[4]{0,1}

\loigiai{ 
 Ta có: $\mathop{\lim}\limits_{x \to  -3}f(x)=\frac{4}{3}$.

$f(-3)=3 - 15 m$.

Hàm số liên tục tại $x=-3$ khi $3 - 15 m=\frac{4}{3}$.

Suy ra $m=\frac{1}{9}=0,1$. 
 }\end{ex}

\Closesolutionfile{ans}

 \begin{center}
-----HẾT-----
\end{center}

 %\newpage 
%\begin{center}
%{\bf BẢNG ĐÁP ÁN MÃ ĐỀ 1 }
%\end{center}
%{\bf Phần 3 }
% \inputansbox{6}{ans001-3}
\newpage 




\end{document}