\documentclass[12pt,a4paper]{article}
\usepackage[top=1.5cm, bottom=1.5cm, left=2.0cm, right=1.5cm] {geometry}
\usepackage{amsmath,amssymb,txfonts}
\usepackage{tkz-euclide}
\usepackage{setspace}
\usepackage{lastpage}

\usepackage{tikz,tkz-tab}
%\usepackage[solcolor]{ex_test}
\usepackage[dethi]{ex_test} % Chỉ hiển thị đề thi
%\usepackage[loigiai]{ex_test} % Hiển thị lời giải
%\usepackage[color]{ex_test} % Khoanh các đáp án
\usetikzlibrary{shapes.geometric,arrows,calc,intersections,angles,quotes,patterns,snakes,positioning}
\everymath{\displaystyle}

\def\colorEX{\color{purple}}
%\def\colorEX{}%Không tô màu đáp án đúng trong tùy chọn loigiai
\renewtheorem{ex}{\color{violet}Câu}
\renewcommand{\FalseEX}{\stepcounter{dapan}{{\bf \textcolor{blue}{\Alph{dapan}.}}}}
\renewcommand{\TrueEX}{\stepcounter{dapan}{{\bf \textcolor{blue}{\Alph{dapan}.}}}}

%---------- Khai báo viết tắt, in đáp án
\newcommand{\hoac}[1]{ %hệ hoặc
    \left[\begin{aligned}#1\end{aligned}\right.}
\newcommand{\heva}[1]{ %hệ và
    \left\{\begin{aligned}#1\end{aligned}\right.}

%Tiêu đề
\newcommand{\tenso}{}
\newcommand{\tentruong}{}
\newcommand{\tenkythi}{ĐỀ ÔN TẬP}
\newcommand{\tenmonthi}{Môn học: }
\newcommand{\thoigian}{}
\newcommand{\tieude}[1]{
    \noindent
     \begin{minipage}[b]{6cm}
    \centerline{\textbf{\fontsize{11}{0}\selectfont \tenso}}
    \centerline{\fontsize{11}{0}\selectfont \tentruong}  
  \end{minipage}\hspace{1cm}
  \begin{minipage}[b]{11cm}
    \centerline{\textbf{\fontsize{11}{0}\selectfont \tenkythi}}
    \centerline{\textbf{\fontsize{11}{0}\selectfont \tenmonthi}}
    \centerline{\textit{\fontsize{11}{0}\selectfont Thời \underline{gian làm bài: \thoigian  } phút }}
  \end{minipage}
  \vspace*{3mm}
  \noindent
  \begin{minipage}[t]{12cm}
    \textbf{Họ, tên thí sinh:}\dotfill\\
    \textbf{Số báo danh:}\dotfill
  \end{minipage}\hfill
  \begin{minipage}[b]{3cm}
    \setlength\fboxrule{1pt}
    \setlength\fboxsep{3pt}
    \vspace*{3mm}\fbox{\bf Mã đề thi #1}
  \end{minipage}\\
}

\newcommand{\chantrang}[2]{\rfoot{Trang \thepage $-$ Mã đề #2}}
\pagestyle{fancy}
\fancyhf{}
\renewcommand{\headrulewidth}{0pt} 
\renewcommand{\footrulewidth}{0pt}

\begin{document}
%Thiết lập giãn dọng 1.5cm 
%\setlength{\lineskip}{1.5em}



%Nội dung trắc nghiệm bắt đầu ở đây


\tieude{001}
\chantrang{\pageref{LastPage}}{001}
\setcounter{page}{1}
{\bf PHẦN I. Câu trắc nghiệm nhiều phương án lựa chọn.}
\setcounter{ex}{0}
\Opensolutionfile{ans}[ans/ans001-1]
\begin{ex}
 Cho hai tập hợp $A=\{{1, -5, 4, -4}\}$ và $B=\{{2, -8, -4, -3, -1}\}$. Tìm tập hợp $A\cap B$. \\ 
\choice
{ \True $\{{-4}\}$ }
   { $\{{1, 2, 4, -8, -5, -4, -3, -1}\}$ }
     { $\{{1, -5, 4}\}$ }
    { $\{{-8, 2, -3, -1}\}$ }
\loigiai{ 
  
 }\end{ex}

\begin{ex}
 Cho hai tập hợp $A=\{{1, 3, -5, -4, -2}\}$ và $B=\{{0, 3, 4, -8, -7, -5}\}$. Tìm tập hợp $A\cup B$. \\ 
\choice
{ $\{{1, -4, -2}\}$ }
   { $\{{3, -5}\}$ }
     { \True $\{{0, 1, 3, 4, -8, -7, -5, -4, -2}\}$ }
    { $\{{0, -8, 4, -7}\}$ }
\loigiai{ 
  
 }\end{ex}

\begin{ex}
 Cho hai tập hợp $A=\{{1, 4, -5, -4, -1}\}$ và $B=\{{1, 2, 3, 4, -8, -4}\}$. Tìm tập hợp $A\backslash B$. \\ 
\choice
{ $\{{1, 2, 3, 4, -8, -5, -4, -1}\}$ }
   { $\{{1, 4, -4}\}$ }
     { $\{{-8, 2, 3}\}$ }
    { \True $\{{-5, -1}\}$ }
\loigiai{ 
  
 }\end{ex}

\begin{ex}
 Cho hai tập hợp $E=\{{1, 3, -9, -5, -4, -2}\}$ và $F=\{{-5, -4}\}$. Tìm tập hợp $C_E F$.\\ 
\choice
{ \True $\{{1, 3, -2, -9}\}$ }
   { $\emptyset$ }
     { $\{{-5, -4}\}$ }
    { $\{{1, 3, -9, -5, -4, -2}\}$ }
\loigiai{ 
 Do $F \subset E$ nên $C_E F=E\backslash F=\{{1, 3, -2, -9}\}$ 
 }\end{ex}

\begin{ex}
 Cho hai tập hợp $C=\{{0, -7, -1, 7}\}$ và $D=\{{0, -7, -1}\}$. Tìm tập hợp $C_C D$.\\ 
\choice
{ $\emptyset$ }
   { $\{{0, 7, -7, -1}\}$ }
     { \True $\{{7}\}$ }
    { $\{{0, -7, -1}\}$ }
\loigiai{ 
 Do $D \subset C$ nên $C_C D=C\backslash D=\{{7}\}$ 
 }\end{ex}

\begin{ex}
 Cho hai tập hợp $P=\{x\in \mathbb{Z}|1 < x \le 5\}$ và $Q=\{x\in \mathbb{Z}|-6 \le x < 0\}$. Tìm tập hợp $P \cap Q$.\\ 
\choice
{ $\{{2, 3, 4, 5}\}$ }
   { $\{{-2, -6, -5, -4, -3, -1}\}$ }
     { \True $\emptyset$ }
    { $\{{2, 3, 4, 5, -2, -6, -5, -4, -3, -1}\}$ }
\loigiai{ 
 Ta có: $P=\{{2, 3, 4, 5}\}$ và $Q=\{{-1, -6, -5, -4, -3, -2}\}$. Do đó $P\cap Q =\emptyset$. 
 }\end{ex}

\begin{ex}
 Cho hai tập hợp $E=\{x\in \mathbb{Z}|1 < x \le 7\}$ và $F=\{x\in \mathbb{Z}|-1 \le x < 2\}$. Tìm tập hợp $E \cup F$.\\ 
\choice
{ $\emptyset$ }
   { $\{{0, 1, -1}\}$ }
     { $\{{2, 3, 4, 5, 6, 7}\}$ }
    { \True $\{{0, 1, 2, 3, 4, 5, 6, 7, -1}\}$ }
\loigiai{ 
 Ta có: $E=\{{2, 3, 4, 5, 6, 7}\}$ và $F=\{{0, 1, -1}\}$. Do đó $E\cap F =\{{0, 1, 2, 3, 4, 5, 6, 7, -1}\}$. 
 }\end{ex}

\begin{ex}
 Cho hai tập hợp $P=\{x\in \mathbb{Z}|-6 \le x < -3\}$ và $Q=\{x\in \mathbb{Z}|1 < x \le 5\}$. Tìm tập hợp $P \backslash Q$.\\ 
\choice
{ $\{{2, 3, 4, 5}\}$ }
   { $\{{2, 3, 4, 5, -6, -5, -4}\}$ }
     { $\emptyset$ }
    { \True $\{{-6, -5, -4}\}$ }
\loigiai{ 
 Ta có: $P=\{{-6, -5, -4}\}$ và $Q=\{{2, 3, 4, 5}\}$. Do đó $P \backslash Q =\{{-6, -5, -4}\}$. 
 }\end{ex}

\begin{ex}
 Cho tập hợp $A=\{{1, 14}\}$. Tìm số tập hợp con của tập hợp ${A}$. \\ 
\choice
{ \True ${4}$ }
   { ${3}$ }
     { ${5}$ }
    { ${8}$ }
\loigiai{ 
  
 }\end{ex}

\begin{ex}
 Cho tập hợp $A=\{{-8, -7, -4, 0}\}$. Tìm số tập hợp con gồm 3 phần tử của tập hợp ${A}$. \\ 
\choice
{ ${8}$ }
   { ${16}$ }
     { ${7}$ }
    { \True ${4}$ }
\loigiai{ 
  
 }\end{ex}

\Closesolutionfile{ans}
{\bf PHẦN II. Câu trắc nghiệm đúng sai.}
\setcounter{ex}{0}
\Opensolutionfile{ans}[ans/ans001-2]
\begin{ex}
 Cho hai tập hợp $A=\{{-5, -4, -3, -2}\}$ và $B=\{{-5, -4, -3}\}$. Xét tính đúng-sai của các khẳng định sau. 
\choiceTFt
{ Số phần tử của $A \cup B$ là ${7}$  }
   { \True $A \backslash B = \{{-2}\}$ }
     { Tập hợp ${A}$ là tập hợp con của tập hợp ${B}$ }
    { \True $A\cap B= \{{-5, -4, -3}\}$ }
\loigiai{ 
 

 a) Khẳng định đã cho là khẳng định sai.

 $A\cup B=\{{-5, -4, -3, -2}\}$. Nên số phần tử của $A\cup B$ là ${4}$.

b) Khẳng định đã cho là khẳng định đúng.

 $A \backslash B = \{{-2}\}$.

c) Khẳng định đã cho là khẳng định sai.

 Ta thấy mọi phần tử của ${A}$ không nằm trong ${B}$ nên $A \not\subset B$.

d) Khẳng định đã cho là khẳng định đúng.

 $A\cap B= \{{-5, -4, -3}\}$.

 
 }\end{ex}

\begin{ex}
 Cho hai tập hợp $A=\{x\in \mathbb{Z}|-6 \le x \le -4\}$ và $B=\{x\in \mathbb{Z}|-4 \le x \le 1\}$. Xét tính đúng-sai của các khẳng định sau. 
\choiceTFt
{ \True Số phần tử của $A \cup B$ là ${8}$ }
   { Tập hợp ${A}$ là tập hợp con của tập hợp ${B}$ }
     { $A\cap B= \emptyset$ }
    { $A \backslash B = \{{0, 1, -2, -3, -1}\}$  }
\loigiai{ 
 

 a) Khẳng định đã cho là khẳng định đúng.

 $A=\{{-6, -5, -4}\},B=\{{0, 1, -1, -4, -3, -2}\}\Rightarrow A\cup B=\{{0, 1, -2, -6, -5, -4, -3, -1}\}$.

 Nên số phần tử của $A\cup B$ là ${8}$.

b) Khẳng định đã cho là khẳng định sai.

 $A=\{{-6, -5, -4}\},B=\{{0, 1, -1, -4, -3, -2}\}$.

 Ta thấy mọi phần tử của ${A}$ không nằm trong ${B}$ nên $A \not\subset B$.

c) Khẳng định đã cho là khẳng định sai.

 $A=\{{-6, -5, -4}\},B=\{{0, 1, -1, -4, -3, -2}\}\Rightarrow A\cap B= \{{-4}\}$.

d) Khẳng định đã cho là khẳng định sai.

 $A=\{{-6, -5, -4}\},B=\{{0, 1, -1, -4, -3, -2}\}\Rightarrow A \backslash B = \{{-6, -5}\}$.

 
 }\end{ex}

\begin{ex}
 Cho hai tập hợp $P=\{x\in \mathbb{Z}|2 \le x \le 5\}$ và $Q=\{{3, 4, 5, 6, 7}\}$. Xét tính đúng-sai của các khẳng định sau:
\choiceTFt
{ $P= \{ {2, 3, 4}\}$  }
   { \True  Số tập hợp con của tập hợp ${P}$ là ${16}$ }
     { $P\cap Q=\{{2, 3, 4, 5, 6, 7}\}$ }
    { Số tập hợp ${X}$ để $\{{3, 5, 7}\} \subset X \subset \{{3, 4, 5, 7, 8, -2}\}$ là ${7}$ }
\loigiai{ 
 

 a) Khẳng định đã cho là khẳng định sai.

 $P= \{ {2, 3, 4, 5}\}$.

b) Khẳng định đã cho là khẳng định đúng.

 Số tập hợp con của tập hợp ${P}$ là $2^{4}={16}.$

c) Khẳng định đã cho là khẳng định sai.

 $P\cap Q=\{{3, 4, 5}\}$.

d) Khẳng định đã cho là khẳng định sai.

 Các tập hợp ${X}$ để $\{{3, 5, 7}\} \subset X \subset \{{3, 4, 5, 7, 8, -2}\}$ là ${8}$ là:

$\{3, 5, 7\}$

$\{3, 4, 5, 7\}$

$\{3, 5, 7, 8\}$

$\{3, 5, 7, -2\}$

$\{3, 4, 5, 7, 8\}$

$\{3, 4, 5, 7, -2\}$

$\{3, 5, 7, 8, -2\}$

$\{3, 4, 5, 7, 8, -2\}$



 
 }\end{ex}

\begin{ex}
 Biết rằng lớp 11B5 có ${19}$ bạn thích môn Toán và ${22}$ bạn thích môn Văn. Trong số các bạn thích môn Toán hoặc thích môn Văn có ${9}$ bạn thích cả hai môn. Lớp vẫn còn ${6}$ bạn không thích môn Toán và không thích môn Văn. Xét tính đúng-sai của các khẳng định sau.
\choiceTFt
{ \True Số học sinh chỉ thích môn Toán là ${10}$ }
   { \True Số học sinh chỉ thích môn Văn là ${13}$ }
     { Số học sinh thích môn Toán hoặc thích môn Văn là ${35}$ }
    { Tổng số học sinh của lớp 11B5 là ${42}$ }
\loigiai{ 
 

 a) Khẳng định đã cho là khẳng định đúng.

 Số học sinh chỉ thích môn Toán là: $19-9=10$.

b) Khẳng định đã cho là khẳng định đúng.

 Số học sinh chỉ thích môn Văn là: $22-9=13$.

c) Khẳng định đã cho là khẳng định sai.

 Số học sinh thích môn Toán hoặc thích môn Văn là: $19+22-9=32$.

d) Khẳng định đã cho là khẳng định sai.

 Số học sinh thích môn Toán hoặc thích môn Văn là: $19+22-9=32$.

Tổng số học sinh của lớp là: $32+6=38$.

 
 }\end{ex}

\Closesolutionfile{ans}
{\bf PHẦN III. Câu trắc nghiệm trả lời ngắn.}
\setcounter{ex}{0}
\Opensolutionfile{ans}[ans/ans001-3]
\begin{ex}
 Lớp 11A5 có tổng cộng ${41}$ học sinh, các học sinh này đều thích nấu ăn hoặc thích đọc sách. Có ${25}$ học sinh thích nấu ăn (trong số này có các học sinh thích đọc sách) và ${24}$ học sinh thích đọc sách (trong số này có các học sinh thích nấu ăn). Hỏi lớp 11A5 có bao nhiêu học sinh thích nấu ăn và thích đọc sách.
\shortans[oly]{${8}$}

\loigiai{ 
 Số học sinh chỉ thích nấu ăn và không thích đọc sách là: $41-24=17$.

Số học sinh thích nấu ăn và thích đọc sách là: $25-17=8$. 
 }\end{ex}

\begin{ex}
 Mỗi học sinh của lớp 10B5 đều thích nghe nhạc hoặc thích xem phim. Biết rằng lớp có ${24}$ bạn thích nghe nhạc (trong số này có các bạn thích xem phim), có ${21}$ bạn thích xem phim (trong số này có các bạn thích nghe nhạc) và có ${8}$ bạn thích nghe nhạc và thích xem phim. Hỏi lớp 10B5 có tổng cộng bao nhiêu học sinh?
\shortans[oly]{${37}$}

\loigiai{ 
 Gọi ${A}$ là tập hợp các bạn thích nghe nhạc, ta có: $n(A)=24$.

Gọi ${B}$ là tập hợp các bạn thích xem phim, ta có: $n(B)=21$.

$A\cap B$ là tập hợp các bạn thích nghe nhạc và thích xem phim, ta có: $n(A\cap B)=8$.

$C=A\cup B$ là tập hợp tất cả các bạn của lớp 10B5, ta có: $n(C)=24+21-8=37$. 
 }\end{ex}

\begin{ex}
 Biết rằng lớp 11B2 có ${17}$ bạn thích các môn Tự nhiên và ${20}$ bạn thích các môn Xã hội. Trong số các bạn thích các môn Tự nhiên hoặc thích các môn Xã hội có ${8}$ bạn thích cả hai nhóm môn. Lớp vẫn còn ${7}$ bạn không thích các môn Tự nhiên và không thích các môn Xã hội. Hỏi lớp 11B2 có tổng cộng bao nhiêu học sinh?
\shortans[oly]{${36}$}

\loigiai{ 
 Gọi ${A}$ là tập hợp các bạn thích các môn Tự nhiên, ta có: $n(A)=17$.

Gọi ${B}$ là tập hợp các bạn thích các môn Xã hội, ta có: $n(B)=20$.

$A\cap B$ là tập hợp các bạn thích các môn Tự nhiên và thích các môn Xã hội, ta có: $n(A\cap B)=8$.

$A\cup B$ là tập hợp tất cả các bạn thích các môn Tự nhiên hoặc thích các môn Xã hội, ta có: $n(A\cup B)=17+20-8=29$.

Tổng số học sinh của lớp là: $29+7=36$. 
 }\end{ex}

\begin{ex}
 Lớp 10B1 có ${20}$ bạn biết chơi cầu lông, ${18}$ bạn biết chơi bóng chuyền, ${17}$ bạn biết chơi cờ vua, ${6}$ bạn biết chơi cầu lông và biết chơi bóng chuyền, ${5}$ bạn biết chơi bóng chuyền và biết chơi cờ vua, ${8}$ bạn biết chơi cầu lông và biết chơi cờ vua và ${4}$ biết chơi cả ba môn. Hỏi lớp 10B1 có tất cả bao nhiêu học sinh biết ít nhất một môn thể thao?
\shortans[oly]{40}

\loigiai{ 
 Gọi A là tập hợp các bạn biết chơi cầu lông, B là tập hợp các bạn biết chơi bóng chuyền, C là tập hợp các bạn biết chơi cờ vua.

Ta có: $n(A)=20, n(B)=18, n(C)=17$.

$n(A\cap B)=6, n(A\cap C)=8, n(B\cap C)=5$.

$n(A\cap B \cap C)=4$.

Số học sinh biết ít nhất một môn thể thao là:

$n(A\cup B \cup C)=n(A)+n(B)+n(C)+n(A\cap B \cap C)-n(A\cap B)-n(A\cap C)-n(B\cap C)$

$=20+18+17+4-6-8-5=40$.

Đáp án: 40 
 }\end{ex}

\begin{ex}
 Lớp 10A1 có ${19}$ bạn thích xem phim rạp, ${18}$ bạn thích đi du lịch, ${17}$ bạn thích đọc truyện, ${5}$ bạn thích xem phim rạp và thích đi du lịch, ${6}$ bạn thích đi du lịch và thích đọc truyện, ${7}$ bạn thích xem phim rạp và thích đọc truyện, ${3}$ có cả ba sở thích. Hỏi lớp 10A1 có tất cả bao nhiêu học sinh chỉ có đúng một sở thích trong các sở thích trên?
\shortans[oly]{27}

\loigiai{ 
 Gọi A là tập hợp các bạn thích xem phim rạp, B là tập hợp các bạn thích đi du lịch, C là tập hợp các bạn thích đọc truyện.

Ta có: $n(A)=19, n(B)=18, n(C)=17$.

$n(A\cap B)=5, n(A\cap C)=7, n(B\cap C)=6$.

$n(A\cap B \cap C)=3$.

Số học sinh chỉ thích xem phim rạp là: $19-5-7+3=10$

Số học sinh chỉ thích đi du lịch là: $18-5-6+3=10$

Số học sinh chỉ thích đọc truyện là: $17-7-6+3=7$

Số học sinh chỉ có đúng một sở thích: $10+10+7=27$.

Đáp án: 27 
 }\end{ex}

\Closesolutionfile{ans}

 \begin{center}
-----HẾT-----
\end{center}

 %\newpage 
%\begin{center}
%{\bf BẢNG ĐÁP ÁN MÃ ĐỀ 1 }
%\end{center}
%{\bf Phần 1 }
% \inputansbox{6}{ans001-1}
%{\bf Phần 2 }
% \inputansbox{2}{ans001-2}
%{\bf Phần 3 }
% \inputansbox{6}{ans001-3}
\newpage 




\end{document}