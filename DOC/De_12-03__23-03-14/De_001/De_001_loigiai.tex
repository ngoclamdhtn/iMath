\documentclass[12pt,a4paper]{article}
\usepackage[top=1.5cm, bottom=1.5cm, left=2.0cm, right=1.5cm] {geometry}
\usepackage{amsmath,amssymb,fontawesome}
\usepackage{tkz-euclide}
\usepackage{setspace}
\usepackage{lastpage}

\usepackage{tikz,tkz-tab}
%\usepackage[solcolor]{ex_test}
%\usepackage[dethi]{ex_test} % Chỉ hiển thị đề thi
\usepackage[loigiai]{ex_test} % Hiển thị lời giải
%\usepackage[color]{ex_test} % Khoanh các đáp án
\everymath{\displaystyle}

\def\colorEX{\color{purple}}
%\def\colorEX{}%Không tô màu đáp án đúng trong tùy chọn loigiai
\renewtheorem{ex}{\color{violet}Câu}
\renewcommand{\FalseEX}{\stepcounter{dapan}{{\bf \textcolor{blue}{\Alph{dapan}.}}}}
\renewcommand{\TrueEX}{\stepcounter{dapan}{{\bf \textcolor{blue}{\Alph{dapan}.}}}}

%---------- Khai báo viết tắt, in đáp án
\newcommand{\hoac}[1]{ %hệ hoặc
    \left[\begin{aligned}#1\end{aligned}\right.}
\newcommand{\heva}[1]{ %hệ và
    \left\{\begin{aligned}#1\end{aligned}\right.}

%Tiêu đề
\newcommand{\tenso}{iMath}
\newcommand{\tentruong}{}
\newcommand{\tenkythi}{ĐỀ ÔN TẬP}
\newcommand{\tenmonthi}{Môn thi: TOÁN 12}
\newcommand{\thoigian}{}
\newcommand{\tieude}[1]{
   \begin{tabular}{cm{3cm}cm{3cm}cm{3cm}}
    {\bf \tenso} & & {\bf \tenkythi} \\
    {\bf \tentruong} & & {\bf \tenmonthi}\\
    && {\bf Thời gian: \bf \thoigian \, phút}\\
    && { \fbox{\bf Mã đề: #1}}
   \end{tabular}\\\\
    
   {Họ tên HS: \dotfill Số báo danh \dotfill}\\
}
\newcommand{\chantrang}[2]{\rfoot{Trang \thepage $-$ Mã đề #2}}
\pagestyle{fancy}
\fancyhf{}
\renewcommand{\headrulewidth}{0pt} 
\renewcommand{\footrulewidth}{0pt}
\usetikzlibrary{shapes.geometric,arrows,calc,intersections,angles,quotes,patterns,snakes,positioning}

\begin{document}
%Thiết lập giãn dọng 1.5cm 
%\setlength{\lineskip}{1.5em}
%Nội dung trắc nghiệm bắt đầu ở đây


\tieude{001}
\chantrang{\pageref{LastPage}}{001}
\setcounter{page}{1}
{\bf PHẦN I. Câu trắc nghiệm nhiều phương án lựa chọn.}
\setcounter{ex}{0}
\Opensolutionfile{ans}[ans/ans001-1]
\begin{ex}
 Cho các số thực ${a,c,\alpha,\gamma}$ thỏa mãn $a>0,c>0$. Tìm khẳng định đúng trong các khẳng định sau 
\choice
{ $\left(a - c\right)^{\alpha}=a^{\alpha}-c^{\alpha}$ }
   { \True $(a^{\alpha})^{\gamma}=a^{\alpha \gamma}$ }
     { $a^{\alpha + \gamma}=a^{\alpha}+a^{\gamma}$ }
    { $\dfrac{1}{a^{\alpha}}=\dfrac{1}{\alpha a}$ }
\loigiai{ 
 $(a^{\alpha})^{\gamma}=a^{\alpha \gamma}$ là khẳng định đúng. 
 }\end{ex}

\begin{ex}
 Cho biểu thức $P=\sqrt[5] {x^{20}}$ với $x>0$. Khẳng định nào sau đây đúng?
 
\choice
{ $P={x^{100}}$ }
   { $P={x^{\frac{1}{4}}}$ }
     { $P={x^{25}}$ }
    { \True $P={x^{4}}$ }
\loigiai{ 

 $P=\sqrt[5] {x^{20}} = {x}^{\tfrac{20}{5}}$. Vậy $P=x^{4}$. 
 }\end{ex}

\begin{ex}
 Cho biểu thức $P=\sqrt[5] {a.\sqrt[4]{ a^3.\sqrt[k]{a^{10}} } }$ với $a>0$. Tìm ${k}$ để $P=a^{\tfrac{9}{20}}$.
 
\choice
{ \True ${k= 5}$ }
   { ${k= 6}$ }
     { ${k= 10}$ }
    { ${k= 12}$ }
\loigiai{ 

 $P=\sqrt[5] {a.\sqrt[4]{ a^3.\sqrt[k]{a^{10}} } }= a^{\left((3+\frac{10}{k})\frac 1 4+1\right).\frac 1 5}$.
 
 }\end{ex}

\begin{ex}
  Cho $a$ là số thực dương khác 1. Mệnh đề nào dưới đây đúng?
 
\choice
{ $\log_a a^{15}=-\dfrac{1}{15}$ }
   { \True $\log_a a^{15}=15$ }
     { $\log_a a^{15}=\dfrac{1}{15}$ }
    { $\log_a a^{15}=-15$ }
\loigiai{ 

 Theo công thức logarit ta có: $\log_a a^{15}=15$. 
 }\end{ex}

\begin{ex}
  Cho $a$ là số thực dương khác 1. Mệnh đề nào dưới đây đúng?
 
\choice
{ $\log_{\sqrt[{}]{a}} \left(\dfrac{1}{ a^{10} }\right)=\dfrac{1}{20}$ }
   { \True $\log_{\sqrt[{}]{a}} \left(\dfrac{1}{ a^{10} }\right)=-20$ }
     { $\log_{\sqrt[{}]{a}} \left(\dfrac{1}{ a^{10} }\right)=20$ }
    { $\log_{\sqrt[{}]{a}} \left(\dfrac{1}{ a^{10} }\right)=-\dfrac{1}{20}$ }
\loigiai{ 

 Ta có: $\log_{\sqrt[{}]{a}} \left(\dfrac{1}{ a^{10} }\right)=\log_{a^{ \frac{1}{2} } } a^{-10}=-10.2\log_a a=-20$. 
 }\end{ex}

\begin{ex}
 Cho $a=\log_2 3, b=\log_2 5$. Hãy biểu diễn $\log_{108}{640}$ theo $a$ và $b$.
 
\choice
{ $P={\dfrac{b -7}{3a - 4}}$ }
   { \True $P={\dfrac{b + 7}{3a +2}}$ }
     { $P={\dfrac{7b}{3a +5}}$ }
    { $P={\dfrac{7b +2}{3a}}$ }
\loigiai{ 

 $P=\log_{108}{640}=\dfrac{ \log_2 (5.2^{7}) } { \log_2 (3^{3}.2^{2}) }
= \dfrac{\log_2 5 + \log_2 2^{7} }{ \log_2 3^{3}+ \log_2 2^{2} } =\dfrac{b + 7}{3a +2}$. 
 }\end{ex}

\begin{ex}
 Cho $\log 3=a$. Biểu diễn $\log 900000$ theo $a$ ta được
 
\choice
{ \True ${2 a + 5}$ }
   { ${4 - 15 a}$ }
     { ${2 a - 5}$ }
    { ${10 a}$ }
\loigiai{ 

 $\log 900000=\log \left(3^2.10^5\right)=\log 3^2 +\log 10^5=2\log 3 +5=2 a + 5$. 
 }\end{ex}

\begin{ex}
 Đồ thị như hình vẽ dưới đây là của hàm số nào trong các hàm số sau?
\begin{center}
\begin{tikzpicture}[>=stealth] 
            \draw[->] (-2.5,0) -- (2.5,0) node[below] {$x$}; 
            \draw[->] (0,-1) -- (0,4) node[left] {$y$}; 
            \draw[fill=black] (0,0) node[below right]{$O$} circle (1pt);
            \draw (0,1) node[above left]{$1$}; 
            \draw (1,0) node[below]{$1$}; 
            \draw (0,3) node[left]{\footnotesize$3$}; 
            \tkzDefPoints{1/3/A} 
            \tkzDrawPoints[fill=black](A) 
            \draw [dashed] (0,3)--(1,3)--(1,0); 
            \begin{scope}
             \clip (-2,-0.5) rectangle (2,4);
            \draw[color=blue,very thick,smooth,samples=200, domain=-2:2.01] plot (\x,{(3)^(\x)}); 
            \end{scope}
            \end{tikzpicture} 

\end{center}

\choice
{ $y=\left(\frac{1}{3}\right)^x$ }
   { $y=\sqrt{3}^x$ }
     { $y=x^3$ }
    { \True $y=3^x$ }
\loigiai{ 

 Đồ thị có hình dáng đồ thị của hàm số mũ, đi qua điểm $(1,3)$ nên đây là đồ thị hàm số $y=3^x$. 
 }\end{ex}

\begin{ex}
 Đồ thị như hình vẽ dưới đây là của hàm số nào trong các hàm số sau?
\begin{center}
\begin{tikzpicture}[>=stealth] 
            \draw[->] (-1,0) -- (7.5,0) node[below] {$x$}; 
            \draw[->] (0,-1.5) -- (0,2.5) node[left] {$y$}; 
            \draw[fill=black] (0,0) node[below right]{$O$} circle (1pt);
            \draw (0,1) node[left]{$1$}; 
            \draw (1,0) node[below]{$1$}; 
            \draw (7,0) node[below]{$7$}; 
            \draw[fill=black] (7, 1) circle (1pt);
            \draw [dashed] (0,1)--(7,1)--(7,0); 
            \begin{scope}
             \clip (-1,-1.5) rectangle (7.5,2.5);
            \draw[color=blue,thick,smooth, samples = 200, domain=0.001:7.5] plot (\x,{(ln(\x)/ln(7)}); 
            \end{scope}
            \end{tikzpicture} 

\end{center}

\choice
{ $y=\log_{\frac{1}{7}} x$ }
   { $y=\sqrt{7}^x$ }
     { $y=7^x$ }
    { \True $y=\log_7 x$ }
\loigiai{ 

 Đồ thị có hình dáng đồ thị của hàm số logarit, đi qua điểm $(7,1)$ nên đây là đồ thị hàm số $y=\log_7 x$. 
 }\end{ex}

\begin{ex}
 Đồ thị như hình vẽ dưới đây là của hàm số nào trong các hàm số sau?
\begin{center}
\begin{tikzpicture}[>=stealth] 
            \draw[->] (-2.5,0) -- (1.8,0) node[below] {$x$}; 
            \draw[->] (0,-1) -- (0,4.317) node[left] {$y$}; 
            \draw[fill=black] (0,0) node[below right]{$O$} circle (1pt);
            \draw (0,1) node[above left]{$1$}; 
            \draw (1,0) node[below]{$1$}; 
            \draw (0,3.317) node[left]{\footnotesize $\sqrt{11}$}; 
            \draw[fill=black] (1,3.317) circle (1pt);
            \draw [dashed] (0,3.317)--(1,3.317)--(1,0); 
            \begin{scope}
             \clip (-2,-0.5) rectangle (2,12);
            \draw[color=blue,very thick,smooth,samples=200, domain=-2:1.31] plot (\x,{(3.317)^(\x)}); 
            \end{scope}
            \end{tikzpicture} 

\end{center}

\choice
{ $y=x^{11}$ }
   { $y={11}^x$ }
     { $y=\left( \frac{1}{11} \right)^x$ }
    { \True $y=\sqrt{11}^x$ }
\loigiai{ 

 Đồ thị có hình dáng đồ thị của hàm số mũ, đi qua điểm $(1,\sqrt{11})$ nên đây là đồ thị hàm số $y=\sqrt{11}^x$. 
 }\end{ex}

\begin{ex}
 Cho hàm số $y=\left(\dfrac{11}{17}\right)^x$. Tìm khẳng định đúng trong các khẳng định sau.
\choice
{ Hàm số đồng biến trên khoảng $(-\infty; 6)$ }
   { Hàm số chỉ nghịch biến trên khoảng $(0; +\infty)$ }
     { \True Hàm số nghịch biến trên ${ \mathbb{R} }$ }
    { Hàm số đồng biến trên ${\mathbb{R} }$ }
\loigiai{ 

 Hàm số $y=\left(\dfrac{11}{17}\right)^x$ có cơ số $0<\dfrac{11}{17}<1$ nên hàm số nghịch biến trên ${\mathbb{R}}$. 
 }\end{ex}

\begin{ex}
 Cho hàm số $y=\log_{e} x$. Tìm khẳng định đúng trong các khẳng định sau.
\choice
{ \True Hàm số đồng biến trên $(0; +\infty)$ }
   { Hàm số đồng biến trên khoảng $(-1;+\infty)$ }
     { Hàm số nghịch biến trên ${\mathbb{R} }$ }
    { Hàm số nghịch biến trên khoảng $(-1;+\infty)$ }
\loigiai{ 

 Hàm số $y=\log_{e} x$ có cơ số $e>1$ nên hàm số đồng biến trên $(0; +\infty)$. 
 }\end{ex}

\begin{ex}
 Một người gửi tiết kiệm ngân hàng số tiền 223 triệu đồng theo hình thức lãi suất kép với lãi suất 0,44\%/tháng. Tính tổng tiền cả vốn lẫn lãi người đó nhận được sau 5 tháng.
 
\choice
{ 225,96 triệu đồng }
   { 228,95 triệu đồng }
     { \True 227,95 triệu đồng }
    { 226,95 triệu đồng }
\loigiai{ 

 Tổng tiền cả vốn lẫn lãi người đó nhận được sau 5 tháng:

$S=223\left(1+0,44\%\right)^{5}=227,95$ (triệu đồng). 
 }\end{ex}

\begin{ex}
 Một người gửi tiết kiệm ngân hàng số tiền 108 triệu đồng theo hình thức lãi suất kép với lãi suất 6\%/năm. Tính thời gian tối thiểu để người đó nhận được số tiền là 183 triệu đồng.
 
\choice
{ 9 năm }
   { 8 năm }
     { \True 10 năm }
    { 11 năm }
\loigiai{ 

 Tổng tiền cả vốn lẫn lãi người đó nhận được sau 9 năm:

$S=108\left(1+6\%\right)^{n}=183\Leftrightarrow n=10$ (năm). 
 }\end{ex}

\begin{ex}
 Biết rằng vào ngày 01/3/2022, dân số của thành phố Y có khoảng 2,28 (triệu người). Nếu tỉ lệ tăng dân số của thành phố Y là 1,3\%/năm và giữ ổn định qua các năm thì vào ngày 01/3/2027, dân số của thành phố Y là
 
\choice
{ \True 2,43 (triệu người) }
   { 2,37 (triệu người) }
     { 2,46 (triệu người) }
    { 2,40 (triệu người) }
\loigiai{ 

 Dân số của thành phố Y sau 5 năm là:

$S=2,28\left(1+1,3\%\right)^{5}=2,43$ (triệu người). 
 }\end{ex}

\begin{ex}
 Một hộp chứa 21 quả cầu cùng kích thước được đánh số từ 1 đến 21. Chọn ngẫu nhiên 1 quả cầu từ hộp. Gọi ${A}$ là biến cố “Số ghi trên quả cầu được chọn là một số lẻ”, $B$ là biến cố “ Số ghi trên quả cầu được chọn là số chia hết cho 4”. Xác định số phần tử của biến cố ${AB}$.
 
\choice
{ \True ${0}$ }
   { ${1}$ }
     { ${5}$ }
    { ${4}$ }
\loigiai{ 

 ${AB}$ là biến cố số ghi trên quả cầu được chọn vừa là một số lẻ vừa là số chia hết cho 4.
${AB}=\{\}$.
 Vậy số phần tử của biến cố ${AB}$ là ${0}$. 
 }\end{ex}

\begin{ex}
 Một hộp chứa 25 quả cầu cùng kích thước được đánh số từ 1 đến 25. Chọn ngẫu nhiên 1 quả cầu từ hộp. Gọi ${A}$ là biến cố “Số ghi trên quả cầu được chọn là một số thuộc đoạn ${[1;23]}$”, $B$ là biến cố “ Số ghi trên quả cầu được chọn là số chia hết cho 5”. Xác định số phần tử của biến cố ${AB}$.
 
\choice
{ ${5}$ }
   { \True ${4}$ }
     { ${9}$ }
    { ${8}$ }
\loigiai{ 

 ${AB}$ là biến cố số ghi trên quả cầu được chọn vừa là một số thuộc đoạn ${[1;23]}$, vừa là số chia hết cho 5.
${AB}=\{5, 10, 15, 20\}$.
 Vậy số phần tử của biến cố ${AB}$ là ${4}$. 
 }\end{ex}

\begin{ex}
 Gieo một con xúc xắc cân đối và đồng chất hai lần. Gọi ${A}$ là biến cố “Tổng số chấm của hai lần gieo lớn hơn 8”, $B$ là biến cố “Tổng số chấm của hai lần gieo nhỏ hơn 10”. Xác định số phần tử của biến cố ${AB}$.
 
\choice
{ ${2}$ }
   { ${7}$ }
     { \True ${4}$ }
    { ${20}$ }
\loigiai{ 

 ${AB}$ là biến cố tổng số chấm của hai lần gieo lớn hơn 8 và nhỏ hơn 10.
${AB}=\{(3;6), (4;5), (5;4), (6;3)\}$. 
 Vậy số phần tử của biến cố ${AB}$ là ${4}$. 
 }\end{ex}

\begin{ex}
 Một hộp chứa 8 viên bi có màu sắc khác nhau. Lấy ngẫu nhiên một viên bi, quan sát màu sắc rồi trả lại vào hộp. Tiếp tục lấy lần 2 rồi trả lại, cứ tiếp tục như thế đến 5 lần. Gọi ${A_i}$ là biến cố "Lần thứ ${i}$ lấy được viên bi màu tím". Mệnh đề nào dưới đây mô tả biến cố $A_1 \cap A_2$?


 
\choice
{ Lần rút đầu tiên lấy được bi màu tím là lần lấy thứ 1 }
   { Cả hai lần lấy thứ ${1}$ và thứ ${2}$ đều không lấy được bi màu tím }
     { \True Cả hai lần lấy thứ ${1}$ và thứ ${2}$ đều lấy được bi màu tím }
    { Ít nhất một trong các lần lấy thứ ${1}$ và thứ ${2}$ đều lấy được bi màu tím }
\loigiai{ 

 Biến cố $A_1 \cap A_2$ là Cả hai lần lấy thứ ${1}$ và thứ ${2}$ đều lấy được bi màu tím. 
 }\end{ex}

\begin{ex}
 Cho ${A}$ và ${B}$ là hai biến cố độc lập. Biết $P(A)=0,69$ và $P(B)=0,59$. Tính xác suất của biến cố ${AB}$ (kết quả làm tròn đến hàng phần nghìn).
 
\choice
{ ${0,183}$ }
   { ${0,283}$ }
     { \True ${0,407}$ }
    { ${0,127}$ }
\loigiai{ 

 $P{AB}=P(A).P(B)=0,69.0,59=0,407$ .
 
 }\end{ex}

\begin{ex}
 Một xạ thủ bắn lần lượt 2 viên đạn vào một bia. Xác suất trúng đích của viên thứ nhất là ${0,27}$ và của viên thứ hai là ${0,16}$. Biết rằng kết quả của các lần bắn là độc lập với nhau. Tính xác suất của biến cố "Cả hai lần bắn đều trúng đích" (kết quả làm tròn đến hàng phần trăm).
 
\choice
{ ${0,12}$ }
   { ${0,23}$ }
     { ${0,61}$ }
    { \True ${0,04}$ }
\loigiai{ 

 Gọi ${A}$ là biến cố viên thứ nhất bắn trúng. Ta có: $P(A)=0,27$.

Gọi ${B}$ là biến cố viên thứ hai bắn trúng. Ta có: $P(B)=0,16$.

Do ${A}$ và ${B}$ là độc lập nên xác suất để cả hai lần bắn đều trúng đích là:

$P(AB)=P(A).P(B)=0,27.0,16=0,04$ .
 
 }\end{ex}

\begin{ex}
 Hai bệnh nhân ${X}$ và ${Y}$ bị nhiễm một loại vi rút. Biết rằng xác suất bị biến chứng nặng của bệnh nhân ${X}$ là ${0,65}$ và của bệnh nhân ${Y}$ là ${0,7}$. Khả năng bị biến chứng nặng của hai bệnh nhân là độc lập. Tính xác suất của biến cố "Cả hai bệnh nhân đều không bị biến chứng nặng" (kết quả làm tròn đến hàng phần trăm).
 
\choice
{ ${0,24}$ }
   { ${0,45}$ }
     { ${0,20}$ }
    { \True ${0,11}$ }
\loigiai{ 

 Gọi ${A}$ là biến cố "Bệnh nhân ${X}$ bị biến chứng nặng". Ta có: $P(A)=0,65$ và $P\left(\overline{A}\right)=0,35$.

Gọi ${B}$ là biến cố "Bệnh nhân ${Y}$ bị biến chứng nặng". Ta có: $P(B)=0,7$ và $P\left(\overline{B}\right)=0,3$.

Do $\overline{A}$ và $\overline{B}$ là độc lập nên xác suất để cả hai bệnh nhân đều không bị biến chứng nặng là:

$P\left(\overline{A}\overline{B}\right)=P(\overline{A}).P(\overline{B})=0,35.0,3=0,11$ .
 
 }\end{ex}

\begin{ex}
 Một hộp chứa 15 viên bi màu hồng và 17 viên bi màu tím. Lấy ngẫu nhiên hai viên bi. Xét các biến cố:

${P}$ : Hai viên bi lấy được có màu hồng.

${Q}$ : Hai viên bi lấy được có màu tím.

Khi đó biến cố hợp của hai biến cố ${P}$ và ${Q}$ là:
 
\choice
{ Hai viên bi lấy ra chỉ có màu màu tím }
   { Hai viên bi lấy ra chỉ có màu màu hồng }
     { \True Hai viên bi lấy ra có cùng màu }
    { Hai viên bi lấy ra có màu khác nhau }
\loigiai{ 

 Biến cố hợp của hai biến cố ${P}$ và ${Q}$  là ${P}$  hoặc ${P}$ xảy ra.

Do đó $P \cup Q$ là biến cố hai viên bi lấy ra có cùng màu. 
 }\end{ex}

\begin{ex}
 Gieo một đồng xu và một con xúc xắc.

Gọi ${A}$ là biến cố: "Đồng xu xuất hiện mặt sấp và xúc xắc xuất hiện mặt chứa số chẵn".

Gọi ${B}$ là biến cố: "Xúc xắc xuất hiện mặt chứa số nguyên tố".

Số phần tử của biến cố $A \cup B$.
 
\choice
{ ${4}$ }
   { ${11}$ }
     { ${13}$ }
    { \True ${8}$ }
\loigiai{ 

 Biến cố ${A}$ có các phần tử là $\{S2, S4, S6\}$.

Biến cố ${B}$ có các phần tử là $\{N2, N3, N5, S2, S3, S5\}$.

Biến cố $A\cup B=\{N2, N3, N5, S2, S3, S4, S5, S6\}$. Số phần tử của $A\cup B$ là $8$. 
 }\end{ex}

\begin{ex}
 Cho ${A}$ và ${B}$ là hai biến cố xung khắc. Biết $P(A)=0,27$ và $P(B)=0,1$. Tính xác suất của biến cố ${A \cup B}$.
 
\choice
{ ${0,03}$ }
   { ${0,24}$ }
     { \True ${0,37}$ }
    { ${0,66}$ }
\loigiai{ 

 $P(A \cup B)=P(A)+P(B)=0,27+0,1=0,37$ .
 
 }\end{ex}

\begin{ex}
 Một lớp học có ${8}$ học sinh nam và ${10}$ học sinh nữ. Chọn ngẫu nhiên ${3}$ học sinh từ lớp học. Tính xác suất của biến cố "Cả ${3}$ học sinh được chọn đều cùng giới tính".
 
\choice
{ ${\dfrac{7}{102}}$ }
   { \True ${\dfrac{11}{51}}$ }
     { ${\dfrac{11}{306}}$ }
    { ${\dfrac{5}{34}}$ }
\loigiai{ 

 Số cách chọn ${3}$ học sinh là: $C^{3}_{18}=816$.

Số cách chọn ${3}$ học sinh từ học sinh nam là: $C^{3}_{8}=56$.

Số cách chọn ${3}$ học sinh từ học sinh nữ là: $C^{3}_{10}=120$.

Xác suất cần tính là: $P=\dfrac{56+120} {816}=\dfrac{11}{51}$. 
 }\end{ex}

\begin{ex}
 Một thư viện có ${15}$ cuốn truyện trinh thám và ${10}$ cuốn truyện cổ tích, các cuốn truyện là khác nhau. Chọn ngẫu nhiên ${4}$ cuốn truyện từ thư viện. Tính xác suất của biến cố "Cả ${4}$ cuốn truyện được chọn đều cùng thể loại truyện".
 
\choice
{ ${\dfrac{21}{1265}}$ }
   { ${\dfrac{21}{4048}}$ }
     { \True ${\dfrac{63}{506}}$ }
    { ${\dfrac{273}{2530}}$ }
\loigiai{ 

 Số cách chọn ${4}$ cuốn truyện là: $C^{4}_{25}=12650$.

Số cách chọn ${4}$ cuốn truyện từ cuốn truyện trinh thám là: $C^{4}_{15}=1365$.

Số cách chọn ${4}$ cuốn truyện từ cuốn truyện cổ tích là: $C^{4}_{10}=210$.

Xác suất cần tính là: $P=\dfrac{1365+210} {12650}=\dfrac{63}{506}$. 
 }\end{ex}

\begin{ex}
 Cho bảng số liệu ghép nhóm về độ tuổi và số lượng khách hàng của một cửa hàng như sau:
 
\begin{center}\begin{tabular}{|c|c|c|c|c|c|c|}
        \hline
        Khoảng tuổi   & [24 ; 33) & [33 ; 42) & [42 ; 51) & [51 ; 60) & [60 ; 69)\\  
        \hline 
        Số người & 12 & 5 & 13 & 14 & 12 \\ 
        \hline 
    \end{tabular}
\end{center}
 Tính giá trị đại diện của nhóm $[51;60)$.
\choice
{ $ {60}$ }
   { $ {51}$ }
     { $ {27,75}$ }
    { \True ${55,5}$ }
\loigiai{ 

 Giá trị đại diện của nhóm là ${55,5}$. 
 }\end{ex}

\begin{ex}
 Cho mẫu số liệu ghép nhóm về quãng đường chạy bộ(đơn vị: km) và số ngày chạy bộ như sau:
 
\begin{center}\begin{tabular}{|c|c|c|c|c|c|c|}
        \hline
        Quãng đường chạy bộ(đơn vị: km)   & [1 ; 2) & [2 ; 3) & [3 ; 4) & [4 ; 5) & [5 ; 6) & [6 ; 7)\\  
        \hline 
        Số ngày chạy bộ & 1 & 3 & 8 & 5 & 5 & 2 \\ 
        \hline 
    \end{tabular}
\end{center}
 Tính quãng đường chạy bộ(đơn vị: km) trung bình từ mẫu số liệu ghép nhóm trên.
\choice
{ $ {2,67}$ }
   { \True ${4,17}$ }
     { $ {4,00}$ }
    { $ {3,67}$ }
\loigiai{ 

 Các giá trị đại diện của mẫu số liệu là: 1,5; 2,5; 3,5; 4,5; 5,5; 6,5

Tổng tần số là: $n=24$

Số trung bình của mẫu số liệu ghép nhóm là:

$\overline{x}=\dfrac{1,5.1+2,5.3+3,5.8+4,5.5+5,5.5+6,5.2}{ 24}=4,17$.

 
 }\end{ex}

\Closesolutionfile{ans}
{\bf PHẦN II. Câu trắc nghiệm đúng sai.}
\setcounter{ex}{0}
\Opensolutionfile{ans}[ans/ans001-2]
\begin{ex}
 Cho hàm số $y=7^{x}$. Xét tính đúng sai của các khẳng định sau



\choiceTF
{ Hàm số đã cho có tập xác định là $(0;+\infty)$ }
   { Đồ thị hàm số đã cho luôn nằm bên phải trục tung }
     { \True Hàm số đã cho đồng biến trên $\mathbb{R}$ }
    { Hàm số không liên tục trên $\mathbb{R}$ }
\loigiai{ 
 a) Hàm số đã cho có tập xác định là $(0;+\infty)$ là khẳng định sai. Hàm số đã cho có tập giá trị là $\mathbb{R}$.\\ 
b) Đồ thị hàm số đã cho luôn nằm phía dưới trục hoành là khẳng định sai.\\ 
c) Vì ${7}>1$ nên hàm số đã cho đồng biến trên $\mathbb{R}$.\\ 
d) Hàm số không liên tục trên $\mathbb{R}$ là khẳng định sai.\\ 
 
 }\end{ex}

\begin{ex}
 Xét tính đúng-sai của các khẳng định sau. 
\choiceTFt
{ \True  Hàm số $y=e^x$ là hàm số mũ }
   { Tập xác định của hàm số $y=\ln {3x}$ là $D=\mathbb{R}\backslash \{0\}$ }
     { \True  Hàm số $y=\log_{\frac{2}{5}}x$ nghịch biến trên $(0;+\infty)$ }
    { \True  Hàm số $y=\log_{2029} (3 x^{2} - 30 x)$ có tập xác định là $D=(-\infty;0)\cup (10;+\infty)$ }
\loigiai{ 
 

 a) Khẳng định đã cho là khẳng định đúng.

 Hàm số $e^x$ là hàm số mũ.

b) Khẳng định đã cho là khẳng định sai.

 Tập xác định của hàm số $y=\ln {3x}$ là $D=(0;+\infty)$.

c) Khẳng định đã cho là khẳng định đúng.

 Hàm số $y=\log_{\frac{2}{5}}x$ nghịch biến trên $(0;+\infty)$.

d) Khẳng định đã cho là khẳng định đúng.

 Điều kiện xác định: $3 x^{2} - 30 x>0 \Leftrightarrow x<0$ hoặc $x>10$.

Tập xác định: $D=(-\infty;0)\cup (10;+\infty)$.

 
 }\end{ex}

\begin{ex}
 Cho đồ thị hàm số $y=a^x$ với $a>0$ có đồ thị như hình vẽ. Xét tính đúng sai của các khẳng định sau



\begin{center}
\begin{tikzpicture}[>=stealth] 
        \draw[->] (-2.5,0) -- (2.5,0) node[below] {$x$}; 
        \draw[->] (0,-1) -- (0,7) node[left] {$y$}; 
        \draw[fill=black] (0,0) node[below right]{$O$} circle (1pt);
        \draw (0,1) node[right]{$1$}; 
        \draw (1,0) node[below]{$1$}; 
        \draw (0,0.16666666666666666) node[left]{\footnotesize$\frac{1}{6}$}; 
        \tkzDefPoints{1/0.16666666666666666/A} 
        \tkzDrawPoints[fill=black](A) 
        \draw [dashed] (0,0.16666666666666666)--(1,0.16666666666666666)--(1,0); 
        \begin{scope}
         \clip (-2,-0.5) rectangle (2,7);
        \draw[color=blue,very thick,smooth,samples=200, domain=-2.01:2] plot (\x,{(6)^(-\x)}); 
        \end{scope}
        \end{tikzpicture} 

\end{center}

\choiceTF
{ \True Hàm số đã cho có tập giá trị là $\mathbb{R}$ }
   { Hàm số chỉ liên tục trên khoảng $(-7; +\infty)$ }
     { \True Đồ thị hàm số đã cho đi qua điểm ${(2;36)}$ }
    { Trên khoảng $(7;+\infty)$ thì hàm số đã cho đồng biến }
\loigiai{ 
 a) Hàm số đã cho có tập giá trị là $\mathbb{R}$ là khẳng định đúng\\ 
b) Hàm số chỉ liên tục trên khoảng $(-7; +\infty)$ là khẳng định sai. Hàm số đã cho liên tục trên $(0;+\infty)$.\\ 
c) Dựa vào đồ thị ta có $\displaystyle a=\frac{1}{6} \Rightarrow y=\left(\frac{1}{6}\right)^x$ nên đồ thị hàm số đi qua điểm ${(2;36)}$ là khẳng định đúng.\\ 
d) Trên khoảng $(7;+\infty)$ thì hàm số đã cho đồng biến là khẳng định sai.Trên khoảng $(7;+\infty)$ thì hàm số đã cho nghịch biến.\\ 
 
 }\end{ex}

\Closesolutionfile{ans}
{\bf PHẦN III. Câu trắc nghiệm trả lời ngắn.}
\setcounter{ex}{0}
\Opensolutionfile{ans}[ans/ans001-3]
\begin{ex}
 Cho mẫu số liệu ghép nhóm về điểm thi và số người dự thi như bảng sau. Tìm tứ phân vị thứ nhất ${Q_1}$ của mẫu số liệu ghép nhóm đã cho(kết quả làm tròn đến hàng phần mười).
\begin{center}\begin{tabular}{|c|c|c|c|c|c|c|}
        \hline
        Điểm thi   & [0 ; 3,5) & [3,5 ; 7) & [7 ; 10,5) & [10,5 ; 14) & [14 ; 17,5)\\  
        \hline 
        Số người dự thi & 15 & 19 & 14 & 3 & 9 \\ 
        \hline 
    \end{tabular}
\end{center}


\shortans[4]{3,5}

\loigiai{ 
 Bước 1: Tổng tần số là: $N=60$.

Bước 2: Xác định vị trí của $Q_1$: $Q_1$ nằm ở vị trí $\dfrac{60}{4}=15.0$.

Bước 3: Xác định lớp chứa $Q_1$: tính tần số tích lũy từ lớp đầu tiên đến khi đạt hoặc vượt qua vị trí của $Q_1$ ta được lớp $[0.0;3.5)$.

Bước 4: Xác định các thông số của công thức tính $Q_1$.

 Cận dưới của lớp chứa $Q_1$: $L=0.0$

 Tổng tần số của các lớp trước lớp chứa $Q_1$: $F=0$

 Tần số của lớp chứa $Q_1$: $f=15$.

 Độ rộng lớp chứa $Q_1$: $h=3.5 - 0.0=3.5$.

Áp dụng công thức: $Q_1=L+\left(\dfrac{ \dfrac{N}{4}-F }{f}\right).h=0.0+\left(\dfrac{ \dfrac{60}{4}-0 }{15}\right).3.5=3,5$. 
 }\end{ex}

\begin{ex}
 Cho mẫu số liệu ghép nhóm về quãng đường chạy bộ(đơn vị: km) và số ngày chạy bộ như bảng sau. Tìm tứ phân vị thứ hai ${Q_3}$ của mẫu số liệu ghép nhóm đã cho (kết quả làm tròn đến hàng phần mười).
\begin{center}\begin{tabular}{|c|c|c|c|c|c|c|}
        \hline
        Quãng đường chạy bộ(đơn vị: km)   & [1,2 ; 1,7) & [1,7 ; 2,2) & [2,2 ; 2,7) & [2,7 ; 3,2) & [3,2 ; 3,7)\\  
        \hline 
        Số ngày chạy bộ & 7 & 6 & 4 & 3 & 7 \\ 
        \hline 
    \end{tabular}
\end{center}


\shortans[4]{3,22}

\loigiai{ 
 Bước 1: Tổng tần số là: $N=27$.

Bước 2: Xác định vị trí của $Q_3$: $Q_3$ nằm ở vị trí $\dfrac{3.27}{4}=20.3$.

Bước 3: Xác định lớp chứa $Q_3$: tính tần số tích lũy từ lớp đầu tiên đến khi đạt hoặc vượt qua vị trí của $Q_3$ ta được lớp $[3.2;3.7)$.

Bước 4: Xác định các thông số của công thức tính $Q_3$.

 Cận dưới của lớp chứa $Q_3$: $L=3.2$

 Tổng tần số của các lớp trước lớp chứa $Q_3$: $F=20$

 Tần số của lớp chứa $Q_3$: $f=7$.

 Độ rộng lớp chứa $Q_3$: $h=3.7 - 3.2=0.5$.

Áp dụng công thức: $Q_3=L+\left(\dfrac{ \dfrac{3N}{4}-F }{f}\right).h=3.2+\left(\dfrac{ \dfrac{3.27}{4}-20 }{7}\right).0.5=3,22$. 
 }\end{ex}

\Closesolutionfile{ans}
{\bf PHẦN IV. Tự luận.}
\setcounter{ex}{0}
\Opensolutionfile{ans}[ans/ans001-4]
\begin{ex}
 Giải phương trình $\sqrt[7]{ {3}^{11 x} }-3^{131}=0$.
\loigiai{ 
 $\sqrt[7]{ {3}^{11 x} }-3^{131}=0$ $\Leftrightarrow 3^{\frac{11 x}{7}}=3^{131}$

$ \Leftrightarrow \frac{11 x}{7}=131 \Rightarrow x=\frac{917}{11}$.

 
 }\end{ex}

\begin{ex}
 Giải bất phương trình $\left(\frac{1}{8}\right)^{4 x^{2} + 28 x + 44}<\left(\frac{1}{8}\right)^{x^{2} + x + 2}$.
\loigiai{ 
 $\left(\frac{1}{8}\right)^{4 x^{2} + 28 x + 44}<\left(\frac{1}{8}\right)^{x^{2} + x + 2}$

$\Leftrightarrow 4 x^{2} + 28 x + 44>x^{2} + x + 2$

$\Leftrightarrow 3 x^{2} + 27 x + 42>0$

$\Leftrightarrow x<-7$ hoặc $x>-2$ 
 }\end{ex}

\Closesolutionfile{ans}

 \begin{center}
-----HẾT-----
\end{center}

 %\newpage 
%\begin{center}
%{\bf BẢNG ĐÁP ÁN MÃ ĐỀ 1 }
%\end{center}
%{\bf Phần 1 }
% \inputansbox{6}{ans001-1}
%{\bf Phần 2 }
% \inputansbox{2}{ans001-2}
%{\bf Phần 3 }
% \inputansbox{6}{ans001-3}
\newpage 
%{\bf Phần 4 }
% \inputansbox{6}{ans001-4}
\newpage 



\end{document}